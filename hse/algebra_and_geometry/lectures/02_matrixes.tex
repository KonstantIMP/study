% Лекция по Алгебре и геометрии 21.09.2022
% Михедов Константин Константиновчи, БИБ 224

% Тип документа: статья, размер бумаги - A4, написано 14 кегелем
% Предназначено для импорта из другого документа
\documentclass[class=article,a4paper,12pt,crop=false]{standalone}

% Поиск по скомпилированному PDF
\usepackage{cmap}
% Кодировка выходного текста
\usepackage[T2A]{fontenc}
% Кодировка исходного текста
\usepackage[utf8]{inputenc}
% Поддержка необходимых языков
\usepackage[english,russian]{babel}

% Поддержка изображений
\usepackage{graphicx}
% Путь до внешних изображений
\graphicspath{ {./figures/}}

% Умная запятая
\usepackage{icomma}

% Ссылки на электронные ресурсы
\usepackage{hyperref}
% Настройка внешнего вида ссылок
\hypersetup{
  % Отключить прямоугольную рамку
  pdfborder={0 0 0},
  % Включить цветные ссылки
  colorlinks=true,
  % Цвет для ссылок на веб-ресурсы
  urlcolor=blue,
  % Цвет внутренних ссылок
  linkcolor=black
}

% Дополнительная математика
\usepackage{amsmath,amsfonts,amssymb,amsthm,mathtools}
% Показывать номера только у тех выржений, на которые кто-то ссылается
\mathtoolsset{showonlyrefs=true}

% Дополнительные символы
\usepackage{mathbbol}

% Подключние пакетов для импорта других .tex
\usepackage[subpreambles=true]{standalone}
\usepackage{import}

\begin{document}
  
\subsubsection{Определитель n-ого порядка}

\begin{equation}
  \det{
    \begin{pmatrix}
      a_{11} & a_{12} & \dots & a_{1n} \\
      a_{21} & a_{22} & \dots & a_{2n} \\
      \vdots & \vdots & \ddots & \vdots \\
      a_{n1} & a_{n2} & \dots & a_{nn}
    \end{pmatrix}
  } = \sum\limits_{(\alpha_1, \alpha_2, \dots, \alpha_n)}{
    \left(a_{1\alpha_1} a_{2\alpha_2} \dots a_{n\alpha_n}
    (-1)^{f(\alpha_1, \alpha_2, \dots, \alpha_n)}\right)
  }
\end{equation}

\begin{equation}
  f(\alpha_1, \alpha_2, \dots, \alpha_n) = \begin{cases}
    1 & \text{ если } (\alpha_1, \alpha_2, \dots, \alpha_n) \text{ чётная} \\
    0 & \text{ в других случаях}
  \end{cases}
\end{equation}

\end{document}
