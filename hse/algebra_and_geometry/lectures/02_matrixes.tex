% Лекция по Алгебре и геометрии 21.09.2022
% Михедов Константин Константиновчи, БИБ 224

% Тип документа: статья, размер бумаги - A4, написано 14 кегелем
% Предназначено для импорта из другого документа
\documentclass[class=article,a4paper,12pt,crop=false]{standalone}

% Поиск по скомпилированному PDF
\usepackage{cmap}
% Кодировка выходного текста
\usepackage[T2A]{fontenc}
% Кодировка исходного текста
\usepackage[utf8]{inputenc}
% Поддержка необходимых языков
\usepackage[english,russian]{babel}

% Вращения текста
\usepackage{rotating}

% Поддержка изображений
\usepackage{graphicx}
% Путь до внешних изображений
\graphicspath{ {./figures/}}

% Умная запятая
\usepackage{icomma}

% Ссылки на электронные ресурсы
\usepackage{hyperref}
% Настройка внешнего вида ссылок
\hypersetup{
  % Отключить прямоугольную рамку
  pdfborder={0 0 0},
  % Включить цветные ссылки
  colorlinks=true,
  % Цвет для ссылок на веб-ресурсы
  urlcolor=blue,
  % Цвет внутренних ссылок
  linkcolor=black
}

% Дополнительная математика
\usepackage{amsmath,amsfonts,amssymb,amsthm,mathtools}
% Показывать номера только у тех выржений, на которые кто-то ссылается
\mathtoolsset{showonlyrefs=true}

% Дополнительные символы
\usepackage{mathbbol}

% Подключние пакетов для импорта других .tex
\usepackage[subpreambles=true]{standalone}
\usepackage{import}

% Правильное оформление
% Настройка отступов
\usepackage[left=2cm,right=1cm,top=2cm,bottom=2cm]{geometry}
% Настройка шрифта
\usepackage{fontspec}
\setmainfont{Times New Roman}
% Настройка межстрочных интервалов
\usepackage{setspace}
\onehalfspacing
% и межабзацных
\usepackage{parskip}
\setlength{\parindent}{1.25cm} 

% Красная строка
\usepackage{indentfirst}
\setlength{\parindent}{1.25cm} 

% Корректное положение рисунков?
\usepackage{float}

% Расположение блоков относительно текста
\usepackage{wrapfig}

\begin{document}
  
\subsubsection{Определитель n-ого порядка}

\begin{equation}
  \det{
    \begin{pmatrix}
      a_{11} & a_{12} & \dots & a_{1n} \\
      a_{21} & a_{22} & \dots & a_{2n} \\
      \vdots & \vdots & \ddots & \vdots \\
      a_{n1} & a_{n2} & \dots & a_{nn}
    \end{pmatrix}
  } = \sum\limits_{(\alpha_1, \alpha_2, \dots, \alpha_n)}{
    \left(a_{1\alpha_1} a_{2\alpha_2} \dots a_{n\alpha_n}
    (-1)^{f(\alpha_1, \alpha_2, \dots, \alpha_n)}\right)
  }
\end{equation}

\begin{equation}
  f(\alpha_1, \alpha_2, \dots, \alpha_n) = \begin{cases}
    1 & \text{ если } (\alpha_1, \alpha_2, \dots, \alpha_n) \text{ чётная} \\
    0 & \text{ в других случаях}
  \end{cases}
\end{equation}

\subsubsection{Свойства определителей}

\begin{enumerate}
  \item {
    Определители исходной и транспонированной матриц равны: 
    \begin{equation}
      \det{A}=\det{A^\tau}
    \end{equation}
  }
  \item {
    Если в исходной матрице попарно поменять местами 2 строки или 2
    столбца, то знак определителя поменяется на противоположный.
    %\begin{equation}
    %  \det{
    %    \begin{pmatrix}
    %      a_{11} & a_{12} & a_{13} \\
    %      a_{21} & a_{22} & a_{23} \\
    %      a_{31} & a_{32} & a_{33} 
    %    \end{pmatrix}
    %  } = -\det{
    %    \begin{pmatrix}
    %      a_{11} & a_{12} & a_{13} \\
    %      a_{31} & a_{32} & a_{33} \\
    %      a_{21} & a_{22} & a_{23} 
    %    \end{pmatrix}
    %  }
    %\end{equation}
  }
  \item {
    Если в матрице есть нулевой столбец или строка, то ее определитель равен нулю.
    %\begin{eqnarray}
    %  \begin{vmatrix}
    %    1 & 3 & -2 \\
    %    4 & 9 & 0 \\
    %    0 & 0 & 0
    %  \end{vmatrix} = 
    %  \begin{vmatrix}
    %    3 & 3 & 0 \\
    %    5 & 7 & 0 \\
    %    4 & -14 & 0
    %  \end{vmatrix} = 0
    %\end{eqnarray}
  }
  \item {
    Если в матрице есть два одинаковых столбца или строки, то ее определитель равен нулю.
  }
  \item {
    Если в $\Delta$ $\exists$ строка или столбец: $\forall i,j : a_{ij} = \lambda a^{'}_{ij}$, то $\Delta = \lambda \Delta^{'}$
    \begin{equation}
      \det{
        \begin{pmatrix}
          4 & 8 & 14 \\
          5 & 6 & 1 \\
          10 & 7 & 0
        \end{pmatrix}
      } = 2 \times \det{
        \begin{pmatrix}
          2 & 3 & 7 \\
          5 & 6 & 1 \\
          10 & 7 & 0
        \end{pmatrix}
      }
    \end{equation}
  }
  \item {
    Свойсто, название которому не придумали
    %\begin{eqnarray}
      \begin{equation}
      %\det{
        \begin{vmatrix}
          a_{11} & \dots & a_{1j} + b_{1j} & \dots & a_{1n} \\
          \vdots & \ddots & \vdots & \ddots & \vdots \\
          a_{n1} & \dots & a_{nj} + b_{nj} & \dots & a_{nn} \\
        \end{vmatrix}
      %}
      = %\det{
        \begin{vmatrix}
          a_{11} & \dots & a_{1j} & \dots & a_{1n} \\
          \vdots & \ddots & \vdots & \ddots & \vdots \\
          a_{n1} & \dots & a_{nj} & \dots & a_{nn} \\
        \end{vmatrix}
      %} + \det{
       +
        \begin{vmatrix}
          a_{11} & \dots & b_{1j} & \dots & a_{1n} \\
          \vdots & \ddots & \vdots & \ddots & \vdots \\
          a_{n1} & \dots & b_{nj} & \dots & a_{nn} \\
        \end{vmatrix}
      %}
      \end{equation}
    %\end{eqnarray}
  }
  \item {
    Включение линейной комбинации
    \begin{equation}
      \begin{vmatrix}
        a_{11} & a_{12} & \dots & a_{1j} & \dots & a_{1n} \\
        a_{21} & a_{22} & \dots & a_{2j} & \dots & a_{2n} \\
        \vdots & \vdots & \ddots & \vdots & \ddots & \vdots \\
        a_{n1} & a_{n2} & \dots & a_{nj} & \dots & a_{jj} \\
      \end{vmatrix}
      =
      \begin{vmatrix}
        a_{11} + \lambda a_{1j} & a_{12} & \dots & a_{1j} & \dots & a_{1n} \\
        a_{21} + \lambda a_{2j} & a_{22} & \dots & a_{2j} & \dots & a_{2n} \\
        \vdots & \vdots & \ddots & \vdots & \ddots & \vdots \\
        a_{n1} + \lambda a_{nj} & a_{n2} & \dots & a_{nj} & \dots & a_{jj} \\
      \end{vmatrix}
    \end{equation}
  }
\end{enumerate}

\subsection{Миноры и алгебраические дополнения}

\subsubsection{Для одного элемента}

\paragraph{Минор} определитель некоторой меньшей квадратной матрицы, вырезанной из заданной матрицы путем удаления одной/нескольких строк и столбцов
\paragraph{Дополнительный минор} $n$-ого порядка - определитель порядка $n - 1$, полученный вычеркиванием $i$ строки и $j$ столбца
\paragraph{Алгебраическое дополнение} $A_{ij} = (-1)^{i + j}M_{ij}$ где $M_{ij}$ - дополнительный минор

\subsubsection{Общий случай}

\textbf{Эта штука выглядит максимально страшно...}

\begin{picture}(300,200)

  \put(0,100){ $\begin{pmatrix}
    a_{11} & a_{12} & \dots &
    a_{1j_1} & a_{1j_2} & \dots &
    a_{1j_{k}} & a_{1j_{k+1}} & \dots & a_{1n} \\
    a_{21} & a_{22} & \dots &
    a_{2j_1} & a_{2j_2} & \dots &
    a_{2j_{k}} & a_{2j_{k+1}} & \dots & a_{2n} \\
    \vdots & \vdots & \ddots &
    \vdots & \vdots & \ddots &
    \vdots & \vdots & \ddots & \vdots \\
    a_{i_11} & a_{i_12} & \dots &
    a_{i_1j_1} & a_{i_1j_2} & \dots &
    a_{i_1j_{k}} & a_{i_1j_{k+1}} & \dots & a_{i_1n} \\
    a_{i_21} & a_{i_22} & \dots &
    a_{i_2j_1} & a_{i_2j_2} & \dots &
    a_{i_2j_{k}} & a_{i_2j_{k+1}} & \dots & a_{i_2n} \\
    \vdots & \vdots & \ddots &
    \vdots & \vdots & \ddots &
    \vdots & \vdots & \ddots & \vdots \\
    a_{i_{k}1} & a_{i_{k}2} & \dots &
    a_{i_{k}j_1} & a_{i_{k}j_2} & \dots &
    a_{i_{k}j_{k}} & a_{i_{k}j_{k+1}} & \dots & a_{i_{k}n} \\
    a_{i_{k+1}1} & a_{i_{k+1}2} & \dots &
    a_{i_{k+1}j_1} & a_{i_{k+1}j_2} & \dots &
    a_{i_{k+1}j_{k}} & a_{i_{k+1}j_{k+1}} & \dots & a_{i_{k+1}n} \\
    \vdots & \vdots & \ddots &
    \vdots & \vdots & \ddots &
    \vdots & \vdots & \ddots & \vdots \\
    a_{n1} & a_{n2} & \dots &
    a_{nj_1} & a_{nj_2} & \dots &
    a_{nj_{k}} & a_{nj_{k+1}} & \dots & a_{nn} 
  \end{pmatrix}$ }

  \put(112,190){\color{magenta}\line(0,-1){180}}
  \put(260,190){\color{magenta}\line(0,-1){180}}

  \put(0,135){\color{magenta}\line(1,0){385}}
  \put(0,70){\color{magenta}\line(1,0){385}}

  
  \put(186,102){\color{cyan}\oval(145, 65)}

  \put(60,160){\color{green}\oval(100, 45)[br]}
  \put(60,45){\color{green}\oval(100, 45)[tr]}
  \put(312,160){\color{green}\oval(100, 45)[bl]}
  \put(312,45){\color{green}\oval(100, 45)[tl]}

\end{picture}

$M_k$ - минор k-ого порядка, $M_k^{'}$ - его дополнительный минор а $A_k$ - алгебраическое дополнение

\begin{equation}
  M_k = \det{
    \begin{pmatrix}
      a_{i_1j_1} & a_{i_1j_2} & \dots & a_{i_1j_k} \\
      a_{i_2j_1} & a_{i_2j_2} & \dots & a_{i_2j_k} \\
      \vdots & \vdots & \ddots & \vdots \\
      a_{i_kj_1} & a_{i_kj_2} & \dots & a_{i_kj_k}
    \end{pmatrix}
  }
\end{equation}
\begin{equation}
  A_k = (-1)^{S_m}M_k^{'} = (-1)^{S_m}\det{
    \begin{pmatrix}
      a_{11} & a_{12} & \dots & a_{1,j_1-1} & a_{1j_{k+1}} & \dots & a_{1n} \\
      a_{21} & a_{22} & \dots & a_{2,j_1-1} & a_{2j_{k+1}} & \dots & a_{2n} \\
      \vdots & \vdots & \ddots & \vdots & \vdots & \ddots & \vdots \\
      a_{i_1-1,1} & a_{i_1-1,2} & \dots & a_{i_1-1,j_1-1} & a_{i_1-1,j_{k+1}} & \dots & a_{i_1-1,n} \\
      a_{i_{k+1}1} & a_{i_{k+1}2} & \dots & a_{i_{k+1},j_1-1} & a_{i_{k+1}j_{k+1}} & \dots & a_{i_{k+1}n} \\
      \vdots & \vdots & \ddots & \vdots & \vdots & \ddots & \vdots \\
      a_{n1} & a_{n2} & \dots & a_{n,j_1-1} & a_{nj_{k+1}} & \dots & a_{nn}
    \end{pmatrix}
  }
\end{equation}

И еще кое-что: $S_m = (i_1+i_2+\dots+i_k) \times (j_1+j_2+\dots+j_k)$

\subsubsection{Теорема о произведении минора на его алгебраическое дополнение}

Произведение минора $k$-ого порядка на его алгебраическое дополнение равно сумме некоторого числа членов определителя исходной матрицы

\begin{wrapfigure}{l}{0.25\textwidth}
  \centering
  \begin{picture}(70, 70)
    \put(-20,-20){\line(1,0){100}}
    \put(-20,80){\line(1,0){100}}
    \put(80,-20){\line(0,1){100}}
    \put(-20,-20){\line(0,1){100}}

    \put(-20,30){\line(1,0){100}}
    \put(30,-20){\line(0,1){100}}

    \put(-10,-20){\line(0,1){50}}
    \put(0,-20){\line(0,1){50}}
    \put(10,-20){\line(0,1){50}}
    \put(20,-20){\line(0,1){50}}

    \put(30,40){\line(1,0){50}}
    \put(30,50){\line(1,0){50}}
    \put(30,60){\line(1,0){50}}
    \put(30,70){\line(1,0){50}}

    \put(47, 2){$M_k^{'}$}
    \put(-3, 52){$M_k$}

    \put(-19.5, 82.5){$1 \: 2 \: \dots \: k$}

    \put(-35,77.5){
      \begin{turn}{-90}
        $1 \: 2 \: \dots \: k$
      \end{turn}
    }

    \put(72.5, 82.5){$n$}
    \put(-32.5, -12.5){
      \begin{turn}{-90}
        $n$
      \end{turn}
    }
  \end{picture}
\end{wrapfigure}

\paragraph{Доказательство. Частный случай} пусть минор $M_k$ располагается в верхнем левом углу матрицы и занимает $k$ строк и $k$ столбцов ($1, 2, \dots, k$),
тогда его дополнительный минор будет занимать строки и столбцы с номерами: $k+1, k+2, \dots, n$

Тогда алгебраическое дополнение $A_k = (-1)^{S_m}M_k^{'} = M_k^{'}$, так как в данном случае $S_m = (1 + 2 + \dots
+ k) + (1 + 2 + \dots + k) = 2\cdot (1 + 2 + \dots + k)$, следовательно $(-1)^{S_m} = ((-1)^2)^{1 + 2 + \dots + k}
= 1^{1 + 2 + \dots + k} = 1$

Получаем, что $M_k\cdot A_k = M_k \cdot M_k^{'}$, теперь выберем произвольный член минора $M_k$ и 
произвольный член дополнительного минора: 
$(-1)^{\sigma}a_{1\alpha_1}a_{2\alpha_2}\dots a_{k\alpha_k}$ и
$(-1)^{\sigma^{*}}a_{k+1,\beta_{k+1}}a_{k+2,\beta_{k+2}}\dots a_{n,\beta_{n}}$

Здесь $\sigma$ - количество инверсий в подстановке $\begin{pmatrix}
  1 & 2 & \dots & k\\
  \alpha_1 & \alpha_2 & \dots & \alpha_k
\end{pmatrix}$, а $\sigma^{*}$ - количество инверсий в подстановке $\begin{pmatrix}
  k + 1 & k + 2 & \dots & n\\
  \beta{k + 1} & \beta_{k + 2} & \dots & \beta_{n}
\end{pmatrix}$ (по определению алгебраического дополнения)

Перемножим полученные выражения:
\begin{multline}
  ((-1)^{\sigma}a_{1\alpha_1}a_{2\alpha_2}\dots a_{k\alpha_k}) \cdot ((-1)^{\sigma^{*}}a_{k+1,\beta_{k+1}}a_{k+2,\beta_{k+2}}\dots a_{n,\beta_{n}}) = \\
  = (-1)^{\sigma}(-1)^{\sigma^{*}}a_{1\alpha_1}a_{2\alpha_2}\dots a_{k\alpha_k}a_{k+1,\beta_{k+1}}a_{k+2,\beta_{k+2}}\dots a_{n,\beta_{n}} = \\
  = (-1)^{\sigma + \sigma^{*}}a_{1\alpha_1}a_{2\alpha_2}\dots a_{k\alpha_k}a_{k+1,\beta_{k+1}}a_{k+2,\beta_{k+2}}\dots a_{n,\beta_{n}} 
\end{multline}

Полученное произведение состоит из $n$ элементов, каждый из которых расположен в различных строках и столбцах
исходного определителя $\Rightarrow$ это произведение является членом исходного определителя

Очевидно, что $\sigma_k = \sigma + \sigma^{*}$ и численно равна количеству инверсий в подстановке:
\begin{eqnarray}
  \begin{pmatrix}
    1 & 2 & \dots & k & k + 1 & k + 2 & \dots & n\\
    \alpha_1 & \alpha_2 & \dots & \alpha_k & \beta{k + 1} & \beta_{k + 2} & \dots & \beta_{n}
  \end{pmatrix}
\end{eqnarray}
что численно будет эквивалентно $S_m$ (как в формуле исходного определителя)

\paragraph{Доказательство. Общий случай} пусть минор $M_k$ расположен в строках с номерами
$i_1, i_2, \dots, i_k$ и столбцах $j_1, j_2, \dots, j_k$, причем $i_1 < i_2 < \dots < i_k$ и
$j_1 < j_2 < \dots < j_k$, тогда при помощи свойств определителя, можно свести задачу к базовому случаю:

\begin{figure}[H]
  \centering
  \begin{picture}(280, 100)
    \put(0,0){\line(1,0){102}}
    \put(0,102){\line(1,0){102}}
    \put(102,0){\line(0,1){102}}
    \put(0,0){\line(0,1){102}}

    \put(0,34){\line(1,0){34}}
    \put(0,68){\line(1,0){34}}
    \put(68,34){\line(1,0){34}}
    \put(68,68){\line(1,0){34}}

    \put(34,0){\line(0,1){34}}
    \put(34,68){\line(0,1){34}}
    \put(68,68){\line(0,1){34}}
    \put(68,0){\line(0,1){34}}

    \multiput(7,40)(0,11){3}{
      \multiput(0,0)(11,0){3}{
        \put(0,0){\circle*{2}}
      }
    }

    \multiput(34,0)(5,0){7}{
      \put(0,0){\line(0,1){34}}
    }

    \multiput(68,34)(0,5){7}{
      \put(0,0){\line(1,0){34}}
    }

    \multiput(38,38)(11,0){3}{
      \multiput(0,0)(0,11){3}{
        \put(0,0){\line(1,0){3}}
        \put(0,3){\line(1,0){3}}
        \put(0,0){\line(0,1){3}}
        \put(3,0){\line(0,1){3}}
      }
    }

    \multiput(36.5,73)(11,0){3}{
      \multiput(0,0)(0,11){3}{
        \circle{4}
      }
    }

    \put(135,47.5){$\Rightarrow$}

    \put(180,0){\line(1,0){102}}
    \put(180,102){\line(1,0){102}}
    \put(282,0){\line(0,1){102}}
    \put(180,0){\line(0,1){102}}

    \put(214,0){\line(0,1){68}}
    \put(214,68){\line(1,0){68}}

    \put(12.5,80){$1$}
    \put(82.5,80){$2$}
    \put(12.5,12.5){$3$}
    \put(82.5,12.5){$4$}

    \put(226.5,46){$1$}
    \put(262.5,46){$2$}
    \put(226.5,12.5){$3$}
    \put(262.5,12.5){$4$}

    \multiput(184,72)(11,0){3}{
      \multiput(0,0)(0,11){3}{
        \put(0,0){\line(1,0){3}}
        \put(0,3){\line(1,0){3}}
        \put(0,0){\line(0,1){3}}
        \put(3,0){\line(0,1){3}}
      }
    }

    \multiput(220,74)(0,11){3}{
      \multiput(0,0)(11,0){3}{
        \put(0,0){\circle*{2}}
      }
    }

    \multiput(180,0)(5,0){7}{
      \put(0,0){\line(0,1){34}}
    }

    \multiput(248,68)(0,5){7}{
      \put(0,0){\line(1,0){34}}
    }

    \multiput(182.5,39)(11,0){3}{
      \multiput(0,0)(0,11){3}{
        \circle{4}
      }
    }
  \end{picture}
\end{figure}

Для данного перехода потребуется $(i_1 - 1) + (i_2 - 2) + \dots + (i_k - k)$ перемещений строк и
$(j_1 - 1) + (j_2 - 2) + \dots + (j_k - k)$ перемещений столбцов, то есть всего:
\begin{eqnarray}
  (i_1 + i_2 + \dots + i_k + j_1 + j_2 + \dots + j_k) - 2(1 + 2 + \dots + k) = S_m - 2(1 + 2 + \dots + k)
\end{eqnarray}

Из-за того, что $(-1)^{S_m - 2(1 + 2 + \dots + k)} = \frac{(-1)^{S_m}}{(-1)^{2(1 + 2 + \dots + k)}}
=\frac{(-1)^{S_m}}{1^{1 + 2 + \dots + k}} = (-1)^{S_m}$, знак получается такой же, какой был бы
в выражении исходного определителя $(\blacksquare)$

\subsubsection{Зачем все это?}

\paragraph{Для вычисления определителя}

При помощи алгебраических дополнений можно несколько проще считать определители для
матриц больших размерностей:

\begin{equation}
  \Delta = \sum\limits_{j=1}^{n}(-1)^{k+j}M_{kj}A_{kj}
\end{equation}

В данной формуле представлено разложение по строке (с номером k), но такое можно
делать и для столбцов. Также $M_{kj}$ - минор из одного элемента с координатами $(k; j)$,
а $A_{kj}$ - его элгебраическое дополнение.

\subsubsection{Теорема Лапласа}

Пусть выбраны любые $k$ строк матрицы $A$, тогда определитель этой матрицы равен сумме всевозможных
произведений миноров $k$-ого порядка на иъ алгебраические дополнения:
\begin{eqnarray}
  |A| = \sum\limits_{j_1 < \dots < j_k}{M_{j_1,\dots ,j_2}^{i_1,\dots ,i_k}A_{j_1,\dots ,j_2}^{i_1,\dots ,i_k}}
\end{eqnarray}

В данной формуле опять же представлено разложение по столбцам, которые всегда можно заменить на строки.
Данная теорема вытекает из множественного применения теоремы о произведении минора на его алгебраическое
дополнение

\paragraph{Доказательство} все слагаемые из правой части содержаться в выражении исходного определителя,
а так как все элементы в каждом слагаемом различны и эти слагаемые не повторяются, то теорема верна $(\blacksquare)$

\end{document}
