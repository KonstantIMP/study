% Лекция по Алгебре и геометрии 21.09.2022
% Михедов Константин Константиновчи, БИБ 224

% Тип документа: статья, размер бумаги - A4, написано 14 кегелем
% Предназначено для импорта из другого документа
\documentclass[class=article,a4paper,12pt,crop=false]{standalone}

% Поиск по скомпилированному PDF
\usepackage{cmap}
% Кодировка выходного текста
\usepackage[T2A]{fontenc}
% Кодировка исходного текста
\usepackage[utf8]{inputenc}
% Поддержка необходимых языков
\usepackage[english,russian]{babel}

% Поддержка изображений
\usepackage{graphicx}
% Путь до внешних изображений
\graphicspath{ {./figures/}}

% Умная запятая
\usepackage{icomma}

% Ссылки на электронные ресурсы
\usepackage{hyperref}
% Настройка внешнего вида ссылок
\hypersetup{
  % Отключить прямоугольную рамку
  pdfborder={0 0 0},
  % Включить цветные ссылки
  colorlinks=true,
  % Цвет для ссылок на веб-ресурсы
  urlcolor=blue,
  % Цвет внутренних ссылок
  linkcolor=black
}

% Дополнительная математика
\usepackage{amsmath,amsfonts,amssymb,amsthm,mathtools}
% Показывать номера только у тех выржений, на которые кто-то ссылается
\mathtoolsset{showonlyrefs=true}

% Дополнительные символы
\usepackage{mathbbol}

% Подключние пакетов для импорта других .tex
\usepackage[subpreambles=true]{standalone}
\usepackage{import}

\begin{document}
  
\subsubsection{Определитель n-ого порядка}

\begin{equation}
  \det{
    \begin{pmatrix}
      a_{11} & a_{12} & \dots & a_{1n} \\
      a_{21} & a_{22} & \dots & a_{2n} \\
      \vdots & \vdots & \ddots & \vdots \\
      a_{n1} & a_{n2} & \dots & a_{nn}
    \end{pmatrix}
  } = \sum\limits_{(\alpha_1, \alpha_2, \dots, \alpha_n)}{
    \left(a_{1\alpha_1} a_{2\alpha_2} \dots a_{n\alpha_n}
    (-1)^{f(\alpha_1, \alpha_2, \dots, \alpha_n)}\right)
  }
\end{equation}

\begin{equation}
  f(\alpha_1, \alpha_2, \dots, \alpha_n) = \begin{cases}
    1 & \text{ если } (\alpha_1, \alpha_2, \dots, \alpha_n) \text{ чётная} \\
    0 & \text{ в других случаях}
  \end{cases}
\end{equation}

\subsubsection{Свойства определителей}

\begin{enumerate}
  \item {
    Определители исходной и транспонированной матриц равны: 
    \begin{equation}
      \det{A}=\det{A^\tau}
    \end{equation}
  }
  \item {
    Если в исходной матрице попарно поменять местами 2 строки или 2
    столбца, то знак определителя поменяется на противоположный.
    %\begin{equation}
    %  \det{
    %    \begin{pmatrix}
    %      a_{11} & a_{12} & a_{13} \\
    %      a_{21} & a_{22} & a_{23} \\
    %      a_{31} & a_{32} & a_{33} 
    %    \end{pmatrix}
    %  } = -\det{
    %    \begin{pmatrix}
    %      a_{11} & a_{12} & a_{13} \\
    %      a_{31} & a_{32} & a_{33} \\
    %      a_{21} & a_{22} & a_{23} 
    %    \end{pmatrix}
    %  }
    %\end{equation}
  }
  \item {
    Если в матрице есть нулевой столбец или строка, то ее определитель равен нулю.
    %\begin{eqnarray}
    %  \begin{vmatrix}
    %    1 & 3 & -2 \\
    %    4 & 9 & 0 \\
    %    0 & 0 & 0
    %  \end{vmatrix} = 
    %  \begin{vmatrix}
    %    3 & 3 & 0 \\
    %    5 & 7 & 0 \\
    %    4 & -14 & 0
    %  \end{vmatrix} = 0
    %\end{eqnarray}
  }
  \item {
    Если в матрице есть два одинаковых столбца или строки, то ее определитель равен нулю.
  }
  \item {
    Если в $\Delta$ $\exists$ строка или столбец: $\forall i,j : a_{ij} = \lambda a^{'}_{ij}$, то $\Delta = \lambda \Delta^{'}$
    \begin{equation}
      \det{
        \begin{pmatrix}
          4 & 8 & 14 \\
          5 & 6 & 1 \\
          10 & 7 & 0
        \end{pmatrix}
      } = 2 \times \det{
        \begin{pmatrix}
          2 & 3 & 7 \\
          5 & 6 & 1 \\
          10 & 7 & 0
        \end{pmatrix}
      }
    \end{equation}
  }
  \item {
    Свойсто, название которому не придумали
    %\begin{eqnarray}
      \begin{multline}
      %\det{
        \begin{vmatrix}
          a_{11} & \dots & a_{1j} + b_{1j} & \dots & a_{1n} \\
          \vdots & \ddots & \vdots & \ddots & \vdots \\
          a_{n1} & \dots & a_{nj} + b_{nj} & \dots & a_{nn} \\
        \end{vmatrix}
      %}
      = \\ = %\det{
        \begin{vmatrix}
          a_{11} & \dots & a_{1j} & \dots & a_{1n} \\
          \vdots & \ddots & \vdots & \ddots & \vdots \\
          a_{n1} & \dots & a_{nj} & \dots & a_{nn} \\
        \end{vmatrix}
      %} + \det{
       +
        \begin{vmatrix}
          a_{11} & \dots & b_{1j} & \dots & a_{1n} \\
          \vdots & \ddots & \vdots & \ddots & \vdots \\
          a_{n1} & \dots & b_{nj} & \dots & a_{nn} \\
        \end{vmatrix}
      %}
      \end{multline}
    %\end{eqnarray}
  }
  \item {
    Включение линейной комбинации
    \begin{equation}
      \begin{vmatrix}
        a_{11} & a_{12} & \dots & a_{1j} & \dots & a_{1n} \\
        a_{21} & a_{22} & \dots & a_{2j} & \dots & a_{2n} \\
        \vdots & \vdots & \ddots & \vdots & \ddots & \vdots \\
        a_{n1} & a_{n2} & \dots & a_{nj} & \dots & a_{jj} \\
      \end{vmatrix}
      =
      \begin{vmatrix}
        a_{11} + \lambda a_{1j} & a_{12} & \dots & a_{1j} & \dots & a_{1n} \\
        a_{21} + \lambda a_{2j} & a_{22} & \dots & a_{2j} & \dots & a_{2n} \\
        \vdots & \vdots & \ddots & \vdots & \ddots & \vdots \\
        a_{n1} + \lambda a_{nj} & a_{n2} & \dots & a_{nj} & \dots & a_{jj} \\
      \end{vmatrix}
    \end{equation}
  }
\end{enumerate}

\subsection{Миноры и алгебраические дополнения}

\paragraph{Минор} значения определителя матрицы, полученной на пересечении одного или нескольких строк и столбцов
\paragraph{Алгебраическое дополнение} значение определителя матрицы, полученный за счет вычеркивания строк и столбцов, образующих минор

\subsubsection{Общий случай}

\textbf{Эта штука выглядит максимально страшно...}

\begin{picture}(300,200)

  \put(0,100){ $\begin{pmatrix}
    a_{11} & a_{12} & \dots &
    a_{1j_1} & a_{1j_2} & \dots &
    a_{1j_{k}} & a_{1j_{k+1}} & \dots & a_{1n} \\
    a_{21} & a_{22} & \dots &
    a_{2j_1} & a_{2j_2} & \dots &
    a_{2j_{k}} & a_{2j_{k+1}} & \dots & a_{2n} \\
    \vdots & \vdots & \ddots &
    \vdots & \vdots & \ddots &
    \vdots & \vdots & \ddots & \vdots \\
    a_{i_11} & a_{i_12} & \dots &
    a_{i_1j_1} & a_{i_1j_2} & \dots &
    a_{i_1j_{k}} & a_{i_1j_{k+1}} & \dots & a_{i_1n} \\
    a_{i_21} & a_{i_22} & \dots &
    a_{i_2j_1} & a_{i_2j_2} & \dots &
    a_{i_2j_{k}} & a_{i_2j_{k+1}} & \dots & a_{i_2n} \\
    \vdots & \vdots & \ddots &
    \vdots & \vdots & \ddots &
    \vdots & \vdots & \ddots & \vdots \\
    a_{i_{k}1} & a_{i_{k}2} & \dots &
    a_{i_{k}j_1} & a_{i_{k}j_2} & \dots &
    a_{i_{k}j_{k}} & a_{i_{k}j_{k+1}} & \dots & a_{i_{k}n} \\
    a_{i_{k+1}1} & a_{i_{k+1}2} & \dots &
    a_{i_{k+1}j_1} & a_{i_{k+1}j_2} & \dots &
    a_{i_{k+1}j_{k}} & a_{i_{k+1}j_{k+1}} & \dots & a_{i_{k+1}n} \\
    \vdots & \vdots & \ddots &
    \vdots & \vdots & \ddots &
    \vdots & \vdots & \ddots & \vdots \\
    a_{n1} & a_{n2} & \dots &
    a_{nj_1} & a_{nj_2} & \dots &
    a_{nj_{k}} & a_{nj_{k+1}} & \dots & a_{nn} 
  \end{pmatrix}$ }

  \put(112,190){\color{magenta}\line(0,-1){180}}
  \put(260,190){\color{magenta}\line(0,-1){180}}

  \put(0,135){\color{magenta}\line(1,0){385}}
  \put(0,70){\color{magenta}\line(1,0){385}}

  
  \put(186,102){\color{cyan}\oval(145, 65)}

  \put(60,160){\color{green}\oval(100, 45)[br]}
  \put(60,45){\color{green}\oval(100, 45)[tr]}
  \put(312,160){\color{green}\oval(100, 45)[bl]}
  \put(312,45){\color{green}\oval(100, 45)[tl]}

\end{picture}

Тогда $M_k$ - минор k-ого порядка, а $A_k$ - его алгебраическое дополнение

\begin{equation}
  M_k = \det{
    \begin{pmatrix}
      a_{i_1j_1} & a_{i_1j_2} & \dots & a_{i_1j_k} \\
      a_{i_2j_1} & a_{i_2j_2} & \dots & a_{i_2j_k} \\
      \vdots & \vdots & \ddots & \vdots \\
      a_{i_kj_1} & a_{i_kj_2} & \dots & a_{i_kj_k}
    \end{pmatrix}
  }
\end{equation}
\begin{equation}
  A_k =\det{
    \begin{pmatrix}
      a_{11} & a_{12} & \dots & a_{1,j_1-1} & a_{1j_{k+1}} & \dots & a_{1n} \\
      a_{21} & a_{22} & \dots & a_{2,j_1-1} & a_{2j_{k+1}} & \dots & a_{2n} \\
      \vdots & \vdots & \ddots & \vdots & \vdots & \ddots & \vdots \\
      a_{i_1-1,1} & a_{i_1-1,2} & \dots & a_{i_1-1,j_1-1} & a_{i_1-1,j_{k+1}} & \dots & a_{i_1-1,n} \\
      a_{i_{k+1}1} & a_{i_{k+1}2} & \dots & a_{i_{k+1},j_1-1} & a_{i_{k+1}j_{k+1}} & \dots & a_{i_{k+1}n} \\
      \vdots & \vdots & \ddots & \vdots & \vdots & \ddots & \vdots \\
      a_{n1} & a_{n2} & \dots & a_{n,j_1-1} & a_{nj_{k+1}} & \dots & a_{nn}
    \end{pmatrix}
  }
\end{equation}

И еще кое-что: $S_m = (i_1+i_2+\dots+i_k) \times (j_1+j_2+\dots+j_k)$

\subsubsection{Зачем все это?}

\paragraph{Для вычисления определителя}

При помощи алгебраических дополнений можно несколько проще считать определители для
матриц больших размерностей:

\begin{equation}
  \Delta = \sum\limits_{j=1}^{n}(-1)^{k+j}M_{kj}A_{kj}
\end{equation}

В данной формуле представлено разложение по строке (с номером k), но такое можно
делать и для столбцов. Также $M_{kj}$ - минор из одного элемента с координатами $(k; j)$,
а $A_{kj}$ - его элгебраическое дополнение.

\end{document}
