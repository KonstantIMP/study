% Лекция по Алгебре и геометрии 04.10.2022
% Михедов Константин Константиновчи, БИБ 224

% Тип документа: статья, размер бумаги - A4, написано 14 кегелем
% Предназначено для импорта из другого документа
\documentclass[class=article,a4paper,12pt,crop=false]{standalone}

% Поиск по скомпилированному PDF
\usepackage{cmap}
% Кодировка выходного текста
\usepackage[T2A]{fontenc}
% Кодировка исходного текста
\usepackage[utf8]{inputenc}
% Поддержка необходимых языков
\usepackage[english,russian]{babel}

% Вращения текста
\usepackage{rotating}

% Поддержка изображений
\usepackage{graphicx}
% Путь до внешних изображений
\graphicspath{ {./figures/}}

% Умная запятая
\usepackage{icomma}

% Ссылки на электронные ресурсы
\usepackage{hyperref}
% Настройка внешнего вида ссылок
\hypersetup{
  % Отключить прямоугольную рамку
  pdfborder={0 0 0},
  % Включить цветные ссылки
  colorlinks=true,
  % Цвет для ссылок на веб-ресурсы
  urlcolor=blue,
  % Цвет внутренних ссылок
  linkcolor=black
}

% Дополнительная математика
\usepackage{amsmath,amsfonts,amssymb,amsthm,mathtools}
% Показывать номера только у тех выржений, на которые кто-то ссылается
\mathtoolsset{showonlyrefs=true}

% Дополнительные символы
\usepackage{mathbbol}

% Подключние пакетов для импорта других .tex
\usepackage[subpreambles=true]{standalone}
\usepackage{import}

% Правильное оформление
% Настройка отступов
\usepackage[left=2cm,right=1cm,top=2cm,bottom=2cm]{geometry}
% Настройка шрифта
\usepackage{fontspec}
\setmainfont{Times New Roman}
% Настройка межстрочных интервалов
\usepackage{setspace}
\onehalfspacing
% и межабзацных
\usepackage{parskip}
\setlength{\parindent}{1.25cm} 

% Красная строка
\usepackage{indentfirst}
\setlength{\parindent}{1.25cm} 

% Корректное положение рисунков?
\usepackage{float}

% Расположение блоков относительно текста
\usepackage{wrapfig}

% Ранг матрицы
\DeclareMathOperator{\rank}{rank}

\begin{document}

\begin{wrapfigure}{r}{0.4\textwidth}
    \centering
    \begin{picture}(180,140)
        \put(90,70){\oval(180, 140)}
        \put(90,80){\oval(170, 110)}
        \put(155,90){\oval(30, 80)}
        \put(72.5,90){\oval(125, 80)}
        \put(72.5,100){\oval(115, 55)}
        \put(72.5,110){\oval(105, 30)}

        \put(112.5, 105){$\mathbb{N}$}
        \put(112.5, 80){$\mathbb{Z}$}
        \put(112.5, 55){$\mathbb{Q}$}
        \put(152.5, 55){$\mathbb{I}$}
        \put(150, 32.5){$\mathbb{R}$}
        \put(150, 10){$\mathbb{C}$}
    \end{picture}
\end{wrapfigure}

$2 + x = 5$ $\Rightarrow$ $\mathbb{N}$ - натуральные числа

$3 + x = 1$ и $3 + x = 3$ $\Rightarrow$ $\mathbb{Z}$ - целые числа 

$3x + 4 = 8$ $\Rightarrow$ $\mathbb{Q}$ - рациональные числа 

$x^2 - 2 = 0$ $\Rightarrow$ $\mathbb{I}$ - иррациональные числа 

$\mathbb{N} \cup \mathbb{Z} \cup \mathbb{Q} \cup \mathbb{I} = \mathbb{R}$ - действительные числа 

$x^2 + 1 = 0$ $\Rightarrow$ $\mathbb{C}$ - комплексные числа

\paragraph{Фундаментальная идея} если корней нет, то почему бы их не придумать

\subsection{Мнимая единица}

Некая мнимая единица: $i = \sqrt{-1}$ (почти эквивалентно записи $i^2 = -1$)

\subsection{Операции над комплексными числами}

\begin{itemize}
    \item {
        Сложение и вычитание:
        $(a + bi) \pm (c + di) = (a \pm c) + (b \pm d)i$
    }
    \item {
        Умножение:
        $(a + bi)(c + di) = ac + adi + bci + bdi^2 = (ac - bd) + (ad + bc)i$
    }
    \item {
        Сопряжение: $z = a + bi$ $\Rightarrow$ $\overline{z} = a - bi$
    }
    \item {
        Деление:
        \begin{equation}
            \frac{a + bi}{c + di} = \frac{(a + bi)(c - di)}{(c + di)(c - di)} =
            \frac{ac - adi + bci -bdi^2}{c + d} = \frac{(ac + bd) + (bc - ad)i}{c + d}
        \end{equation}
    }
    \item {
        Взятие действительной части: $\Re{(a + bi)} = a$
    }
    \item {
        Взятие мнимой части $\Im{(a + bi)} = b$
    }
\end{itemize}

Определены также взятия корня $n$-ой степени и возведения в степень, но о них позже

\subsubsection{Геометрическое определение комплексного числа}

\begin{wrapfigure}{l}{0.3\textwidth}
    \centering
    \begin{picture}(120, 120)
        \put(0, 20){\vector(1,0){120}}
        \put(20, 0){\vector(0,1){120}}

        \put(5,110){$I$}
        \put(110,5){$R$}

        \put(20,20){\line(4,5){60}}
        \put(80,95){\circle*{4}}
        \put(85,90){$A$}

        \multiput(20,95)(10,0){6}{
            \put(0,0){\line(1,0){5}}
        }
        \multiput(80,20)(0,10){8}{
            \put(0,0){\line(0,1){5}}
        }

        \qbezier(35,20)(35,25)(28,30)
        \put(35,30){$\varphi$}

        \put(50,65){$r$}

        \put(80,10){$a_r$}
        \put(10,90){$a_i$}
    \end{picture}
\end{wrapfigure}

$A(a_r; a_i) = a_r + a_ii$

$r = \sqrt{a_i^2 + a_r^2}$ (по теореме Пифагора)

$\cos{\varphi} = \frac{a_r}{r}$ и $\sin{\varphi} = \frac{a_i}{r}$

$A = a_r + a_ii = r(\frac{a_r}{r} + \frac{a_i}{r}i) = r(\cos{\varphi} + i\sin\varphi)$

Это тригонометрическая форма комплексного числа.

Так как $e^{i\varphi} = (\cos{\varphi} + i\sin{\varphi})$, то $A = re^{i\varphi}$ (а это уже экспоненциальная форма тригонометрического числа)

В этих формах можно удобно умножать и делить комплексные числа:
\begin{equation}
    \begin{aligned}
        z_1 =& r_1(\cos{\varphi_1} + i\sin{\varphi_1}) & & z_2 = r_2(\cos{\varphi_2} + i\sin{\varphi_2}) \\
        z_1\cdot{z_2} =& r_1r_2(\cos{(\varphi_1 + \varphi_2)} + i\sin{(\varphi_1 + \varphi_2)}) & &
        \frac{z_1}{z_2} = \frac{r_1}{r_2}(\cos{(\varphi_1 - \varphi_2)} + i\sin{(\varphi_1 - \varphi_2)})
    \end{aligned}
\end{equation}

\subsection{Формула Муавра}

Применяется для возведения комплексного числа в степень:
\begin{equation}
    (r(\cos{\varphi} + i\sin{\varphi}))^n = r^n(\cos{(n\varphi) + i\sin{(n\varphi)}})
\end{equation}

\textbf{Доказательство} (по индукции)
\begin{enumerate}
    \item {
        Для $n = 2$:
        \begin{multline}
            \text{По определению: } (r(\cos{\varphi} + i\sin{\varphi}))^2 = r^2(\cos{\varphi} + i\sin{\varphi})^2 =\\
            = r^2(\cos^2\varphi + 2\sin{\varphi}\cos{\varphi} + \sin^2\varphi)
            = r^2((\cos^2\varphi + \sin^2\varphi) + 2\sin{\varphi}\cos{\varphi}) =\\
            = r^2(\cos{(2\varphi)} + \sin{(2\varphi)})
        \end{multline}     
    
        Что абсолютно удовлетворяет формуле Муавра $\Rightarrow$ она верна для $n = 2$
    }
    \item {
        Допустим, что формула верна для $n=k$
    }
    \item {
        Рассмотрим случай, когда $n = k + 1$
        \begin{multline}
            (r(\cos{\varphi} + i\sin{\varphi}))^{k + 1} = 
            (r(\cos{\varphi} + i\sin{\varphi}))^k\cdot (r(\cos{\varphi} + i\sin{\varphi})) = \\ =
            (r^k(\cos{(k\varphi)} + i\sin{(k\varphi)}))\cdot(r(\cos{\varphi} + i\sin{\varphi})) = \\ =
            r^{k + 1}(\cos{(k\varphi)} + i\sin{(k\varphi)})\cdot(\cos{\varphi} + i\sin{\varphi}) =\\
            = r^{k + 1}(\cos{(k\varphi)}\cos{\varphi} + i\sin{(n\varphi)}\cos{\varphi}
            +i\cos{(n\varphi)\cos{\varphi} - \sin{(k\varphi)\sin{\varphi}}}) =\\
            = r^{k + 1}(\cos{((k + 1)\varphi)} + i\sin{((k + 1)\varphi)}) (\blacksquare)
        \end{multline}
    }
\end{enumerate}

\subsubsection{Синус и косинус n-ого угла}

Пусть $z$ - комплексное число такое, что $|z| = 1$

Тогда $z = \cos{\varphi} + i\sin{\varphi}$,
тогда $z^n = 1^n(\cos{(n\varphi)} + i\sin{(n\varphi)}) = \cos{(n\varphi)} + i\sin{(n\varphi)}$,
или же $z^n = (\cos{\varphi} + i\sin{\varphi})^n$. Разложим второе равенство по формуле бинома Ньютона:
\begin{multline}
    (\cos{\varphi} + i\sin{\varphi})^n =
    C_n^0\cos^n\varphi + C_n^1\cos^{n-1}\varphi\cdot{i\sin{\varphi}} - C_n^2\cos^{n-1}\varphi\cdot\sin^2\varphi - \\
    - C_n^3\cos^{n-3}\varphi\cdot{i\sin^3\varphi} + C_n^4\cos^{n-4}\varphi\cdot{\sin^4\varphi} +
    C_n^5\cos^{n-5}\varphi\cdot{i\sin^5\varphi} - \dots = \\
    = \cos{(n\varphi)} + i\sin{(n\varphi)}
\end{multline}

Плюсы и минусы в разложении берутся исходя из возведения мнимой единицы в степень. В правой части
$\cos$ не умножается на $i$, а $\sin$ - умножается, поэтому:
\begin{equation}
    \cos{(n\varphi)} = C_n^0\cos^n\varphi - C_n^2\cos^{n-2}\varphi\cdot{\sin^2{\varphi}} +
    C_n^4\cos^{n-4}\varphi\cdot{\sin^4\varphi}
    - C_n^6\cos^{n-6}\varphi\cdot{\sin^6{\varphi}} + \dots
\end{equation}
\begin{equation}
    \sin{(n\varphi)} = C_n^1\cos^{n-1}\varphi\cdot\sin{\varphi} -
    C_n^3\cos^{n-3}\varphi\cdot{\sin^3{\varphi}} + C_n^5\cos^{n-5}\varphi\cdot{\sin^5{\varphi}} -
    \dots
\end{equation}

\subsection{Корень из комплексного числа}

$z_1 = \sqrt[n]{z_2} \Leftrightarrow z_1^n = z_2$, то есть
$r_1^n(\cos{(n\varphi_1)} + i\sin{(n\varphi_1)}) = r_2(\cos{\varphi_2} + i\sin{\varphi_2})$ $\Rightarrow$
\begin{equation}
\Rightarrow \begin{cases}
    r_2 = r_1^n \\
    n\varphi_1 = \varphi_2
\end{cases} \Rightarrow
\begin{cases}
    r_1 = \sqrt[n]{r_2}\\
    \varphi_1 = \frac{\varphi_2 + 2\pi{k}}{n}, k = 0, 1, \dots, n-1
\end{cases}
\end{equation}

\end{document}

