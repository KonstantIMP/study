% Лекция по Алгебре и геометрии 04.10.2022
% Михедов Константин Константиновчи, БИБ 224

% Тип документа: статья, размер бумаги - A4, написано 14 кегелем
% Предназначено для импорта из другого документа
\documentclass[class=article,a4paper,12pt,crop=false]{standalone}

% Поиск по скомпилированному PDF
\usepackage{cmap}
% Кодировка выходного текста
\usepackage[T2A]{fontenc}
% Кодировка исходного текста
\usepackage[utf8]{inputenc}
% Поддержка необходимых языков
\usepackage[english,russian]{babel}

% Вращения текста
\usepackage{rotating}

% Поддержка изображений
\usepackage{graphicx}
% Путь до внешних изображений
\graphicspath{ {./figures/}}

% Умная запятая
\usepackage{icomma}

% Ссылки на электронные ресурсы
\usepackage{hyperref}
% Настройка внешнего вида ссылок
\hypersetup{
  % Отключить прямоугольную рамку
  pdfborder={0 0 0},
  % Включить цветные ссылки
  colorlinks=true,
  % Цвет для ссылок на веб-ресурсы
  urlcolor=blue,
  % Цвет внутренних ссылок
  linkcolor=black
}

% Дополнительная математика
\usepackage{amsmath,amsfonts,amssymb,amsthm,mathtools}
% Показывать номера только у тех выржений, на которые кто-то ссылается
\mathtoolsset{showonlyrefs=true}

% Дополнительные символы
\usepackage{mathbbol}

% Подключние пакетов для импорта других .tex
\usepackage[subpreambles=true]{standalone}
\usepackage{import}

% Правильное оформление
% Настройка отступов
\usepackage[left=2cm,right=1cm,top=2cm,bottom=2cm]{geometry}
% Настройка шрифта
\usepackage{fontspec}
\setmainfont{Times New Roman}
% Настройка межстрочных интервалов
\usepackage{setspace}
\onehalfspacing
% и межабзацных
\usepackage{parskip}
\setlength{\parindent}{1.25cm} 

% Красная строка
\usepackage{indentfirst}
\setlength{\parindent}{1.25cm} 

% Корректное положение рисунков?
\usepackage{float}

% Расположение блоков относительно текста
\usepackage{wrapfig}

% Ранг матрицы
\DeclareMathOperator{\rank}{rank}

\begin{document}

\subsection{Линейное преобразование}

Преобразование $f$ называется линейным, если $\forall x, y, \alpha, \beta \in \mathbb{R}$ верно:
\begin{equation}
    f(\alpha\cdot x + \beta\cdot y) = \alpha f(x) + \beta f(y)
\end{equation}

Предположим, что $\exists$ $n$ переменных: $x_1, x_2, \dots, x_n \in \mathbb{R}$, тогда
$y = ax + b$ - линейная функция ($y$ - зависимый аргумент,  $x$ - независимый аргумент,
$a, b \in \mathbb{R}$)

\subsubsection{Отношение к матрицам}

Пусть $x_1, x_2, \dots, x_n \in \mathbb{R}$ и $y_1, y_2, \dots, y_n \in \mathbb{R}$ и есть некоторое
линейное преобразование вида:
\begin{equation}
    \begin{cases}
        y_1 = a_{11}x_1 + a_{12}x_2 + \dots + a_{1n}x_n\\
        y_2 = a_{21}x_1 + a_{22}x_2 + \dots + a_{2n}x_n\\
        \vdots \\
        y_n = a_{n1}x_1 + a_{n2}x_2 + \dots + a_{nn}x_n\\
    \end{cases}
\end{equation}

Тогда существует матрица $A$ этого линейного преобразования:
\begin{eqnarray}
    A = \begin{pmatrix}
        a_{11} & a_{12} & \dots & a_{1n} \\
        a_{21} & a_{22} & \dots & a_{2n} \\
        \vdots & \vdots & \ddots & \vdots \\
        a_{n1} & a_{n2} & \dots & a_{nn} \\
    \end{pmatrix}
\end{eqnarray}

\textbf{P.s.} количество $x$ может быть отличным от количества $y$

\subsection{Алгебра матриц}

\subsubsection{Умножение матриц}

Введем два линейных преобразование:
\begin{equation}
    \begin{aligned}
        \begin{cases}
            y_1 = a_{11}x_1 + a_{12}x_2 + \dots + a_{1n}x_n\\
            y_2 = a_{21}x_1 + a_{22}x_2 + \dots + a_{2n}x_n\\
            \vdots \\
            y_n = a_{n1}x_1 + a_{n2}x_2 + \dots + a_{nn}x_n\\
        \end{cases} & &
        \begin{cases}
            z_1 = b_{11}y_1 + b_{12}y_2 + \dots + b_{1n}y_n\\
            z_2 = b_{21}y_1 + b_{22}y_2 + \dots + b_{2n}y_n\\
            \vdots \\
            z_n = b_{n1}y_1 + b_{n2}y_2 + \dots + b_{nn}y_n\\
        \end{cases}
    \end{aligned}
\end{equation}

И матрицы этих линейных преобразований:
\begin{eqnarray}
    \begin{aligned}
        A = \begin{pmatrix}
            a_{11} & a_{12} & \dots & a_{1n} \\
            a_{21} & a_{22} & \dots & a_{2n} \\
            \vdots & \vdots & \ddots & \vdots \\
            a_{n1} & a_{n2} & \dots & a_{nn} \\
        \end{pmatrix} & &
        B = \begin{pmatrix}
            b_{11} & b_{12} & \dots & b_{1n} \\
            b_{21} & b_{22} & \dots & b_{2n} \\
            \vdots & \vdots & \ddots & \vdots \\
            b_{n1} & b_{n2} & \dots & b_{nn} \\
        \end{pmatrix}
    \end{aligned}
\end{eqnarray}

Тогда пусть $y = f_A(x)$, а $z = f_B(y) = f_B(f_A(x)) = f_{A\cdot B}(x)$, получаем еще одно
линейное преобразование вида:
\begin{eqnarray}
    \begin{cases}
        z_1 = b_{11}(a_{11}x_1 + a_{12}x_2 + \dots + a_{1n}x_n) + \dots + b_{1n}(a_{n1}x_1 + a_{n2}x_2 + \dots + a_{nn}x_n) \\
        z_2 = b_{21}(a_{11}x_1 + a_{12}x_2 + \dots + a_{1n}x_n) + \dots + b_{2n}(a_{n1}x_1 + a_{n2}x_2 + \dots + a_{nn}x_n) \\
        \vdots \\
        z_1 = b_{n1}(a_{11}x_1 + a_{12}x_2 + \dots + a_{1n}x_n) + \dots + b_{nn}(a_{n1}x_1 + a_{n2}x_2 + \dots + a_{nn}x_n)
    \end{cases}
\end{eqnarray}

Рассмотрим множители для каждого $x_n$:
\begin{equation}
    \begin{aligned}
        \text{При } x_1: & & z_1: & & b_{11}a_{11} + b_{12}a_{21} + \dots + b_{1n}a_{n1} = c_{11}\\
        & & z_2: & & b_{21}a_{11} + b{22}a_{21} + \dots + b_{2n}a_{n1} = c_{12} \\
        & & \dots \\
        & & z_n: & & b_{n1}a_{11} + b{n2}a_{21} + \dots + b_{nn}a_{n1} = c_{1n} \\
        \text{При } x_2: & & \dots
    \end{aligned}
\end{equation}

Получается, любой элемент реузльтирующей матрицы: $c_{ij} = \sum\limits_{k = 1}^nb_{ik}a_{kj}$

Итого - строки умножаем на столбцы поэлементно, сумму произведений записываем

\subsubsection{Сложение и вычитание}

\begin{equation}
    \begin{pmatrix}
        a_{11} & a_{12} & \dots & a_{1n} \\
        a_{21} & a_{22} & \dots & a_{2n} \\
        \vdots & \vdots & \ddots & \vdots \\
        a_{n1} & a_{n2} & \dots & a_{nn} \\
    \end{pmatrix} \pm \begin{pmatrix}
        b_{11} & b_{12} & \dots & b_{1n} \\
        b_{21} & b_{22} & \dots & b_{2n} \\
        \vdots & \vdots & \ddots & \vdots \\
        b_{n1} & b_{n2} & \dots & b_{nn} \\
    \end{pmatrix} = \begin{pmatrix}
        a_{11} \pm b_{11} & a_{12} \pm b_{12} & \dots & a_{1n} \pm b_{1n} \\
        a_{21} \pm b_{21} & a_{22} \pm b_{22} & \dots & a_{2n} \pm b_{2n} \\
        \vdots & \vdots & \ddots & \vdots \\
        a_{n1} \pm b_{n1} & a_{n2} \pm b_{n2} & \dots & a_{nn} \pm b_{nn} \\
    \end{pmatrix}
\end{equation}

\subsubsection{Умножение на число}

\begin{equation}
    \lambda \cdot \begin{pmatrix}
        a_{11} & a_{12} & \dots & a_{1n} \\
        a_{21} & a_{22} & \dots & a_{2n} \\
        \vdots & \vdots & \ddots & \vdots \\
        a_{n1} & a_{n2} & \dots & a_{nn}
    \end{pmatrix} = \begin{pmatrix}
        \lambda a_{11} & \lambda a_{12} & \dots & \lambda a_{1n} \\
        \lambda a_{21} & \lambda a_{22} & \dots & \lambda a_{2n} \\
        \vdots & \vdots & \ddots & \vdots \\
        \lambda a_{n1} & \lambda a_{n2} & \dots & \lambda a_{nn}
    \end{pmatrix}
\end{equation}

\textbf{Осторожно} не путать с умножение определителя на число

\subsubsection{Нулевая и единичная матрицы}

Матрицы $Z$, состоящая только из нулей, обладает следующим свойством: $A + Z = Z + A = A$

Матрица $E$, состоящая из нулей везде, кроме главной диагонали, и из 1 на главной диагонали,
обладает следующим свойством: $A\cdot E = E\cdot A = A$

\subsection{Линейное векторное пространство}

$F$ - поле ($F \subseteq \mathbb{R}$ или $F \subseteq \mathbb{C}$), тогда $V(F)$ -
линейное векторное пространство:
\begin{equation}
    \begin{aligned}
        x\in V + y \in V = z \in V & & x \in V \cdot y \in V = z\in V
    \end{aligned}
\end{equation}

Для такого пространства работают коммутативность, ассоциативность,
существование 0, 1 и обратного элемента, дистрибутивность и комбинация операций

\subsubsection{Линейно-зависимое}

$\left\{x_1, x_2, \dots, x_n\right\}$ - линейно-зависимое, если $\exists \alpha_i \neq 0:$
$\alpha_1x_1 + \alpha_2x_2 + \dots + \alpha_nx_n = 0$

\subsubsection{Линейно-независимые}

Вектора $\{\vec{x_1}, \vec{x_2}, \dots, \vec{x_n}\}$ линейно-независимы, если
$\nexists (\alpha_1, \alpha_2, \dots, \alpha_n) \neq (0, 0, \dots, 0)$ таких, что
$\vec{x_1}\alpha_1 + \vec{x_2}\alpha_2 + \dots + \vec{x_n}\alpha_n = (0, 0, \dots, 0)$

$\dim{V}$ - максимальное количество линейно-независимых векторов вида $\left(\begin{aligned}
    \mathbb{R} \\ \mathbb{R}
\end{aligned}\right)$ (так называемый базис пространства $V$)

Базис - набор векторов, с помощью которых можно выразить любой другой (максимальная по 
мощности система линейно-независимых векторов)

$\rank{A}$ - количество линейно-независимых строк (или столбцов) в матрице $A$

\subsection{Системы линейных преобразований}

Пусть $y - f_A(x)$, то есть $y = A\cdot x$ $\Leftrightarrow$ $x = y\cdot A^{-1}$, причем
\begin{equation}
    \begin{aligned}
        y = \begin{pmatrix}
            y_1 \\ y_2 \\ \vdots \\ y_n
        \end{pmatrix} & &
        A = \begin{pmatrix}
            a_{11} & a_{12} & \dots & a_{1n} \\
            a_{21} & a_{22} & \dots & a_{2n} \\
            \vdots & \vdots & \ddots & \vdots \\
            a_{n1} & a_{n2} & \dots & a_{nn}
        \end{pmatrix} & &
        x = \begin{pmatrix}
            x_1 \\ x_2 \\ \vdots \\ x_n
        \end{pmatrix}
    \end{aligned}
\end{equation}

\subsection{Обратная матрица}

Если матрица $A$ не вырождена, то $\exists A^{-1}$: $A\cdot A^{-1} = A^{-1}\cdot A = E$

\subsubsection{Поиск матодом Гаусса}

Чтобы найти обратную матрицу для матрицы $A$, нужно при помощи базовых преобразований
привести матрицу вида $\left(A | E\right)$, к виду $\left(E | B\right)$, тогда матрица
$B$ будетя являться обратной к A

\paragraph{Например} пусть $A = \begin{pmatrix}
    4 & 5 & 10 \\ 1 & 2 & 3 \\ 0 & 1 & 1
\end{pmatrix}$, тогда $\left(A | E\right) = \begin{pmatrix}
    4 & 5 & 10 & \vrule & 1 & 0 & 0 \\
    1 & 2 & 3 & \vrule & 0 & 1 & 0 \\
    0 & 1 & 1 & \vrule & 0 & 0 & 1
\end{pmatrix} \rightarrow$
\begin{multline}
    \begin{pmatrix}
        0 & -3 & -2 & \vrule & 1 & -4 & 0 \\
        1 & 2 & 3 & \vrule & 0 & 1 & 0 \\
        0 & 1 & 1 & \vrule & 0 & 0 & 1
    \end{pmatrix} \rightarrow
    \begin{pmatrix}
        0 & 1 & 0 & \vrule & -1 & 4 & -2 \\
        1 & 2 & 3 & \vrule & 0 & 1 & 0 \\
        0 & 1 & 1 & \vrule & 0 & 0 & 1
    \end{pmatrix} \rightarrow
    \begin{pmatrix}
        0 & 1 & 0 & \vrule & -1 & 4 & -2 \\
        1 & 2 & 3 & \vrule & 0 & 1 & 0 \\
        0 & 0 & 1 & \vrule & 1 & -4 & 3
    \end{pmatrix} \rightarrow \\
    \begin{pmatrix}
        1 & 0 & 3 & \vrule & 2 & -7 & 4 \\
        0 & 1 & 0 & \vrule & -1 & 4 & -2 \\
        0 & 0 & 1 & \vrule & 1 & -4 & 3
    \end{pmatrix} \rightarrow
    \begin{pmatrix}
        1 & 0 & 0 & \vrule & -1 & 5 & -5 \\
        0 & 1 & 0 & \vrule & -1 & 4 & -2 \\
        0 & 0 & 1 & \vrule & 1 & -4 & 3
    \end{pmatrix} 
\end{multline}

Отсюда получается, что $A^{-1} = \begin{pmatrix}
    -1 & 5 & -5 \\ -1 & 4 & -2 \\ 1 & -4 & 3
\end{pmatrix}$

\subsubsection{Поиск через матрицу алгебраических дополнений}

Если составить матрицу, состоящую из алгебраических дополнений исходной матриц, транспонировать ее и
поделить на определитель исходной, то получиться обратная матриц

\paragraph{Пример} возьмем исходную матрицу их предыдушего примера
\begin{equation}
    A = \begin{pmatrix}
        4 & 5 & 10 \\ 1 & 2 & 3 \\ 0 & 1 & 1
    \end{pmatrix} \rightarrow
    A^{*} = \begin{pmatrix}
        (-1)^{1 + 1} \begin{vmatrix}
            2 & 3 \\ 1 & 1
        \end{vmatrix} &
        (-1)^{1 + 2} \begin{vmatrix}
            1 & 3 \\ 0 & 1
        \end{vmatrix} &
        (-1)^{1 + 3} \begin{vmatrix}
            1 & 2 \\ 0 & 1
        \end{vmatrix} \\
        (-1)^{2 + 1} \begin{vmatrix}
            5 & 10 \\ 1 & 1
        \end{vmatrix} &
        (-1)^{2 + 2} \begin{vmatrix}
            4 & 10 \\ 0 & 1
        \end{vmatrix} &
        (-1)^{2 + 3} \begin{vmatrix}
            4 & 5 \\ 0 & 1
        \end{vmatrix} \\
        (-1)^{3 + 1} \begin{vmatrix}
            5 & 10 \\ 2 & 3
        \end{vmatrix} &
        (-1)^{3 + 2} \begin{vmatrix}
            4 & 10 \\ 1 & 3
        \end{vmatrix} &
        (-1)^{3 + 3} \begin{vmatrix}
            4 & 5 \\ 1 & 2
        \end{vmatrix}
    \end{pmatrix}
\end{equation}
\begin{eqnarray}
    A^{*} = \begin{pmatrix}
        -1 & -1 & 1 \\
        5 & 4 & -4 \\
        -5 & -2 & 3
    \end{pmatrix} \rightarrow
    A^{*\tau} = \begin{pmatrix}
        -1 & 5 & -5 \\
        -1 & 4 & -2 \\
        1 & -4 & 3
    \end{pmatrix}
\end{eqnarray}

Так как $\det{A} = 1$, то $A^{-1} = \frac{A^{*\tau}}{\det{A}} = \frac{A^{*\tau}}{1} = A^{*\tau}$,
что совпадает с результатом, полученным ранее

\subsection{Ранг матрицы}

Ранг матрицы - максимальное число линейно-независимых строк/столбцов
\begin{equation}
    A = \begin{pmatrix}
        \vec{a_1} \\ \vec{a_2} \\ \vdots \\ \vec{a_n}
    \end{pmatrix} = \begin{pmatrix}
        a_{11} & a_{12} & \dots & a_{1n} \\
        a_{21} & a_{22} & \dots & a_{2n} \\
        \vdots & \vdots & \ddots & \vdots \\
        a_{n1} & a_{n2} & \dots & a_{nn}
    \end{pmatrix} \rightarrow \begin{pmatrix}
        a_{11}^{'} & a_{12}^{'} & \dots & a_{1n}^{'} \\
        \vdots & \vdots & \ddots & \vdots \\
        a_{r1}^{'} & a_{r2}^{'} & \dots & a_{rn}^{'} \\
        \dots & \dots & \emptyset & \dots
    \end{pmatrix} \Rightarrow \rank{A} = r
\end{equation}

Ранг матрицы - порядок наибольшего ненулевого минора, отсюда вытекает метод
окаймляющих миноров - проверяем минор $k$-ого порядка, если он ненулевой, то
ищем окаймляющие его (если таких нет, ранг матрицы найден), иначе рассматриваем
другой минор $k$-ого порядка

\subsubsection{Теорема Кронекера-Капелли}

Пусть есть некая СЛАУ:
\begin{equation}
    \begin{cases}
        a_{11}x_1 + a_{12}x_2 + \dots + a_{1n}x_n = b_1 \\
        a_{21}x_1 + a_{22}x_2 + \dots + a_{2n}x_n = b_2 \\
        \dots \\
        a_{n1}x_1 + a_{n2}x_2 + \dots + a_{nn}x_n = b_n 
    \end{cases}
\end{equation}

Из нее получаем следующие матрицы:
\begin{equation}
    \begin{aligned}
        A = \begin{pmatrix}
            a_{11} & a_{12} & \dots & a_{1n} \\
            a_{21} & a_{22} & \dots & a_{2n} \\
            \vdots & \vdots & \ddots & \vdots \\
            a_{n1} & a_{n2} & \dots & a_{nn}
        \end{pmatrix} & & B = \begin{pmatrix}
            b_1 \\ b_2 \\ \vdots \\ b_n
        \end{pmatrix} & &
        \overline{A} = (A | B) = \begin{pmatrix}
            a_{11} & a_{12} & \dots & a_{1n} & \vrule & b_1 \\
            a_{21} & a_{22} & \dots & a_{2n} & \vrule & b_2 \\
            \vdots & \vdots & \ddots & \vdots & \vrule & \vdots \\
            a_{n1} & a_{n2} & \dots & a_{nn} & \vrule & b_n
        \end{pmatrix}
    \end{aligned}
\end{equation}

Тогда исходная система имеет хотя бы одно решение (совместна) тогда и только тогда, когда $\rank{A} = \rank{\overline{A}}$

\textbf{Доказательство}

\begin{enumerate}
    \item {
        Если система имеет хотя бы одно решение, то столбец свободных членов - линейная комбинация
        столбцов матрицы $A$ $\Rightarrow$ добавление этого столбца в матрицу не изменит ранга
    }
    \item {
        Если $\rank{A} = \rank{\overline{A}}$, то они имеют один и тот же базисный минор $\Rightarrow$
        столбец свободных членов - линейная комбинация, то есть решение есть по пункту 1
    }
\end{enumerate}

\end{document}

