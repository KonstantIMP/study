% Лекция по Алгебре и геометрии 15.11.2022
% Михедов Константин Константиновчи, БИБ 224

% Тип документа: статья, размер бумаги - A4, написано 14 кегелем
% Предназначено для импорта из другого документа
\documentclass[class=article,a4paper,12pt,crop=false]{standalone}

% Поиск по скомпилированному PDF
\usepackage{cmap}
% Кодировка выходного текста
\usepackage[T2A]{fontenc}
% Кодировка исходного текста
\usepackage[utf8]{inputenc}
% Поддержка необходимых языков
\usepackage[english,russian]{babel}

% Вращения текста
\usepackage{rotating}

% Поддержка изображений
\usepackage{graphicx}
% Путь до внешних изображений
\graphicspath{ {./figures/}}

% Умная запятая
\usepackage{icomma}

% Список в несколько строк
\usepackage{multicol}

% Ссылки на электронные ресурсы
\usepackage{hyperref}
% Настройка внешнего вида ссылок
\hypersetup{
  % Отключить прямоугольную рамку
  pdfborder={0 0 0},
  % Включить цветные ссылки
  colorlinks=true,
  % Цвет для ссылок на веб-ресурсы
  urlcolor=blue,
  % Цвет внутренних ссылок
  linkcolor=black
}

% Дополнительная математика
\usepackage{amsmath,amsfonts,amssymb,amsthm,mathtools}
% Показывать номера только у тех выржений, на которые кто-то ссылается
\mathtoolsset{showonlyrefs=true}

% Дополнительные символы
\usepackage{mathbbol}

% Подключние пакетов для импорта других .tex
\usepackage[subpreambles=true]{standalone}
\usepackage{import}

% перечеркивания
\usepackage{cancel}

% Правильное оформление
% Настройка отступов
\usepackage[left=2cm,right=1cm,top=2cm,bottom=2cm]{geometry}
% Настройка шрифта
\usepackage{fontspec}
\setmainfont{Times New Roman}
% Настройка межстрочных интервалов
\usepackage{setspace}
\onehalfspacing
% и межабзацных
\usepackage{parskip}
\setlength{\parindent}{1.25cm} 

% Красная строка
\usepackage{indentfirst}
\setlength{\parindent}{1.25cm} 

% Корректное положение рисунков?
\usepackage{float}

% Расположение блоков относительно текста
\usepackage{wrapfig}

% Ранг матрицы
\DeclareMathOperator{\rank}{rank}

\begin{document}

Полиномом $n$-ой степени называют выражение $f(x) = a_0 + a_1x + a_2x^2 + \dots + a_nx^n$ 
(притом $a_i \in \mathbb{C}$, $a_n \neq 0$, степени $x$ строго неотрицательны)

Степень данного полинома: $\deg{f(x)} = n$ (degree - степень)

\subsection{Операции над полиномами}

\begin{enumerate}
    \item {
        Сложение и вычитание:

        $f(x) \pm g(x) = c(x) = c_0 + c_1x + c_2x^2 + \dots + c_nx^n$ ($n = \deg{c(x)} = \max{(\deg{f(x)}, \deg{g(x)})}$)
    }
    \item {
        Умножение:

        $c(x) = f(x)\cdot g(x)$ ($\deg{c(x)} = \deg{f(x)} + \deg{g(x)}$)
    }
    \item {
        Деление:

        Если $\deg{f(x)} > \deg{g(x)}$ $\Rightarrow$ $\exists!$ $q(x)$ и $r(x)$:
        $\deg{q(x)} > \deg{r(x)}$ и $f(x) = g(x)q(x) + r(x)$ 

        Причем важно, что подобное разложение \textbf{единственно}:

        Пусть существуют еще многочлены $\overline{q}(x)$ и $\overline{r}(x)$,
        также удовлетворяющие исходному равенству ($f(x) = g(x)\overline{q}(x) + \overline{r}(x)$).
        Приравняем исходное и полученное выражение:
        \begin{equation}
            g(x)\overline{q}(x) + \overline{r}(x) = g(x)q(x) + r(x) \Leftrightarrow
            g(x)(\overline{q}(x) - q(x)) = r(x) - \overline{r}(x)
        \end{equation}

        Так как $\deg{(r(x) - \overline{r}(x))} < \deg{g(x)}$ (из определения деления), то следовательно, если было бы, что 
        $\deg{(\overline{q}(x) - q(x))} \neq 0$, то $\deg{(g(x)(\overline{q}(x) - q(x)))}$
        была бы $\geq \deg{g(x)}$, что не могло бы произойти исходя из первого высказывания.
        Тогда $\deg{(\overline{q}(x) - q(x))}$ должна быть $= 0$, то есть
        $\overline{q}(x) = q(x)$, отсюда $\deg{(\overline{q}(x) - q(x))} = 0$
        $\Rightarrow$ $\deg{(r(x) - \overline{r}(x))} = 0$ $\Rightarrow$ $r(x) = \overline{r}(x)$ $(\blacksquare)$
    
        и в принципе \textbf{существует}:

        Пусть есть полиномы $f(x)$ и $g(x)$ такие, что $\deg{f(x)} = n$, а $\deg{g(x)} = 1$

        \begin{enumerate}
            \item {
                Если $n < s$, то $q(x) = 0$, а $r(x) = f(x)$
            }
            \item {
                Иначе, можно поэлементно все делить, для этого распишем $f(x)$ и $g(x)$
                \begin{equation}
                    \begin{aligned}
                        f(x) &= a_0x^n + a_1x^{n-1} + a_2x^{n-2} + \dots + a_{n-1}x + a_n\\
                        g(x) &= b_0x^s + b_1x^{s-1} + b_2x^{s-2} + \dots + b_{s-1}x + b_s
                    \end{aligned}
                \end{equation}

                Тогда можем выразить:
                \begin{equation}
                    f_1(x) = f(x) - \frac{a_0}{b_0}x^{n-s}g(x)
                \end{equation}

                Обозначим $n_1 = \deg{f_1(x)} < \deg{f(x)}$ и $a_{1,0}$ - как множитель при $x^{n_1}$ в полиноме $f_1(x)$.
                Теперь можем выразить:
                \begin{eqnarray}
                    f_2(x) = f_1(x) - \frac{a_{1,0}}{b_0}x^{n_1-s}g(x)
                \end{eqnarray}

                Таким образом поэлементное деление можно произваодить вплоть до:
                \begin{equation}
                    f_{k} = f_{k-1} - \frac{a_{k-1,0}}{b_0}x^{n_{k - 1} - 1}g(x)
                \end{equation}

                Сложив все подобные равенства в одно целое, получим:
                \begin{equation}
                    f(x) - (\frac{a_0}{b_x}x^{n-s} + \frac{a_{1,0}}{b_0}x^{n_1 - s} + \dots +
                    \frac{a_{k-1,n}}{b_0}x^{n_{k-1} - s})g(x) = f_k(x)
                \end{equation}

                То есть многочлены:
                \begin{equation}
                    \begin{aligned}
                        q(x) &= \frac{a_0}{b_x}x^{n-s} + \frac{a_{1,0}}{b_0}x^{n_1 - s} + \dots +
                        \frac{a_{k-1,n}}{b_0}x^{n_{k-1} - s} \\
                        r(x) &= f_k(x) \:\:\: (\blacksquare)
                    \end{aligned}
                \end{equation}
            }
        \end{enumerate}
    }
\end{enumerate}

\subsection{Делители полиномов}

Пусть есть полиномы, определенные на множестве комплексных чисел: $f(x), \varphi(x), x \in \mathbb{C}$ 

Говорят, что $f(x)$ делится на $\varphi(x)$ (короткая записа $f(x) \vdots \varphi(x)$),
если $\exists \varPsi(x)$: $f(x) = \varphi(x)\varPsi(x)$

Верны также следующие утверждения
\begin{multicols}{2}
    \begin{enumerate}
        \item {
            $f(x) \vdots \varphi(x)$ $\land$ $\varphi(x)\vdots\varPsi(x)$ $\Rightarrow$ $f(x)\vdots\varPsi(x)$
        }
        \item {
            $f(x)$ и $g(x)\vdots\varphi(x)$ $\Rightarrow$ $(\alpha{f(x)} + \beta{g(x)})\vdots\varphi(x)$
        }
        \item {
            $f(x)\vdots\varphi(x)$ $\Rightarrow$ $f(x)h(x)\vdots\varphi(x)$
        }
        \columnbreak
        \item {
            $f(x)\vdots\varphi(x)$ $\Rightarrow$ $cf(x)\vdots\varphi(x)$
        }
        \item {
            $f(x)\vdots{g(x)}$ и $g(x)\vdots{f(x)}$ $\Rightarrow$ $f(x) = g(x)$
        }
        
    \end{enumerate}
\end{multicols}

\subsection{Наибольший общий делитель}

НОД двух многочленов - их общий делитель наивысшей степени

НОД можно расписать с помощью так называемых коэффициентов Безу ($v(x)$ и $\nu(x)$):
\begin{equation}
    d(x) = (f(x), g(x)) \Rightarrow \exists v(x), \nu(x): \:\:\:
    d(x) = v(x)f(x) + \nu(x)g(x)
\end{equation}

\paragraph{Доказательство} так называемой теоремы о линейном представлении НОД
\begin{equation}
    \begin{aligned}
        f(x) &= q_1(x)g(x) + r_1(x) \\
        g(x) &= q_2(x)r_1(x) + r_2(x) \\
        r_1(x) &= q_3(x)r_2(x) + r_3(x) \\
        r_2(x) &= q_4(x)r_3(x) + r_4(x) \\
        &\dots \\
        r_{k - 2}(x) &= q_k(x)r_{k - 1}(x) + r_k(x) \\
        r_k(x) &= q_{k + 2}r_{k + 1}(x) + 0
    \end{aligned}
\end{equation}

Тогда $d(x) = r_k(x) = r_{k - 2}(x) - q_kr_{k-1}(x)$. Пусть $\nu(x) = 1$, а $v(x) = -q_k$, тогда верно, что
$d(x) = \nu(x)r_{k - 2}(x) + v(x)r_{k - 1}(x) = r_{k - 2}(x)$ $(\blacksquare)$

Тут, для нахождения НОД использовался алгоритм Евклида: делить рекуррентно делился на частное,
пока не нашелся такой остаток ($r_k(x)$), на который бы без остатка делилось все

\subsection{Корни многочленов}

Если $f(c) = 0$, то говорят, что $c$ - корень многочлена $f(x)$ (эквивалентно $f(x) \vdots (x - c)$)

\subsubsection{Теорема Безу}

$P_n(x) = (x - \alpha)Q_{n - 1}(x) + r$, тогда $r = P_n(\alpha)$

\textbf{Доказательство:} $P_n(\alpha) = (\alpha - \alpha)Q_{n - 1}(x) + r = 0\cdot{Q_{n - 1}(x)} + r = r$

\textbf{Следствия}
\begin{enumerate}
    \item {
        Если $P_n(\alpha) = 0$ $\Rightarrow$ $\alpha$ - корень многочлена
    }
    \item {
        $(P_n(x) - P_n{\alpha})\vdots{(x - \alpha)}$
    }
    \item {
        Пусть $P_n(x) = a_nx^n + \dots + a_1x + a_0$ ($a_i \in \mathbb{Z}$ $\forall i$)

        Тогда $\alpha \in \mathbb{Z}$ $\land$ $P_n(\alpha) = 0$ $\Rightarrow$ $Q_{n - 1}(x)$ - многочлен с целыми коэффициентами
    }
\end{enumerate}

\subsubsection{Кратные корни}

Число $c \in \mathbb{C}$ является кратным корнем многочлена $P_N(x)$, если
$P_n^{(i)}(c) = 0$, а $P_n^{(k + 1)}(c) \neq 0$ ($i = 0, 1, 2, \dots, k - 1, k$) (здесь
степень означает номер производной)

Или если $P_n(x) \vdots (x - c)^k$ и $P_n(x)\cancel{\vdots}(x - c)^{k + 1}$

\subsection{Основная теорема алгербры}

Всякий многочлен, степень которого не меньше единицы, имеет хотя бы один корень, в общем случае комплексный

Для ее доказательство потребуется несколько сторонних теорем
\subsubsection{Лемма о непрерывности произвольного многочленов}

\textbf{Формулировка} любой многочлен непрерывен на всей комплексной плоскости

Если свободный член многочлена $P_n(x)$ равен нулю ($P_n(0) = 0$, а сам многочлен
имеет вид $P_n(x) = a_0x^n + a_1x^{n - 1} + \dots + a_{n - 1}x$), то $\forall \varepsilon > 0$
можно подобрать такое $\delta$, то $|x| < \delta$ $\Rightarrow$ $|P_n(x)| < \varepsilon$ $\forall x$

Пусть $A = \max\{|a_0|, |a_1|, \dots, |a_{n - 1}|\}$, а число $\varepsilon$ - зафиксировано,
тогда возьмем $\delta = \frac{\varepsilon}{A + \varepsilon}$.

Исходя из свойств модуля верно, что $|P_n(x)| < |a_0||x^n| + |a_1||x^{n - 1}| + \dots + |a_{n - 1}||x|$, что
в свою очередь $< A(|x|^n + |x|^{n - 1} + \dots + |x|) = A\frac{|x| - |x|^{n + 1}}{1 - |x|}$
(сумма n членов геометрической прогрессии)

Так как по условию $|x| < \delta$, а $\delta = \frac{\varepsilon}{A + \varepsilon} < 1$,
то $\frac{|x| - |x|^{n + 1}}{1 - |x|} < \frac{|x|}{1 - |x|}$, тогда
\begin{equation}
    f(x) < A\frac{x}{1 - x} < A\frac{\delta}{1 - \delta} <
    \frac{\frac{A\varepsilon}{A + \varepsilon}}{1 - \frac{\varepsilon}{A + \varepsilon}} = \varepsilon (\blacksquare)
\end{equation}

\subsubsection{Формула Тейлора}
\begin{equation}
    P_n(x + h) = P_n(x) + hP_n^{'}(x) + \frac{h^2}{2!}P_n^{''}(x) + \dots + \frac{h^n}{n!}P_n^{(n)}(x)
\end{equation}

\subsubsection{Теорема о непрерывности произвольного многочлена}

\textbf{Формулировка} произвольный многочлент $P_n(x)$ непрерывен в любой точке $x_0$ точке комплексной плоскости

По формуле Тейлора: $P_n(x_0 + h) - P_n(x_0) = c_1h + c_2h^2 + \dots + c_nh^n = \varphi(h)$,
где $c_1 = P_n^{'}(x_0)$, $c_2 = \frac{P_n^{''}(x_0)}{2!}$, $\dots$, $c_n = \frac{P_n^{(n)}(x_0)}{n!}$
 
Видно, что $\varphi(h)$ - многочлен без свободного члена, тогда по лемме о непрерывности произвольного
многочлена $\forall \varepsilon > 0$ можно подобрать такое $\delta > 0$, что $|h| < d$ $\Rightarrow$ 
$|\varphi(h)| < \varepsilon$, а так как $\varphi(h) = P_n(x_0 + h) - P_n(x_0)$, то
$|h| < d$ $\Rightarrow$ $|P_n(x_0 + h) - P_n(x_0)| < \varepsilon$

По свойству модуля: $||P_n(x_0 + h)| - |P_n(x_0)|| \leq |P_n(x_0 + h) - P_n(x_0)| < \varepsilon$,
откуда вытекает непрерывность многочлена в точке $x_0$

\subsubsection{Лемма о модуле старшего члена}

Пусть дан полином $P_n(x) = a_0x^n + a_1x^{n - 1} + \dots + a_n$, тогда для $\forall k > 0$
при $x \rightarrow \infty$: $|a_0x^n| > k|a_1x^{n-1} + \dots + a_n|$, то есть модуль старшего
члена больше модуля суммы всех остальных членов во сколько угодно раз

\paragraph{Доказательство} Пусть $A = \max\{|a_0|, |a_1|, \dots, |a_n|\}$, тогда по свойству модулей
\begin{equation}
    |a_1x^{n - 1} + \dots + a_n| \leq |a_1||x|^{n - 1} + \dots + |a_n|
    \leq A(|x|^{n - 1} + \dots + 1) = A\frac{|x|^n - 1}{|x| - 1}
\end{equation}

Если $|x| > 1$, то $\frac{|x|^n - 1}{|x| - 1} < \frac{|x|^n}{|x| - 1}$ $\Rightarrow$
$|a_1x^{n - 1} + \dots + a_n| \leq A\frac{|x|^n}{|x| - 1}$

Выберем $x$ так, чтобы $|x| > 1$ (верно, так как при достаточно больших) и
$x > \frac{kA}{|a_0|} + 1$, тогда $kA = \frac{x - 1}{|a_0|}$ $\Rightarrow$
$kA\frac{|x|^n}{|x| - 1} \leq |a_0x^n|$ $(\blacksquare)$

\subsubsection{Лемма о возрастании модуля многочлена}

Для любого комплексного полинома со степенью большей или равной 1 и $\forall M \in \mathbb{R} > 0$
(причем сколь угодно большого), можно подобрать такое действительное положительное число $N$:
$|x| > N$ $\Rightarrow$ $|P_n(x)| > M$

\paragraph{Доказательство} Пусть $P_n(x) = a_0x^n + a_1x^{n - 1} + \dots + a_n$, тогда
по свойства модулей \[|P_n(x)| = |a_0x^n + (a_1x^{n - 1} + \dots + a_n)|
\geq |a_0x^n| - |a_1x^{n - 1} + \dots + a_n|\]

По лемме о модуле старшего члена (при $|x| > N_1$):
\begin{equation}
    |a_0x^n| > 2|a_1x^{n - 1} + \dots + a_n| \Leftrightarrow
    |a_1x^{n - 1} + \dots + a_n| < 0.5|a_0x^n|
\end{equation}

Тогда верно, что $|P_n(x)| > 0.5|a_0x^n|$ $\Rightarrow$ $0.5|a_0x^n|$ будет больше, чем $M$
при $|x| > \sqrt[n]{\frac{2M}{|a_0|}} = N_2$

Тогда при $|x| > \max\{N_1, N_2\}$ $|P_n(x)|$ будет больше $M$ ($\blacksquare$)

\subsubsection{Лемма Д'Аламбера}

Если $P_n(c) \neq 0$, то $\exists h$: $|P_n(c + h)| < |P_n(h)|$

По формуле Тейлора: $P_n(x + h) = P_n(x) + hP_n^{'}(x) + \frac{h^2}{2!}P_n^{''}(x) + \dots + \frac{h^n}{n!}P_n^{(n)}(x)$

По условию $c$ не является корнем $P_n(x)$, но оно случайно может оказаться корнем какой-либо
производной. Тогда пусть $k$-ая производная - первая, для которой
$c$ не является корнем: $P_n^{'}(c) = P_n^{''}(c) = \dots = P_n^{(k - 1)}(c) = 0$, а
$P_n^{k}(c) \neq 0$. Она обязательно существкет, так $P_n^{(n)}(c) = a_0n!$, тогда
\begin{equation}
    P_n(c + h) = P_n(c) + \frac{h^kP_n^{(k)}(c)}{k!} + \frac{h^{k + 1}}{(k + 1)!}P_n^{(k + 1)}(c) + \dots + \frac{h^n}{n!}P_n^{(n)}(c)
\end{equation}

Обе части высказывания поделим на $P_n(c)$, которое по условию не равно нулю:
\begin{equation}
    \frac{P_n(c + h)}{P_n(c)} = 1 + C_kh^k + C_{k + 1}h^{k + 1} + \dots + C_nh^n (C_j = \frac{P_n^{(j)}(c)}{j!P_n(c)}) 
\end{equation}

Так как $P_n^{(k)}(c) \neq 0$, то верно:
\begin{equation}
    \frac{P_n(c + h)}{P_n(c)} = (1 + C_k) + C_kh^k(\frac{C_{k + 1}h}{C_k} + \dots + \frac{C_n}{C_k})
\end{equation}

Данное выражение в модульном эквиваленте:
\begin{equation}
    \left|\frac{P_n(c + h)}{P_n(c)}\right| \leq |(1 + C_k)| + |C_kh^k|\left|(\frac{C_{k + 1}h}{C_k} + \dots + \frac{C_n}{C_k})\right|
\end{equation}

Теперь каким-нибудь образом зафиксируем число $h$

Выберем $|h|$ так, чтобы $\frac{C_{k + 1}}{C_k}h + \dots + \frac{C_n}{C_k} < \frac{1}{2}$ и $|h| < d_1$.
Так как $C_kh^k$ - тоже многочлен без свободного члена, то может быть такое, что $|C_kh^k| < 1$
(при $|h| < d_2 < \sqrt[k]{|C_k^{-1}|}$)

Соответственно пусть $|h| = \min\{d_1, d_2\}$, тогда $\left|\frac{P_n(c + h)}{P_n(c)}\right|
< |1 + C_kh^k| + 0.5|C_kh^k|$

Теперь выберем аргумент $h$ так, чтобы $C_kh^k$ было бы отрицательным действительным (не комплексным)
числом, то есть $\arg(C_kh^k) = \arg{C_k} + k\arg{h} = \pi$, то есть $\arg{h} = \frac{\pi - \arg{C_k}}{k}$

Тогда цисло будет отличаться от своего модуля только знаком: $C_kh^k = -|C_kh^k|$

Тогда $1 + C_kh^k = 1 - |C_kh^k|$, тогда можно утверждать, что
\begin{equation}
    \left|\frac{P_n(c + h)}{P_n(c)}\right|
< |1 + C_kh^k| + 0.5|C_kh^k| = 1 - |C_kh^k| + 0.5|C_kh^k| = 1 - 0.5|C_kh^k|
\end{equation}

А так как $h$ был выбран так, что $|C_kh^k| > 1$ (а по определениб модуля еще и $|C_kh^k| \geq 0$),
то:
\begin{equation}
    1 - 0.5|C_kh^k| < 1 \Rightarrow \left|\frac{P_n(c + h)}{P_n(c)}\right| < 1 \Leftrightarrow
    |P_n(c + h)| < |P_n(c)| (\blacksquare)
\end{equation}

\subsubsection{Доказательство основной теоремы}

Пусть есть $P_n(x)$, такой, что его степень $\geq 1$

\begin{enumerate}
    \item {
        Если свободный член $P_n(x)$ равен нулю, то $P_n(0) = 0$, то есть
        существует минимум один корен $(\blacksquare)$
    }
    \item {
        Иначе очевидно, что свободный член $a_n = P_n(0) \neq 0$

        Исходный многочлен непрерывен на всей плоскости комплексных чисел,
        тогда можно найти число $M = \inf{|P_n(x)|}$, то есть существует
        такая точка $z_0$, что $M = |P_n(z_0)| \leq |P_n(z)|$ $\forall z \in \mathbb{C}$

        Если $P_n(z_0) \neq 0$, то по лемме Д'Аламбера найдется такой $z^{'}$,
        что $|P_n(z^{'})| < |P_n(z_0)|$, чего быть не может $(\blacksquare)$
    }
\end{enumerate}

\subsubsection{Следствие}

Любой неконстантный многочлен над полем комплексных чисел можно разложить в произведелние линейных
множителей (разложение единственно):
\begin{equation}
    P_n(z) = a_n(z - z_0)^{k_0}(z - z_1)^{k_1}\dots(z - z_n)^{k_n}
\end{equation}

\subsection{Интерполяционный многочлен Лагранжа}

Призван решать задачу поиска полинома, который проходит через заданный набор точек с
координатами вида ($x$, $y$).

Базовая формула для его вычисления, при заданных точках
$(x_0, y_0), (x_1, y_1), \dots, (x_n, y_n)$:
\begin{equation}
    \begin{aligned}
        L(x) = \sum\limits_{i = 0}^n(y_il_i(x)) & & & &
        l_i(x) = \underset{j \neq i}{\prod\limits_{j = 0}^n}\frac{x - x_j}{x_i - x_j}
    \end{aligned}
\end{equation}

\subsection{Формулы Виета}

Пусть задан полином $n$-ой степени:
\begin{equation}
    P_n(x) = a_0x^n + a_1x^{n - 1} + \dots + a_{n - 1}x + a_n
\end{equation}

По следствию из ОТА можно записать следующее равенство:
\begin{equation}
    a_0x^n + a_1x^{n - 1} + \dots + a_{n - 1}x + a_n = a_0(x - x_0)^{k_0}\dots(x - x_n)^{k_n}
\end{equation}

Можно покоэффициентно составить выражение для каждого a:
\begin{equation}
    \begin{aligned}
        a_0 =& a_0 \\
        a_1 =& -a_0 \cdot (x_0 + x_1 + \dots + x_n) \\
        \dots & \\
        a_n =& a_0 (-1)^{n}x_0x_2\dots{x_n}
    \end{aligned}
\end{equation}

Откуда вытекает, что сумма корней $= -\frac{a_1}{a_0}$ (обратное к коэффициенту при степени на единицу
меньшей, чем степень многочлена),
а произведение корней $= (-1)^{n}\cdot\frac{a_n}{a_0}$ (где $n$ - степень многочлена)

\subsection{Теорема о разложении многочлена с вещественными коэффициентами}

Многочлен, у которого все коэффициенты при $x$ - действительные числа может быть
разложен следующим образом:

$P_n(x) = a_0(x - \alpha_0)^{k_0}\dots(x - \alpha_v)^{k_v}(x^2 + p_1x + q^1)\dots(x^2 + p_sx + q_s) \:\:\:\: (n = v + s)$

Здесь слева - действительные корни многочлена, а справа - комплексные. Но записаны они 
в виде произведения $(x - z)(x - \overline{z})$, так как если комплексное
число $z$ является корнем уравнения, то и $\overline{z}$ также является ее корнем.


\end{document}

