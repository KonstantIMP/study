% Лекция по Алгебре и геометрии 12.09.2022
% Михедов Константин Константиновчи, БИБ 224

% Тип документа: статья, размер бумаги - A4, написано 14 кегелем
% Предназначено для импорта из другого документа
\documentclass[class=article,a4paper,12pt,crop=false]{standalone}

% Поиск по скомпилированному PDF
\usepackage{cmap}
% Кодировка выходного текста
\usepackage[T2A]{fontenc}
% Кодировка исходного текста
\usepackage[utf8]{inputenc}
% Поддержка необходимых языков
\usepackage[english,russian]{babel}

% Поддержка изображений
\usepackage{graphicx}
% Путь до внешних изображений
\graphicspath{ {./figures/}}

% Умная запятая
\usepackage{icomma}

% Ссылки на электронные ресурсы
\usepackage{hyperref}
% Настройка внешнего вида ссылок
\hypersetup{
  % Отключить прямоугольную рамку
  pdfborder={0 0 0},
  % Включить цветные ссылки
  colorlinks=true,
  % Цвет для ссылок на веб-ресурсы
  urlcolor=blue,
  % Цвет внутренних ссылок
  linkcolor=black
}

% Дополнительная математика
\usepackage{amsmath,amsfonts,amssymb,amsthm,mathtools}
% Показывать номера только у тех выржений, на которые кто-то ссылается
\mathtoolsset{showonlyrefs=true}

% Дополнительные символы
\usepackage{mathbbol}

% Подключние пакетов для импорта других .tex
\usepackage[subpreambles=true]{standalone}
\usepackage{import}

\begin{document}
  \subsection{Определители}
  
  \subsubsection{Определители 3 на 3}

  %\begin{equation}
    \begin{multline}
      \det{
        \begin{pmatrix}
          a_{11} & a_{12} & a_{13} \\
          a_{21} & a_{22} & a_{23} \\
          a_{31} & a_{32} & a_{33}
        \end{pmatrix}
      }
      = a_{11}a_{22}a_{33}
      + a_{12}a_{23}a_{31}
      + a_{21}a_{32}a_{13} - \\
      - a_{31}a_{22}a_{13}
      - a_{21}a_{12}a_{33}
      - a_{32}a_{23}a_{11}
    \end{multline}
  %\end{equation}

  \subsubsection{Определители 2 на 2}

  \begin{equation}
    \det{
      \begin{pmatrix}
        a_{11} &a_{12} \\
        a_{21} & a_{22}
      \end{pmatrix}
    } = a_{11}a_{22} - a_{12}a_{21}
  \end{equation}

  \subsubsection{Определитель 1 на 1}

  \begin{equation}
    \det{
      \begin{pmatrix}
        b
      \end{pmatrix}
    } = b
  \end{equation}

  \subsubsection{Общее графическое правило}

  Понимай это как хочешь, удачи тебе

  \begin{picture}(350,50)
    \put(125,10){
      $\begin{pmatrix}
          \circ & \circ & \circ \\
          \circ & \circ & \circ \\
          \circ & \circ & \circ
        \end{pmatrix}$
    }

    \thicklines

    \put(164,37.5){\color{blue}\line(1,-1){16}}
    \put(149,37.5){\color{green}\line(1,-1){32.5}}
    \put(134,37.5){\color{red}\line(1,-1){47.5}}
    \put(134,22.5){\color{blue}\line(1,-1){32.5}}
    \put(134,7.5){\color{green}\line(1,-1){16}}

    \put(205,10){\text{-}}

    \put(225,10){
      $\begin{pmatrix}
          \circ & \circ & \circ \\
          \circ & \circ & \circ \\
          \circ & \circ & \circ
        \end{pmatrix}$
    }

    \put(264,-11){\color{magenta}\line(1,1){16}}
    \put(249,-11){\color{cyan}\line(1,1){32.5}}
    \put(234,-11){\color{yellow}\line(1,1){47.5}}
    \put(234,4){\color{magenta}\line(1,1){32.5}}
    \put(234,18){\color{cyan}\line(1,1){16}}
  \end{picture}

\end{document}
