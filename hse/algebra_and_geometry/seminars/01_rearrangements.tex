% Семинар и немного лекции по Алгебре и геометрии 21.09.2022
% Михедов Константин Константиновчи, БИБ 224

% Тип документа: статья, размер бумаги - A4, написано 14 кегелем
% Предназначено для импорта из другого документа
\documentclass[class=article,a4paper,12pt,crop=false]{standalone}

% Поиск по скомпилированному PDF
\usepackage{cmap}
% Кодировка выходного текста
\usepackage[T2A]{fontenc}
% Кодировка исходного текста
\usepackage[utf8]{inputenc}
% Поддержка необходимых языков
\usepackage[english,russian]{babel}

% Поддержка изображений
\usepackage{graphicx}
% Путь до внешних изображений
\graphicspath{ {./figures/}}

% Умная запятая
\usepackage{icomma}

% Ссылки на электронные ресурсы
\usepackage{hyperref}
% Настройка внешнего вида ссылок
\hypersetup{
  % Отключить прямоугольную рамку
  pdfborder={0 0 0},
  % Включить цветные ссылки
  colorlinks=true,
  % Цвет для ссылок на веб-ресурсы
  urlcolor=blue,
  % Цвет внутренних ссылок
  linkcolor=black
}

% Дополнительная математика
\usepackage{amsmath,amsfonts,amssymb,amsthm,mathtools}
% Показывать номера только у тех выржений, на которые кто-то ссылается
\mathtoolsset{showonlyrefs=true}

% Дополнительные символы
\usepackage{mathbbol}

% Зачеркивание
\usepackage{cancel}

% Подключние пакетов для импорта других .tex
\usepackage[subpreambles=true]{standalone}
\usepackage{import}

\begin{document}
  
\subsection{Перестановки}

\textbf{Определение:} перестановкой из n элементов называют
последовательность, состоящую из всех элементов некого n-элементного
множества, причем число элементов этой последовательности равно n.

\textbf{Или такое:} расположение n чисел в любом порядке

\subsubsection{Транспозиция}

\textbf{Определение:} попарное позиций двух элементов перестановки (поменяли два элемента местами).

\subsubsection{Инверсия}

Пусть дана некоторая перестановка n различных чисел:
\begin{equation}
  \alpha_1, \alpha_2, \dots, \alpha_i, \dots, \alpha_k, \dots, \alpha_n
\end{equation}

Говорят, что два числа $\alpha_i$ и $\alpha_k$ образуют инверсию, если
большее число стоит левее меньшего, то есть $i < k$ и $\alpha_i > \alpha_k$.
Например:
\begin{equation}
  (1, \:\: 2, \:\: 3, \underbrace{5, \:\: 4}_{\text{инверсия}})
\end{equation}

\subsubsection{Четность перестановок}

Перестановка называется чётной, если емеет четное количество перестановок,
и наоборот (четность перестановки совпадает с четностью количества инверсий).

Например, $(4, 5, 1, 3, 6, 2)$ имеет 8 перестановок, следовательно является
четной.

\subsubsection{Свойства перестановок}

Пусть $\Pi = (\alpha_1, \alpha_2, \dots, \alpha_n)$, $\Pi \in \mathbb{S}_i$,
где $\mathbb{S}_i$ - множество всех перестановок длины n. Тогда:

\begin{enumerate}
  \item {
    Разница между $\Pi_k$ и $\Pi_{k + 1}$ - одна транспозиция
  }
  \item {
    $\forall i,j$ $\Pi_i \mapsto \Pi_j$ за конечное количество транспозиций
  }
\end{enumerate}

\subsubsection{Теорема об изменении четности}

\textbf{Формулировка:} любая транспозиция меняет четность перестановки

\begin{enumerate}
  \item {
    Рассмотрим случай транспозиции соседних элементов, то есть
    \begin{equation}
      (\alpha_1, \alpha_2, \dots, \alpha_i, \alpha_{i + 1}, \dots, \alpha_n) \Rightarrow
      (\alpha_1, \alpha_2, \dots, \alpha_{i + 1}, \alpha_i, \dots, \alpha_n)
    \end{equation}

    Очевидно, что взаимная перестановка $\alpha_i$ и $\alpha_{i + 1}$
    приводит к появлению или исчезновению инверсии между ними $\Rightarrow$
    меняет четность перестановки
  }
  \item {
    Теперь рассмотрим транспозицию элементов $\alpha_i$ и $\alpha_{i + k}$.
    Ее можно рассматривать, как $k$ последовательных
    транспозиций $\alpha_i$ с соседними элементами, расположенными справа
    от $\alpha_i$, и последующих $k - 1$ транспозиций элемента $\alpha_{i + k}$
    с соседними элементами, расположенными слева от $\alpha_{i + k}$
    \begin{equation}
      %\begin{aligned}
        (\dots, \overset{k \text{ транспозиций}}{\overrightarrow{\alpha_i, \underbrace{\dots, \alpha_{i + k}}_{k \text{ элементов}}}}, \dots)
      %\end{aligned}
    \end{equation}
    \begin{equation}
      (\dots, \overset{k - 1 \text{ транспозиция}}{\overleftarrow{\underbrace{\cancel{\alpha_i}, \:\: \dots,}_{k - 1 \text{ элементов}}}}  \alpha_{i + k}, \alpha_i, \dots)
    \end{equation}

    Таким образом, общее число перестановок $= k + (k - 1) = 2k - 1$ является
    нечетным числом $\Rightarrow$ происходит нечетное количество транспозиций
    соседних элементов $\Rightarrow$ четность перестановки меняется нечетное количество
    раз $\Rightarrow$ четность перестановки меняется $(\blacksquare)$
  }
\end{enumerate}

\subsection{Подстановки}

\textbf{Определение:} взаимно-однозначное соответствие

\subsubsection{Четность подстановки}

Подстановка является четной тогда и только тогда, когда перестановка
ключей и перестановка значений являются четными.

\subsubsection{Умножение подстановок}

\textbf{Важно:} умножение подстановок некоммутативно

\begin{equation}
  \begin{aligned}
    A = \begin{pmatrix}
      1 & 2 & 3 & 4 \\
      2 & 3 & 4 & 1
    \end{pmatrix} 
    & & B = \begin{pmatrix}
      1 & 4 & 2 & 3 \\
      2 & 3 & 4 & 1
    \end{pmatrix}\\ \\
    A \times B = \begin{pmatrix}
      1 & 2 & 3 & 4 \\
      4 & 1 & 3 & 2
    \end{pmatrix}
    & \neq & B \times A = \begin{pmatrix}
      1 & 4 & 2 & 3 \\
      3 & 4 & 1 & 2
    \end{pmatrix}
  \end{aligned}
\end{equation}

\subsubsection{Обратная подстановка}

\paragraph{Единичная подстановка}

Единичной подстановкой из n элементов называют подстановку вида:
\begin{equation}
  E = \begin{pmatrix}
    1 & 2 & \dots & n \\
    1 & 2 & \dots & n
  \end{pmatrix}
\end{equation}

Тогда подстановкой, обратной подстановке $A$ называют такую, для которой
выполняется следующее условие: $A\times A^{-1} = A^{-1}\times A = E$

\end{document}
