% Презентация для представления  решения задач коллоквиума по матанализу
% Михедов Константин Константинович БИБ224

% Объявляем презентацию
\documentclass[8pt]{beamer}

% Поиск по скомпилированному PDF
\usepackage{cmap}
% Кодировка выходного текста
\usepackage[T2A]{fontenc}
% Кодировка исходного текста
\usepackage[utf8]{inputenc}
% Поддержка необходимых языков
\usepackage[english,russian]{babel}

% Дополнительная математика
\usepackage{amsmath,amsfonts,amssymb,amsthm,mathtools}
% Показывать номера только у тех выржений, на которые кто-то ссылается
\mathtoolsset{showonlyrefs=true}

% Дополнительные символы
\usepackage{mathbbol}

% Какая-то тема
\usetheme{Berlin}
\usecolortheme{seahorse}

% Информация для заголовка
\title[Коллоквиум]
{Разбор задач к коллоквиуму}
\subtitle{Номера 16 - 20}
\author[БИБ224]
{Борисенкова~Д.А. \and Герцен~А.М. \and Ибрагимова~М.С. \and Колюх~Е.С. \and Михедов~К.К. \and Хисматуллина~А.А}
\institute[НИУ ВШЭ]
{НИУ Высшая Школа Экономики}
\date{Москва, 2022}

\begin{document}
    
  \frame{\titlepage}

  \begin{frame}
    \frametitle{Задание 16}    

    \begin{block}{Вопрос}
        Существуют ли числовые последовательности $x_n$ и $y_n$ со свойствами:
        \begin{enumerate}
            \item $\{x_n\}$ ограничена,
            \item $\{y_n\}$ ограничена и
            \item $\{z_n\} = \{x_n\} + \{y_n\}$ не ограничена?
        \end{enumerate}
    \end{block} \pause

    По определению суммы последовательностей $\{z_n\} = \{x_n + y_n: \forall n \in \mathbb{N}\}$ \pause

    Пусть последовательности $\{x_n\}$ и $\{y_n\}$ ограничены, тогда $\exists$ такие числа $a, b, c$ и $d$,
    что $a < x_n < b$ и $c < y_n < d$ верно $\forall n \in \mathbb{N}$ \pause

    Тогда можно утверждать, что $a + c < x_n + y_n < b + d$ также верно $\forall n \in \mathbb{N}$
    $\Rightarrow$ также верно и то, что $a + c < z_n < b + d$ $\forall n \in \mathbb{N}$ \pause

    Получается, что все элементы $\{z_n\}$ лежат выше $a+c$ и ниже $b + d$, то есть эта последовательность
    ограничена, что противоречит условию \pause

    \begin{flushright}
        Ответ: не существуют
    \end{flushright}
  \end{frame}

  \begin{frame}
    \frametitle{Задание 17}

    \begin{block}{Вопрос}
      Существуют ли числовые последовательности $x_n$ и $y_n$ со свойствами:
      \begin{enumerate}
        \item $\{x_n\}$ ограничена,
        \item $\{y_n\}$ не ограничена и
        \item $\{z_n\} = \{x_n\} + \{y_n\}$ ограничена?
      \end{enumerate}
    \end{block} \pause

    По определению суммы последовательностей $\{z_n\} = \{x_n + y_n: \forall n \in \mathbb{N}\}$ \pause

    Пусть последовательность $\{x_n\}$ ограничена, тогда $\exists$ такие числа $a$ и $b$, что
    $a < x_n < b$ верно $\forall n \in \mathbb{N}$,
    а последовательность $\{y_n\}$ не ограничена, то есть $\forall d > 0$ $\exists N \in \mathbb{N}$,
    что $\forall n > N$ верно $|y_n| > d$ \pause

    По свойствам модуля $|y_n| > d$ $\Leftrightarrow$ $y_n > d$ или $y_n < -d$ \pause

    Зафиксируем такое $d$, что $d > b$ и $-d < a$ (то же самое, что $-a < d$) \pause

    Тогда верно, что $a + d < x_n + y_n < b - d$ $\Rightarrow$ $a + d < z_n < b - d$. Отсюда
    видно, что последовательность $\{z_n\}$ ограничена (все ее члены $> a + d$ и $< b - d$) \pause
    
    Тогда должно быть верно, что $a + d < b - d$ $\Leftrightarrow$ $d + d < b - a$,
    но т.к. $-a < d$ и $b < d$ это высказывание неверно, возникает противоречие \pause

    \begin{flushright}
      Ответ: не существуют
    \end{flushright}
  \end{frame}

  \begin{frame}
    \frametitle{Задание 18}

    \begin{block}{Вопрос}
      Существуют ли числовые последовательности $x_n$ и $y_n$ со свойствами:
      \begin{enumerate}
        \item $\{x_n\}$ ограничена,
        \item $\{y_n\}$ не ограничена и
        \item $\{z_n\} = \{x_n\} \times \{y_n\}$ ограничена?
      \end{enumerate}
    \end{block} \pause

    По определению произведения последовательностей $\{z_n\} = \{x_n \times y_n: \forall n \in \mathbb{N}\}$ \pause

    Пусть последовательность $\{y_n\} = \{n \in \mathbb{N}\}$, а $\{x_n\} = \{0, 0, 0, \dots\}$. Тогда верны первые два условия \pause
  
    Отсюда $\{z_n\} = \{x_1y_1, x_2y_2, x_3y_3, \dots\} = \{0y_1, 0y_2, 0y_3, \dots\} = \{0, 0, 0, \dots\}$ \pause

    Видно, что все элементы $\{z_n\}$ равны нулю $\Rightarrow$ $\{z_n\}$ ограничена \pause

    \begin{flushright}
      Ответ: существуют
    \end{flushright}
  \end{frame}

  \begin{frame}
    \frametitle{Задание 19}

    \begin{block}{Вопрос}
      Существуют ли числовые последовательности $x_n$ и $y_n$ со свойствами:
      \begin{enumerate}
        \item $\{x_n\}$ сходится,
        \item $\{y_n\}$ расходится и
        \item $\{z_n\} = \{x_n\} + \{y_n\}$ сходится?
      \end{enumerate}
    \end{block} \pause

    Пусть эти свойства(условия) верны. \pause

    Известно, что сумма или разность сходящихся последовательностей представляет собой сходящуюся последовательность \pause
  
    Т.к. $\{z_n\} = \{x_n\} + \{y_n\}$, то $\{y_n\} = \{z_n\} - \{x_n\}$. Из свойств известно, что
    последовательности $\{z_n\}$ и $\{x_n\}$ сходятся, следовательно должна сходиться и $\{y_n\}$
    (она как раз является разностю сходящихся последовательностей), но это противоречит условию \pause

    \begin{flushright}
      Ответ: не существуют
    \end{flushright}
  \end{frame}

  \begin{frame}
    \frametitle{Задание 20}

    \begin{block}{Вопрос}
      Существуют ли числовые последовательности $x_n$ и $y_n$ со свойствами:
      \begin{enumerate}
        \item $\{x_n\}$ сходится,
        \item $\{y_n\}$ расходится и
        \item $\{z_n\} = \{x_n\} \times \{y_n\}$ сходится?
      \end{enumerate}
    \end{block} \pause

    По определению произведения последовательностей $\{z_n\} = \{x_n \times y_n: \forall n \in \mathbb{N}\}$ \pause

    Пусть последовательность $\{y_n\} = \{n \in \mathbb{N}\}$, а $\{x_n\} = \{0, 0, 0, \dots\}$. Тогда верны первые два условия \pause
  
    Отсюда $\{z_n\} = \{x_1y_1, x_2y_2, x_3y_3, \dots\} = \{0y_1, 0y_2, 0y_3, \dots\} = \{0, 0, 0, \dots\}$ \pause

    Видно, что все элементы $\{z_n\}$ равны нулю $\Rightarrow$ $\{z_n\}$ сходится \pause

    \begin{flushright}
      Ответ: существуют
    \end{flushright}
  \end{frame}

\end{document}
