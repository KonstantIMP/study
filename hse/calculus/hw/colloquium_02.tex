% Презентация для представления  решения задач коллоквиума по матанализу
% Михедов Константин Константинович БИБ224

% Объявляем презентацию
\documentclass[8pt]{beamer}

% Поиск по скомпилированному PDF
\usepackage{cmap}
% Кодировка выходного текста
\usepackage[T2A]{fontenc}
% Кодировка исходного текста
\usepackage[utf8]{inputenc}
% Поддержка необходимых языков
\usepackage[english,russian]{babel}

% Дополнительная математика
\usepackage{amsmath,amsfonts,amssymb,amsthm,mathtools}
% Показывать номера только у тех выржений, на которые кто-то ссылается
\mathtoolsset{showonlyrefs=true}

% Дополнительные символы
\usepackage{mathbbol}

% Какая-то тема
\usetheme{Berlin}
\usecolortheme{seahorse}

% Поддержка графиков
\usepackage{pgfplots}
\pgfplotsset{compat=1.9}

% Информация для заголовка
\title[Коллоквиум]
{Разбор задач к коллоквиуму}
\subtitle{Номера 35 - 37}
\author[БИБ224]
{Борисенкова~Д.А. \and Герцен~А.М. \and Ибрагимова~М.С. \and Колюх~Е.С. \and Михедов~К.К. \and Хисматуллина~А.А}
\institute[НИУ ВШЭ]
{НИУ Высшая Школа Экономики}
\date{Москва, 2022}

\begin{document}
    
  \frame{\titlepage}

  \begin{frame}
    \frametitle{Задание 35}    

    \begin{block}{Вопрос}
      Остается ли теорема Вейерштрасса об экстремумах верна, если в ее формулировке заменить
      отрезок интервалом?
    \end{block} \pause

    \begin{alertblock}{Теорема Вейерштрасса об экстремумах}
      Если функция $f(x)$ непрерывна на $[a; b]$, то она она ограничена на этом отрезке
      и достигает своего наибольшего и наименьшего значения (экстремумов)
    \end{alertblock} \pause

    Для примера возьмем функцию $f(x) = \frac{1}{x}$ на интервале $(0; +\infty)$ \pause

    Видно, что хоть $f(x) = \frac{1}{x}$ непрерывна на заданном интервале, ограниченной на
    нем она не является \pause

    \begin{flushright}
        Ответ: не остается
    \end{flushright}
  \end{frame}

  \begin{frame}
    \frametitle{Задание 36}

    \begin{block}{Вопрос}
      Бывает ли, что $\exists \lim\limits_{x \rightarrow a}f(x)$, $\nexists \lim\limits_{x \rightarrow a}g(x)$,
      но $\exists \lim\limits_{x \rightarrow a}(f(x) + g(x))$?
    \end{block} \pause

    \begin{alertblock}{Теорема о сумме пределов}
      Если $\exists \lim\limits_{x \rightarrow a}f(x) = l_1$ и $\exists \lim\limits_{x \rightarrow a}g(x) = l_2$,
      то $\exists$ предел суммы этих функций, причем верно, что
      $\lim\limits_{x \rightarrow a}(f(x) + g(x)) = \lim\limits_{x \rightarrow a}f(x) + \lim\limits_{x \rightarrow}g(x) = l_1 + l_2$
    \end{alertblock} \pause

    \begin{exampleblock}{Доказательство}
      Пусть $\lim\limits_{x \rightarrow a}f(x) = l_1$, а 
      $\lim\limits_{x \rightarrow a}g(x) = l_2$, тогда по теореме
      о связи функции, её предела и бесконечно малой функции:
      $f(x) = l_1 + \alpha (x)$ и $g(x) = l_2 + \beta (x)$ $\Rightarrow$
      $f(x) + g(x) = l_1 + l_2 + (\alpha (x) + \beta (x))$, где $\alpha (x)$ и $\beta (x)$
      - бесконечно-малые,
      тогда $\lim\limits_{x \rightarrow a}(f(x) + g(x)) = l_1 + l_2 = \lim\limits_{x \rightarrow a}f(x) + \lim\limits_{x \rightarrow}g(x)$
    \end{exampleblock} \pause

    %В вопросе несостыковка с теоремой $\Rightarrow$ возникает противоречие

    \begin{flushright}
      Ответ: не бывает
    \end{flushright}
  \end{frame}

  \begin{frame}
    \frametitle{Задание 37}

    \begin{block}{Вопрос}
      Бывает ли, что $\exists \lim\limits_{x \rightarrow a}f(x)$, $\nexists \lim\limits_{x \rightarrow a}g(x)$,
      но $\exists \lim\limits_{x \rightarrow a}(f(x) \times g(x))$?
    \end{block} \pause

    \begin{alertblock}{Теорема о произведении пределов}
      Если $\exists \lim\limits_{x \rightarrow a}f(x) = l_1$ и $\exists \lim\limits_{x \rightarrow a}g(x) = l_2$,
      то $\exists$ предел суммы этих функций, причем верно, что
      $\lim\limits_{x \rightarrow a}(f(x) \times g(x)) = \lim\limits_{x \rightarrow a}f(x) \times \lim\limits_{x \rightarrow}g(x) = l_1 \times l_2$
    \end{alertblock} \pause

    \begin{exampleblock}{Доказательство}
      Пусть $\lim\limits_{x \rightarrow a}f(x) = l_1$, а 
      $\lim\limits_{x \rightarrow a}g(x) = l_2$, тогда по теореме
      о связи функции, её предела и бесконечно малой функции:
      $f(x) = l_1 + \alpha (x)$ и $g(x) = l_2 + \beta (x)$ $\Rightarrow$
      $f(x) \times g(x) = (l_1 + \alpha (x))(l_2 + \beta (x)) = l_1l_2 + (l_1\beta(x) + l_2\alpha(x) + \alpha(x)\beta(x))$,
      где $\alpha (x)$ и $\beta (x)$
      - бесконечно-малые тогда $\lim\limits_{x \rightarrow a}(f(x) \times g(x)) = l_1l_2 = \lim\limits_{x \rightarrow a}f(x) \times \lim\limits_{x \rightarrow}g(x)$
    \end{exampleblock} \pause

    \begin{flushright}
      Ответ: не бывает
    \end{flushright}
  \end{frame}

\end{document}
