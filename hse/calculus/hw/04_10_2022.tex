% ДЗ по Математическому Анализу 04.10.2022
% Михедов Константин Константинович, БИБ 224

% Тип документа: статья, размер бумаги - A4, написано 14 кегелем
% Предназначено для импорта из другого документа
\documentclass[class=article,a4paper,12pt,crop=false]{standalone}

% Поиск по скомпилированному PDF
\usepackage{cmap}
% Кодировка выходного текста
\usepackage[T2A]{fontenc}
% Кодировка исходного текста
\usepackage[utf8]{inputenc}
% Поддержка необходимых языков
\usepackage[english,russian]{babel}

% Поддержка изображений
\usepackage{graphicx}
% Путь до внешних изображений
\graphicspath{ {./figures/}}

% Умная запятая
\usepackage{icomma}

% Ссылки на электронные ресурсы
\usepackage{hyperref}
% Настройка внешнего вида ссылок
\hypersetup{
    % Отключить прямоугольную рамку
    pdfborder={0 0 0},
    % Включить цветные ссылки
    colorlinks=true,
    % Цвет для ссылок на веб-ресурсы
    urlcolor=blue,
    % Цвет внутренних ссылок
    linkcolor=black
}

% Дополнительная математика
\usepackage{amsmath,amsfonts,amssymb,amsthm,mathtools}
% Показывать номера только у тех выржений, на которые кто-то ссылается
\mathtoolsset{showonlyrefs=true}

% Дополнительные символы
\usepackage{mathbbol}

% Подключние пакетов для импорта других .tex
\usepackage[subpreambles=true]{standalone}
\usepackage{import}

% Корректная нумерация вложенных списков
\renewcommand{\labelenumii}{\arabic{enumi}.\arabic{enumii}}

\begin{document}
  \textbf{Выполнил:} Михедов Константин Константинович БИБ224

  \subsection{Задача №1}

  \textbf{Условие:} доказать, что $\forall x_0 > 0$ и $\alpha > 1$ последовательность $x_n$ сходится
  \[x_{n} = \frac{x_{n - 1}}{\alpha + x_{n - 1}}\] 

  \begin{enumerate}
    \item {
        Докажем, что $x_n > 0$ $\forall n \in \mathbb{N}$

        \begin{enumerate}
            \item {
                Рассмотрим случай для $n = 1$

                \begin{equation}
                    x_1 = \frac{x_{1 - 0}}{x_{1 - 0} + \alpha} =
                    \frac{x_0}{x_0 + \alpha}
                \end{equation}

                Т.к. $x_0 > 0$ и $\alpha > 1$ (по условию),
                то $x_0 + \alpha > 1 $

                \begin{equation}
                    \begin{cases}
                        x_0 > 0 \\
                        x_0 + \alpha > 1
                    \end{cases}
                    \Rightarrow \frac{x_0}{x_0 + \alpha} > 0
                    \Rightarrow x_1 > 0
                \end{equation}
            }
            \item {
                Предположим, $x_n > 0$ для $n = k$
                
                \begin{equation}
                    x_k = \frac{x_{k - 1}}{x_{k - 1} + \alpha} > 0
                \end{equation}
            }
            \item {
                Рассмотрим случай для $n = k + 1$

                \begin{equation}
                    x_{k + 1} = \frac{x_{(k + 1) - 1}}{x_{(k + 1) - 1} + \alpha}
                    = \frac{x_k}{x_k + \alpha}
                \end{equation}

                Т.к. $x_k > 0$ (см 1.2) и $\alpha > 1$ (по условию), то
                $x_k + \alpha > 1$

                \begin{equation}
                    \begin{cases}
                        x_k > 0 \\
                        x_k + \alpha > 1
                    \end{cases}
                    \Rightarrow \frac{x_k}{x_k + \alpha} > 0
                    \Rightarrow x_{k + 1} > 0
                \end{equation}
            }
        \end{enumerate}

        \textbf{Вывод:} $x_n > 0$ $\forall x \in \mathbb{N}$
    }
    \item {
        Изучим монотонность функции

        \begin{multline}
            x_n \lor x_{n + 1} \Rightarrow x_n \lor \frac{x_n}{x_n + \alpha}
            \Rightarrow x_n - \frac{x_n}{x_n + \alpha} \lor 0 \Rightarrow \\
            \Rightarrow \frac{x_n(x_n + \alpha) - x_n}{x_n + \alpha} \lor 0
            \Rightarrow \frac{x_n((x_n + \alpha) - 1)}{x_n + \alpha} \lor 0 \Rightarrow \\
            \Rightarrow \frac{x_n(x_n + \alpha - 1)}{x_n + \alpha} \lor 0
        \end{multline}

        Т.к. $a > 1$ по условию, то $a - 1 > 0$. Исходя из пункта 1:
        $x_n > 0$ $\forall n \in \mathbb{N}$, тогда $x_n + \alpha > 1$
        и $x_n + \alpha - 1 > 0$

        \begin{multline}
            \begin{cases}
                x_n > 0 \\
                x_n + \alpha - 1 > 0 \\
                x_n + \alpha > 0
            \end{cases}
            \Rightarrow \frac{x_n(x_n + \alpha - 1)}{x_n + \alpha} > 0
            \Rightarrow \\ \Rightarrow x_n - x_{n + 1} > 0 \Rightarrow
            x_n > x_{n + 1} \: \forall n \in \mathbb{N}
        \end{multline}
    }
    \item {
        Так как $x_n > 0$ $\forall n \in \mathbb{N}$, то последовательность
        $x_n$ ограничена снизу.
        
        Последовательность $x_n$ монотонно убывает (см. 2) и
        ограничена снизу $\Rightarrow$ $x_n$ сходится по
        теореме Вейрштрасса, то есть \[\exists \lim\limits_{x \to + \infty}{x_n} = \inf (x_n)\]
    }
  \end{enumerate}

  \textbf{Вывод:} последовательность $x_n$ монотонно убывает ($\blacksquare$)

\end{document}
