% Семинар по Математическому Анализу 07.09.2022
% Михедов Константин Константиновчи, БИБ 224

% Тип документа: статья, размер бумаги - A4, написано 14 кегелем
% Предназначено для импорта из другого документа
\documentclass[class=article,a4paper,12pt,crop=false]{standalone}

% Поиск по скомпилированному PDF
\usepackage{cmap}
% Кодировка выходного текста
\usepackage[T2A]{fontenc}
% Кодировка исходного текста
\usepackage[utf8]{inputenc}
% Поддержка необходимых языков
\usepackage[english,russian]{babel}

% Поддержка изображений
\usepackage{graphicx}
% Путь до внешних изображений
\graphicspath{ {./figures/}}

% Умная запятая
\usepackage{icomma}

% Ссылки на электронные ресурсы
\usepackage{hyperref}
% Настройка внешнего вида ссылок
\hypersetup{
  % Отключить прямоугольную рамку
  pdfborder={0 0 0},
  % Включить цветные ссылки
  colorlinks=true,
  % Цвет для ссылок на веб-ресурсы
  urlcolor=blue,
  % Цвет внутренних ссылок
  linkcolor=black
}

% Дополнительная математика
\usepackage{amsmath,amsfonts,amssymb,amsthm,mathtools}
% Показывать номера только у тех выржений, на которые кто-то ссылается
\mathtoolsset{showonlyrefs=true}

% Дополнительные символы
\usepackage{mathbbol}

% Подключние пакетов для импорта других .tex
\usepackage[subpreambles=true]{standalone}
\usepackage{import}

\begin{document}

- это сопособ доказательства какого-либо высказывания 
для числовой последовательности вида $p_1, p_2 \dots p_n$

\subsection{Алгоритм доказательство} 

\begin{enumerate}
  \item {
    Доказать правдивость для фиксированного $n$ (например для $n = 1$)
  } \item {
    \textbf{Предположить,} что высказывание верно для какого-то $n = k$
  } \item {
    Подтвердить правдивость для $n = k + 1$
  }
\end{enumerate}

\subsection{Пример}

\textbf{Доказать:} формулу суммы $n$ членов арифметической прогрессии
\begin{align*}
  \exists a_1, d & & a_n = a_1 + d(n - 1) & & \sum\limits_{i = 1}^n{a_i} = a_1 + a_2 + \dots + a_n
\end{align*}
\begin{equation}
  \text{Необходимо доказать, что }\sum\limits_{i = 1}^n{a_i} = \frac{n (a_1 + a_n)}{2}
\end{equation}

\begin{enumerate}
  \item {
    Рассмотрим случай, когда $n = 2$
    \begin{equation}
      \begin{aligned}
        \sum\limits_{i = 0}^2{a_i} & = a_1 + a_2 \\
        \sum\limits_{i = 0}^2{a_i} & = \frac{2(a_1 + a_2)}{2} = a_1 + a_2
      \end{aligned}
    \end{equation}
  }
  \item {
    Допустим, что данное выражение верно для $n = k$
  }
  \item {
    Рассмотрим случай, когда $n = k + 1$
    \begin{multline}
      \sum\limits_{i = 0}^{k + 1}{a_i} = a_1 + a_2 + \dots + a_k +
      a_{k + 1} = \sum\limits_{i = 0}^k{a_i} + a_{k + 1} = \\
      = \frac{k(a_1 + a_k)}{2} + a_{k + 1} 
      = \text{*какие-то вычисления*} = \\
      = \frac{(k + 1)(a_1 + a_{k + 1})}{2} \:\:\: (\blacksquare)
    \end{multline}
  }
\end{enumerate}

\end{document}
