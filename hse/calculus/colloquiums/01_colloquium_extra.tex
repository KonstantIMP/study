% Задания и решения к первому коллоквиуму по матанализу
% Михедов Константин Константиновчи БИБ224

% Тип документа: статья, размер бумаги - A4, написано 14 кегелем
\documentclass[a4paper,12pt]{article}

% Поиск по скомпилированному PDF
\usepackage{cmap}
% Кодировка выходного текста
\usepackage[T2A]{fontenc}
% Кодировка исходного текста
\usepackage[utf8]{inputenc}
% Поддержка необходимых языков
\usepackage[english,russian]{babel}

% Поддержка изображений
\usepackage{graphicx}
% Путь до внешних изображений
\graphicspath{ {./../figures/}}

% Умная запятая
\usepackage{icomma}

% Ссылки на электронные ресурсы
\usepackage{hyperref}
% Настройка внешнего вида ссылок
\hypersetup{
  % Отключить прямоугольную рамку
  pdfborder={0 0 0},
  % Включить цветные ссылки
  colorlinks=true,
  % Цвет для ссылок на веб-ресурсы
  urlcolor=blue,
  % Цвет внутренних ссылок
  linkcolor=black
}

% Дополнительная математика
\usepackage{amsmath,amsfonts,amssymb,amsthm,mathtools}
% Показывать номера только у тех выржений, на которые кто-то ссылается
\mathtoolsset{showonlyrefs=true}

% Функция sgn
\DeclareMathOperator{\sgn}{sgn}

% Дополнительные символы
\usepackage{mathbbol}

% Подключние пакетов для импорта других .tex
\usepackage[subpreambles=true]{standalone}
\usepackage{import}

\begin{document}
    % Титульная страница
    \begin{titlepage}
      \begin{center}
        \vspace*{1.5cm}
  
        \Huge
        \textbf{Математический анализ}
  
        \vspace{0.5cm} \large
        \textbf{Доп.вопросы к первому коллоквиуму}
  
        \vspace{1.5cm} \normalsize
        \textbf{Михедов Константин Константинович}
  
        \vfill
  
        \includegraphics[width=0.2\textwidth]{hse_logo}
  
        \vspace{1cm} \footnotesize
        Национальный Исследовательский Университет "Высшая Школа Экономики" \\
        Московский Институт Электроники и Математики им. А. Н. Тихонова \\
        Информационная Безопасность, БИБ224 \\
        Москва, 2022
      \end{center}
    \end{titlepage}

  \begin{enumerate}
    \item {
        \textbf{Может ли числовое множество имень нижнюю грань, но не иметь минимума?}

        Да, может. Например множество $x_n = \frac{1}{n}$ при $n \in \mathbb{N}$ имеет бесконечно много
        нижних граней (неточных) (любое число, лежащее в полуинтервале $(-\infty; 0]$), но не имеет минимума
    }
    \item {
        \textbf{Может ли числовое множество быть ограниченным, но не иметь верхней грани?}

        Нет, не может. По теоремам о точных гранях, если множетсво ограничено сверху/снизу, то
        $\sup$/$\inf$ этого множетсва $\in \mathbb{R}$, то есть существует и является числом

        \textbf{Доказательство для точной нижней грани:} 
        
        Пусть $\mathbb{X} \subseteq \mathbb{R}$ ограничено снизу
    
        Рассмотрим множество $\mathbb{S}$ всех таких $s \in \mathbb{R}$,
        которые ограничивают $\mathbb{X}$ слева:
        $\mathbb{S} = \left\{s \in \mathbb{R}: s \leq \mathbb{X}\right\}$.
        Из определения $\mathbb{S}$ видно, что $\mathbb{S} \leq \mathbb{X}$,
        то есть $\forall x \in \mathbb{X} \land \forall s \in \mathbb{S}: s \leq x$
        $\Rightarrow \exists c \in \mathbb{R}: \mathbb{S} \leq c \leq \mathbb{R}$
    
        Тогда $c$ является одной из тех $s$ для которых $s \leq \mathbb{X}$
        $\Rightarrow c \in \mathbb{S}$

        Т.к. $\mathbb{S} \leq c$ и $c \in \mathbb{S}$
        $\Rightarrow c = \max{\mathbb{S}} = \inf{\mathbb{X}} = c \in \mathbb{R}$
        $\; (\blacksquare)$
    }
    \item {
        \textbf{Верна ли для числовых множеств следущющая импликация: $\mathbb{X} \leq \mathbb{Y} \:\&\: \mathbb{Y} \leq \mathbb{X} \Rightarrow \mathbb{X} = \mathbb{Y}$}
        
        Да, верна. Если $\mathbb{X} \leq \mathbb{Y}$, то любой элемент множества $\mathbb{X}$ меньше или равен,
        чем любой элемент множества $\mathbb{Y}$, также если $\mathbb{Y} \leq \mathbb{X}$, то любой элемент 
        множетсва $\mathbb{Y}$ меньше любого элемента множества $\mathbb{X}$.

        Тогда пусть $a$ - любой элемент множества $\mathbb{X}$, а $b$ - любой элемент множества $\mathbb{Y}$.
        Из ранее сказанного получается, что $a \leq b$ и $b \leq a$ верно всегда, тогда по свойству транзитивности
        можно утверждать, что $a = b$.

        \textbf{P.s.} множества $\mathbb{X}$ и $\mathbb{Y}$ состоят всего из одного элемента, иначе такие
        условия выполнить невозможно
    }
    \item {
      \textbf{Существует ли ограниченное индуктивное множество?}

      Нет, не существует. Наименьшим индуктивным множеством является множество натуральных чисел $\mathbb{N}$,
      то есть любое другое индуктивное множество является подмножеством $\mathbb{N}$. Т.к. множество $\mathbb{N}$
      неограниченно сверху (по лемме) и бесконечно $\Rightarrow$ все его подмножества также бесконечны и неограничены сверху
    }
    \item {
      \textbf{Может ли число быть целым, но не рациональным?}

      Нет, не может. Во-первых потому, что множество $\mathbb{Z}$ является подмножеством $\mathbb{Q}$.
      Или еще можно сказать, что $\mathbb{Q} = \{\frac{z}{n}, z \in \mathbb{Z}, n \in \mathbb{N}^{*}\}$,
      будет эквивалентно множеству $\mathbb{Z}$ при $n = 1$, что еще раз подтверждает предыдущее высказывание
    }
    \item {
      \textbf{Может ли число быть рациональным, но не целым?}

      Да, может. Например число $\frac{3}{2}$ является рациональным, но не является целым
    }
    \item {
      \textbf{Существует ли числовая последовательность $x_n$ со следующими свойствами}
      \begin{itemize}
        \item Почти все элементы $x_n$ лежат  в $(0; 1)$ и
        \item $x_n$ стремится к 2
      \end{itemize}

      Нет, не существует. Если последовательность $x_n \rightarrow 2$, то почти все ее элементы лежат
      в эпсилон окрестности числа 2, то есть в интервале $(2 - \varepsilon; 2 + \varepsilon)$, где
      $\varepsilon$ - действительное число большее нуля.
      
      Чтобы было верно, что $x_n \rightarrow 2$ и что почти все ее члены лежат в $(0; 1)$, то 
      $(2 - \varepsilon; 2 + \varepsilon)$ должно стать равно $(0; 1)$, то есть
      $(2 - \varepsilon; 2 + \varepsilon) = (0; 1)$, что равносильно системе из двух уравнений:
      $2 - \varepsilon = 0$ и $2 + \varepsilon = 1$, которую невозможно решить при заданном $\epsilon$
    }
    \item {
      \textbf{Существует ли числовая последовательность $x_n$ со следующими свойствами}
      \begin{itemize}
        \item Почти все элементы $x_n$ лежат в $(0; 1)$ и
        \item почти все элементы $x_n$ не лежат в $(0; 1)$?
      \end{itemize}

      Нет, не существует. Допустим, первое выражение верно и почти все элементы $x_n \in (0; 1)$,
      тогда только конечное (счетное) количество элементов $x_n$ лежат вне этого интервала, что
      противоречит второму условию  
    }
    \item {
      \textbf{Существует ли ограниченная, но не сходящаяся последовательность?}

      Да, например последовательность $x_n$ вида
      \begin{equation}
        x_n = \begin{cases}
          1 &\text{ если } n = 2k + 1 \\
          -1 &\text{ если } n = 2k
        \end{cases}\: n \in \mathbb{N} \: k \in \mathbb{Z}
      \end{equation}
      ограничена сверху числом 1, и снизу числом -1, однако не сходится.
    }
    \item {
      \textbf{Существует ли сходящаяся, но не ограниченная сверху последовательность?}

      Нет, не существует, потому что всякая сходящаяся последовательность ограничена

      \textbf{Доказательство:} пусть последовательность $x_n$ сходится, тогда для любого
      $\varepsilon > 0$ можно найти такое число $a$, что $\exists N \in \mathbb{N}: \forall n > N : |x_n - a| < \varepsilon$.
      Пусть $r = \max\left\{|x_1 - a|, |x_2 - a|, \dots, |x_N - a|, 1\right\}$, тогда
      можно утверждать, что $\forall n \in \mathbb{N}$ верно $|x_n - a| leq r$ $\Rightarrow$
      последовательность $x_n$ ограничена, что противоречит условию.
    }
    \item {
      \textbf{Существует ли такая последовательность  $x_n$, у которой никакая подпоследовательность не сходится?}
    
      Если последовательность $x_n$ ограничена, то по теореме Больцано-Вейерштрасса она имеет 
      сходящуюся подпоследовательность. Также если $x_n$ сходится, то есть $\rightarrow a$, то
      любая ее подпоследовательность также сходится и $\rightarrow a$.
      
      Остается рассмотреть неограниченную и несходящуюся последовательность $x_n$. Например из
      последовательности вида $x_n = (-1)^nn$, где $n \in \mathbb{N}$, видно, что она не имеет
      никакой сходящейся подпоследовательности ($x_n = \{-1, 2, -3, 4, -5, 6, -7, \dots\}$)
      $\Rightarrow$ такая последовательность существует

      \textbf{P.s.} здесь, наверное, можно сказать, что если исходная последовательность $x_n \rightarrow b$
      , то любая ее подпоследовательность также $\rightarrow b$ при $b = \pm\infty$ или $b \in \mathbb{R}$ $\Rightarrow$
      $x_n = n$ также не имеет никакой схоядщейся подпоследовательности
    }
    \item {
      \textbf{Существует ли неограниченная последовательность, имеющая сходящуюся подпоследовательность?}

      Да, например последовательность вида
      \begin{equation}
        x_n = \begin{cases}
          n &\text{ если } n = 2k + 1 \\
          \frac{1}{n} &\text{ если } n = 2k
        \end{cases} n \in \mathbb{N}, k \in \mathbb{Z}
      \end{equation}
      неограниченна, однако имеет сходящуюся подпоследовательность вида $x_{y_p},\: y_p = \{2, 4, 6, \dots\}$
      (получается, что $x_{y_p} = \{1/2, 1/4, 1/6, \dots\}$ сходится)
    }
    \item {
      \textbf{Существует ли ограниченная последовательность, у которой никакая подпоследовательность не сходится?}
    
      Нет, не существует, т.к. по теореме Больцано-Вейерштрасса из любой ограниченной числовой последовательности
      можно выделить сходящуюся подпоследовательность

      \textbf{А доказательство теоремы надо?}
    }
    \item {
      \textbf{Существует ли бесконечно большая последовательность, содержащая сходящуюся подпоследовательность?}

      Нет, не существует, т.к. если исходная последовательность бесконечно большая,
      то и любая ее подпоследовательность также бесконечно большая, то есть не сходится (теорема такая есть)

      \textbf{А доказательство теоремы надо?}
    }
    \item {
      \textbf{Существует ли бесконечно малая последовательность, содержащая расходящуюся подпоследовательность?}

      Нет, не существует. Если последовательность бесконечно малая, то она стремится к 0, а $\Rightarrow$
      любая ее подпоследовательность также стремится к 0, то есть является сходящейся  
    }
    \item {
      \textbf{Существуют ли последовательности $x_n$ и $y_n$ со следующими свойствами}
      \begin{itemize}
        \item $\{x_n\}$ ограничена,
        \item $\{y_n\}$ ограничена и
        \item $\{x_n + y_n\}$ не ограничена?
      \end{itemize}

      Ограниченные последовательности $\{x_n\}$ и $\{y_n\}$ можно задать в любом случае. Если $\{x_n\}$ ограниченна,
      то любой ее элемент больше какого-то числа $a$ и меньше какого-то числа $b$, аналогично для $\{y_n\}$.
      То есть $a < x_n < b$ и $c < y_n < d$ верно $\forall n \in \mathbb{N}$
      $a + c < x_n + y_n < b + d$ верно $\forall n \in \mathbb{N}$, то есть любой элемент
      последовательности $\{x_n + y_n\}$ больше $a + c$ и меньше $b + d$ $\Rightarrow$
      последовательность $\{x_n + y_n\}$ ограниченна

      Следовательно, такие последовательности существовать не могут
    }
    \item {
      \textbf{Существуют ли последовательности $x_n$ и $y_n$ со следующими свойствами}
      \begin{itemize}
        \item $\{x_n\}$ ограничена,
        \item $\{y_n\}$ не ограничена и
        \item $\{x_n + y_n\}$ ограничена?
      \end{itemize}

      Пусть последовательность $\{x_n\}$ ограничена $\Rightarrow$ $\exists a,b \in \mathbb{R}$ что
      $a < x_n < b$ верно $\forall n \in \mathbb{N}$, а последовательность $\{y_n\}$ не
      ограничена $\Rightarrow \forall d > 0 \: \exists N \in \mathbb{N}: \forall n > N$ верно $|y_n| > d$
      , что равносильно совокупности уравнений $y_n > d$ и $y_n < -d$

      Выберем такое число $d$, что $d > b$ и $-d < a$ ($d > -a$), тогда верно, что
      $a + d < x_n + y_n < b - d$ $\Rightarrow$ $a + d < b - d$ $\Leftrightarrow$
      $d + d < b - a$ что неверно (т.к. $d > b$ и $d > -a$, то $d + d > b - a$)
      $\Rightarrow$ таких последовательностей не существует
    }
    \item {
      \textbf{Существуют ли последовательности $x_n$ и $y_n$ со следующими свойствами}
      \begin{itemize}
        \item $\{x_n\}$ ограничена,
        \item $\{y_n\}$ не ограничена и
        \item $\{x_n \times y_n\}$ ограничена?
      \end{itemize}

      Да, существует. Например $\{x_n\} = \{0\}, \: \forall n \in \mathbb{N}$ ограничена, $\{y_n\} = \{n\}, \: \forall n \in \mathbb{N}$
      неограниченна, однако $\{x_n\times y_n\} = \{0 \times n\} = \{0\}, \: \forall n \in \mathbb{N}$ ограничена
    }
    \item {
      \textbf{Существуют ли последовательности $x_n$ и $y_n$ со следующими свойствами}
      \begin{itemize}
        \item $\{x_n\}$ сходится,
        \item $\{y_n\}$ расходится и
        \item $\{x_n + y_n\}$ сходится?
      \end{itemize}

      Пусть последовательность $\{z_n\} = \{x_n + y_n\}$ сходится, тогда по теореме Коши
      последовательность $\{y_n\} = \{z_n - x_n\}$ также должна сходиться, т.к. является
      разностью двух сходящихс подпоследовательностей, но это противоречит условию
    }
    \item {
      \textbf{Существуют ли последовательности $x_n$ и $y_n$ со следующими свойствами}
      \begin{itemize}
        \item $\{x_n\}$ сходится,
        \item $\{y_n\}$ расходится и
        \item $\{x_n \times y_n\}$ сходится?
      \end{itemize}

      Да, существует. Например последовательности $\{x_n\} = \{0, 0, 0, \dots\}$ и
      $\{y_n\} = \{1, 2, 3, \dots\}$
    }
    \item {
      \textbf{Существует ли монотонная последовательность, не содержащая монотонных подпоследовательностей?}

      Нет, не существует, потому что вообще любая последовательность имеет монотонную подпоследовательность
    }
    \item {
      \textbf{Существует ли монотонная последовательность, не содержащая сходящихся подпоследовательносте?}

      Да, например $x_n = n$ монотонно возрастает, но сходящихся подпоследовательностей не содержит
    }
    \item {
      \textbf{Бывает ли, что $\lim{x_n} \neq \sup{x_n}$?}

      Да, например у последовательности вида $x_n = \frac{-1^n}{n}$
    }
    \item {
      \textbf{Бывает ли, что $\lim{x_n}$ существует, а $\sup{x_n}$ - нет?}

      Да, например у последовательности вида $x_n = \frac{1}{n}$
    }
    \item {
      \textbf{Существует ли непрерывная, но не ограниченная функция на каком-нибудь множестве $\mathbb{M}$?}

      Да, например $f(x) = \frac{1}{x}$ на множестве $(0; 1)$
    }
    \item {
      \textbf{Существет ли ограниченная, но разрывная функция на некотором множестве $\mathbb{M}$?}

      Да, например $f(x) = \frac{x}{x}$ на множестве $(-1; 1)$
    }
    \item {
      \textbf{Бывает ли, что функция $f$ непрерывна на множестве $\mathbb{M}$, функция $g$ на нем разрывна, а функция $f + g$ непрерывна на нем?}

      Нет, не бывает. Если функция $s = f + g$ непрерывна, то функция $g = s - f$ также непрерывна, 
      что противоречит условию
    }
    \item {
      \textbf{Бывает ли, что функция $f$ непрерывна на множестве $\mathbb{M}$, функция $g$ на нем разрывна, а функция $f \times g$ непрерывна на нем?}

      Да, бывает. Например $f(x) = 0$ и $g(x) = \sgn (x)$ на множестве $(-1; 1)$. Тогда $f(x)$ непрерывна,
      $g(x)$ разрывна, а их произвдение $s(x) = f(x)\times g(x) = 0$ на всем множестве является непрерывной
      функцией
    }
    \item {
      \textbf{Бывает ли, что функция $f$ непрерывна на множестве $\mathbb{M}$, функция $g$ разрывна на множестве $\mathbb{P}$, а функция $f \circ g$ непрерывна на множестве $\mathbb{M}$?}

      Бывает, но только в том случае, если множества $\mathbb{P}$ и $\mathbb{M}$ не пересекаются (то есть
      $\mathbb{P}\cap\mathbb{M} = \emptyset$) и функция $g$ непрерына за пределами множества $\mathbb{P}$
    }
    \item {
      \textbf{Остается ли теорема о сохранении знака верна, если из ее формулировки выбросить требование о непрерывности функции $f$?}

      Нет, не остается. Смотрите доказательство теоремы о сохранении знака с лекции.

      \textbf{Существет ли функция $f$ со следующими свойствами:}
      \begin{itemize}
        \item $f$ определена на $\mathbb{R}$,
        \item $f(x) \neq 0$ и
        \item $\nexists$ окрестности нуля $U$ в которой $\sgn f(x) = \sgn f(0)$
      \end{itemize}

      Если функция $f$ непрервна, то исхояди из второго условия видно, что 0 - это ее верхння,
      или нижняя грань $\Rightarrow \sgn f(x) = const$ $\forall x$ $\Rightarrow$ все условия не выполняются

      Если же функция разрывна, то такие условия могу выполняться для:
      \begin{equation}
        f(x) = \begin{cases}
          1 & \text{ если } x = 0 \\
          -1 & \text{ в других случаях }
        \end{cases}
      \end{equation}
    }
    \item {
      \textbf{Остается ли теорема Коши о промежуточном значении верна, если из ее формулировки выбросить требование о непрерывности функции $f$?}

      Нет, не остается. Пример $f(x) = \frac{1}{x}$ на интервале $(-1; 1)$. $f(-1) = -1$ и $f(1) = 1$
      $\Rightarrow$ по теореме Коши должно существовть такое $-1 < c < 1$, что $f(c) = 0$, но уравнение
      $\frac{1}{c} = 0$ не имеет решений
    }
    \item {
      \textbf{Остается ли теорема Вейерштрасса об ограниченности верна, если из ее формулировки выбросить требование о непрерывности функции $f$?}

      Нет, не остается. Пример $f(x) = \frac{1}{x}$ при $x \neq 0$, иначе $f(x) = 0$ на отрезке $[-1; 1]$. Она разрывна в точке 0 и не ограничена.
    } 
    \item {
      \textbf{Остается ли теорема Вейерштрасса об ограниченности верна, если в ее формулировке заменить отрезок интервалом?}

      Нет, не остается. Пример $f(x) = \frac{1}{x}$ на интервале $(0; 1)$ не является ограниченной
    } 
    \item {
      \textbf{Остается ли теорема Вейерштрасса об экстремумах верна, если из формулировки выбросить требование о непрерывности функции?}

      Нет, не остается. Пример $f(x) = \frac{1}{x}$ при $x \neq 0$, иначе $f(x) = 0$ на отрезке $[-1; 1]$ не достигает
      своих локальных экстремумов, потому что их не существует
    }
    \item {
      \textbf{Остается ли теорема Вейерштрасса об экстремумах верна, если в ее формулировке заменить отрезок интервалом?}

      Нет, не остается. Пример $f(x) = \frac{1}{x}$ при $x \neq 0$, иначе $f(x) = 0$ на интервале $(0; 1)$ не достигает
      своего локального максимума, потому что его не существует
    }
    \item {
      \textbf{Бывает ли, что сущесвует $\lim\limits_{x \rightarrow a}f(x)$, не существует $\lim\limits_{x \rightarrow a}g(x)$, но сущесвует $\lim\limits_{x \rightarrow a}(f(x) + g(x))$?}

      По определению предела суммы $\lim\limits_{n \rightarrow a}(f(x) + g(x)) = \lim\limits_{n \rightarrow a}f(x) + \lim\limits_{n \rightarrow a}g(x)$
      и существует только тогда, когда существуют оба слагаемых (предела), что в данном случае не выполняется
    }
    \item {
      \textbf{Бывает ли, что сущесвует $\lim\limits_{x \rightarrow a}f(x)$, не существует $\lim\limits_{x \rightarrow a}g(x)$, но сущесвует $\lim\limits_{x \rightarrow a}(f(x) \times g(x))$?}

      По определению предела произведения $\lim\limits_{n \rightarrow a}(f(x) \times g(x)) = \lim\limits_{n \rightarrow a}f(x) \times \lim\limits_{n \rightarrow a}g(x)$
      и существует только тогда, когда существуют оба множителя (предела), что в данном случае не выполняется
    }
    \item {
      \textbf{Бывает ли, что не существует $\lim\limits_{x\rightarrow a}f(x)$, $g$ непрерывна, но существует $\lim\limits_{x \rightarrow a}g(f(x))$?}

      Пусть \begin{equation}
        f(x) = \begin{cases}
          a &\text{ если } x \geq a \\
          -a &\text{ в других случаях}
        \end{cases} \text{ и }
        g(x) = 0 \times x
      \end{equation} Тогда компизиция функций
      \begin{equation}
        g(f(x)) = 0 \times f(x) = 0
      \end{equation}
      имеет предел в точке $a$ и непрерывна
    }
    \item {
      \textbf{Бывает ли, что не существует $\lim\limits_{x\rightarrow a}f(x)$, $g$ разрывна, но существует $\lim\limits_{x \rightarrow a}g(f(x))$?}

      Пусть \begin{equation}
        f(x) = \begin{cases}
          1 &\text{ если } x \geq a \\
          2 &\text{ в других случаях}
        \end{cases} \text{ и }
        g(x) = \sgn x
      \end{equation} Тогда компизиция функций
      \begin{equation}
        g(f(x)) = \sgn f(x)
      \end{equation}
      имеет предел в точке $a$ и разрывна
    }
  \end{enumerate}
\end{document}
