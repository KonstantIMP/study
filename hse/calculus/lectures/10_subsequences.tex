% Лекция по Математическому Анализу 17.10.2022
% Михедов Константин Константиновчи, БИБ 224

% Тип документа: статья, размер бумаги - A4, написано 14 кегелем
% Предназначено для импорта из другого документа
\documentclass[class=article,a4paper,12pt,crop=false]{standalone}

% Поиск по скомпилированному PDF
\usepackage{cmap}
% Кодировка выходного текста
\usepackage[T2A]{fontenc}
% Кодировка исходного текста
\usepackage[utf8]{inputenc}
% Поддержка необходимых языков
\usepackage[english,russian]{babel}

% Поддержка изображений
\usepackage{graphicx}
% Путь до внешних изображений
\graphicspath{ {./figures/} {./../figures/}}

% .esp support
\usepackage{epstopdf}

% Умная запятая
\usepackage{icomma}

% Ссылки на электронные ресурсы
\usepackage{hyperref}
% Настройка внешнего вида ссылок
\hypersetup{
  % Отключить прямоугольную рамку
  pdfborder={0 0 0},
  % Включить цветные ссылки
  colorlinks=true,
  % Цвет для ссылок на веб-ресурсы
  urlcolor=blue,
  % Цвет внутренних ссылок
  linkcolor=black
}

% Дополнительная математика
\usepackage{amsmath,amsfonts,amssymb,amsthm,mathtools}
% Показывать номера только у тех выржений, на которые кто-то ссылается
\mathtoolsset{showonlyrefs=true}

% Дополнительные символы
\usepackage{mathbbol}

% Подключние пакетов для импорта других .tex
\usepackage[subpreambles=true]{standalone}
\usepackage{import}

% Функция card
\DeclareMathOperator{\card}{card}

% Теперь мы еще и графики рисуем
\usepackage{pgfplots}
\usetikzlibrary{
  calc,
  pgfplots.groupplots,
}
\pgfplotsset{compat=1.16}

% Списки в два и более стобцов
\usepackage{multicol}

% Перечеркивание элементов
\usepackage{cancel}

\begin{document}

\subsection{Подпоследовательности}

Пусть $\{x_n\}$ - последовательность действительных чисел, а $\{n_k\}$ - последовательность
натуральных (бесконечно-большая), тогда $\{y_k\} = \{x_{n_k}\}$ - подпоследовательность $\{x_n\}$

\subsubsection{Свойства подпоследовательностей}

\begin{enumerate}
    \item {
        Если $x_n \underset{n \rightarrow \infty}{\rightarrow} a$, то любая ее
        подпоследовательность тоже $\underset{k \rightarrow \infty}{\rightarrow} a$
    }
    \item {
        Если $x_n \underset{n \rightarrow \infty}{\cancel{\rightarrow}} a$, то $\exists$
        некие $\{x_{n_k}\}$ и $\varepsilon > 0$: $\forall k$
        $x_{n_k} \notin v_{\varepsilon}(a)$
    }
    \item {
        Если $\{x_n\}$ неограничена ($\sup{|x_n|} = \infty$), то 
        $\exists \{x_{n_k}\}$: $x_{n_k} \underset{k \rightarrow \infty}{\rightarrow} \infty$
    }
    \item {
        Если $x_n \in E$ верно для бесконечно-большого набора чисел $n \in \mathbb{N}$, то
        $\exists$ подпоследовательность $\{x_{n_k}\}$ такая, что $x_{n_k} \in E$ $\forall k$
    }
\end{enumerate}

\subsubsection{Теорема Больцано-Вейерштрасса}

Если $\{x_n\}$ ограничена ($\sup{|x_n|} < \infty$ $\forall n \in \mathbb{N}$), то у нее 
обязательно имеется сходящаяся подпоследовательность ($\exists$ $x_{n_k} \underset{k \rightarrow \infty}{\rightarrow} a \in \mathbb{R}$ )

\subsubsection{Критерии Коши}

Для числовой последовательности $\{x_n\}$ следующие условия эквивалентны:

\begin{enumerate}
    \item {
        $\{x_n\}$ сходится $\Leftrightarrow$ $\exists \lim{x_n} \in \mathbb{R}$
    }
    \item {
        $\forall p_k$ и $q_k$ $\underset{k \rightarrow \infty}{\rightarrow} \infty$ $\Leftrightarrow$
        $(x_{p_k} - x_{q_k}) \underset{k \rightarrow \infty}{\rightarrow} 0$
    }
    \item {
        Для любой последовательности $\{\gamma_n\} \in \mathbb{N}$ $(x_{k + \gamma_k} - x_k) \underset{k \rightarrow \infty}{\rightarrow} 0$
    }
\end{enumerate}

\subsubsection{Число Непера (e)}

\paragraph{Лемма} $\exists \lim\limits_{n \rightarrow \infty}(1 + \frac{1}{n})^n = e \in \mathbb{R}$

\paragraph{Доказательство} Рассмотрим последовательность $\{y_n\} = \{(1 + \frac{1}{n})^{n + 1}\}$

\begin{multline}
    \frac{y_{n - 1}}{y_n} =
    \frac{(1 + \frac{1}{n - 1})^n}{(1 + \frac{1}{n})^{n + 1}} =
    \frac{\left(\frac{n}{n - 1}\right)^n}{\left(\frac{n + 1}{n}\right)^{n + 1}} =
    \frac{\frac{n^n}{(n-1)^n}}{\frac{(n + 1)(n + 1)^n}{n^{n + 1}}} = \\
    = \frac{n \times n^{n + 1}}{(n - 1)^n(n + 1)(n + 1)^n} =
    \frac{n\times n^{2n}}{(n + 1)(n^2 - 1)^n} = \\
    = \left(\frac{n^2}{n^2 - 1}\right)^n \times \frac{n}{n + 1} =
    \left(1 + \frac{1}{n^2 - 1}\right)^n \times \frac{n}{n + 1}
\end{multline}

По неравенству Бернулли: $\left(1 + \frac{1}{n^2 - 1}\right)^n \geq 1 + \frac{n}{n ^2 -1}$ $\Rightarrow$

\begin{equation}
    \frac{y_{n - 1}}{y_n} \geq \left(1 + \frac{n}{n^2 - 1}\right)\frac{n}{n + 1}
    \overset{\frac{n}{n^2 - 1} \geq \frac{n}{n^2}}{\geq}
    (1 + \frac{n}{n^2})\frac{n}{n + 1} = 1
\end{equation}

$\Rightarrow$ $\frac{y_{n - 1}}{y_n} \geq 1$ $\forall n \in \mathbb{N}$ $\Rightarrow$
$y_{n - 1} \geq y_n$ $\Leftrightarrow$ $y_n \geq y_{n + 1}$ $\Rightarrow$ $\{y_n\}$
невозрастающая

Получается, что $\{y_n\}$ монотонная (невозрастающая) и всегда $> 0$ (из определения)
$\Rightarrow$ $\{y_n\}$ имеет действительный предел

\begin{equation}
    \lim\limits_{n \rightarrow \infty}{\left(1 + \frac{1}{n}\right)^n} =
    \lim\limits_{n \rightarrow \infty}{\frac{\left(1 + \frac{1}{n}\right)^{n + 1}}{1 + \frac{1}{n}}} =
    \frac{\lim\limits_{n \rightarrow \infty}y_n}{\lim\limits_{n \rightarrow \infty}{\left(1 + \frac{1}{n}\right)}} =
    \frac{\lim\limits_{n \rightarrow \infty}y_n}{1} = \lim\limits_{n \rightarrow \infty}y_n
\end{equation}

Данный предел существует и равен $e$ (без доказательства) $(\blacksquare)$

\end{document}
