% Лекция по Математическому Анализу 19.09.2022
% Михедов Константин Константиновчи, БИБ 224

% Тип документа: статья, размер бумаги - A4, написано 14 кегелем
% Предназначено для импорта из другого документа
\documentclass[class=article,a4paper,12pt,crop=false]{standalone}

% Поиск по скомпилированному PDF
\usepackage{cmap}
% Кодировка выходного текста
\usepackage[T2A]{fontenc}
% Кодировка исходного текста
\usepackage[utf8]{inputenc}
% Поддержка необходимых языков
\usepackage[english,russian]{babel}

% Поддержка изображений
\usepackage{graphicx}
% Путь до внешних изображений
\graphicspath{ {./figures/}}

% Умная запятая
\usepackage{icomma}

% Ссылки на электронные ресурсы
\usepackage{hyperref}
% Настройка внешнего вида ссылок
\hypersetup{
  % Отключить прямоугольную рамку
  pdfborder={0 0 0},
  % Включить цветные ссылки
  colorlinks=true,
  % Цвет для ссылок на веб-ресурсы
  urlcolor=blue,
  % Цвет внутренних ссылок
  linkcolor=black
}

% Дополнительная математика
\usepackage{amsmath,amsfonts,amssymb,amsthm,mathtools}
% Показывать номера только у тех выржений, на которые кто-то ссылается
\mathtoolsset{showonlyrefs=true}

% Дополнительные символы
\usepackage{mathbbol}

% Подключние пакетов для импорта других .tex
\usepackage[subpreambles=true]{standalone}
\usepackage{import}

% Правильное оформление
% Настройка отступов
\usepackage[left=2cm,right=1cm,top=2cm,bottom=2cm]{geometry}
% Настройка шрифта
\usepackage{fontspec}
\setmainfont{Times New Roman}
% Настройка межстрочных интервалов
\usepackage{setspace}
\onehalfspacing
% и межабзацных
\usepackage{parskip}
\setlength{\parindent}{1.25cm} 

% Красная строка
\usepackage{indentfirst}
\setlength{\parindent}{1.25cm} 

% Корректное положение рисунков?
\usepackage{float}

% Расположение блоков относительно текста
\usepackage{wrapfig}

\begin{document}

\subsection{Более сложные примеры индукции}

\subsubsection{Отдаленный пример}

\textbf{Теорема:} если $\mathbb{E} \subseteq \mathbb{N}$, и
$0 \in \mathbb{E}$, и $\forall x \in \mathbb{E}: x + 1 \in \mathbb{E}$
то $\mathbb{E} = \mathbb{N}$

\begin{enumerate}
  \item {
    $0 \in \mathbb{E}$ и $\forall x \in \mathbb{E}: x + 1 \in \mathbb{E}$
    $\Rightarrow \mathbb{E}$ - индуктивное множество 
  }
  \item {
    $\mathbb{E} \subseteq \mathbb{N}$, однако $\mathbb{N}$ - наименьшее
    индуктивное множество
  }
  \item {
    Исходя из пуктов 1 и 2 $\Rightarrow \mathbb{E} = \mathbb{N}$, т.к.
    $\mathbb{N} \subseteq \forall \mathbb{X}$, если $\mathbb{X}$ - 
    индуктивное множество, а $\mathbb{E}$ как раз таковым является
  }
\end{enumerate}

\subsubsection{Строгий пример}

Пусть дана последовательность
высказываний $\left\{ \Phi_n; n\in \mathbb{N}  \right\}$, тогда

\begin{enumerate}
  \item {
    Вручную проверяем, что $\Phi_0$ верно
  }
  \item {
    Проверяем, что $\Phi_{n+1}$ верно при верном $\Phi_n$
  }
  \item {
    Тогда $\Phi_n$ верно $\forall n$
  }
\end{enumerate}

\subsubsection{Ещё пример}

Рассмотрим $\mathbb{E} = \left\{ n \in \mathbb{N}: \Phi_n \text{ - верно} \right\}$

\begin{enumerate}
  \item {
    $0 \in \mathbb{E}$, т.к. $\Phi_0$ верно
  }
  \item {
    Если $n \in \mathbb{E} \Rightarrow \Phi_n$ верно
    $\Rightarrow \Phi_{n + 1}$ верно $\Rightarrow$
    $n + 1 \in \mathbb{E}$
  }
  \item {
    Исходя из пуктов 1 и 2 $\mathbb{E}$ индуктивно и равно $\mathbb{N}$
  }
\end{enumerate}

\subsection{Определение по индукции}

Пусть $\mathbb{X}$ - какое-либо множество, а $\mathbb{G}$ - такое
правило, которое каждой конечной последовательность элементов из
$\mathbb{X} - \left\{ x_0, x_1, \dots, x_n \in \mathbb{X}\right\}$
ставит в соответствие эелемент $\mathbb{G}\left(x_0, x_1, \dots, x_n\right) \in \mathbb{N}$

Тогда $\forall x \in \mathbb{X}$ правило $\mathbb{G}$ однозначно
определяет бесконечную последовательность $x_0, x_1, \dots, x_n$
с таким свойством: $\forall n: x_{n + 1} = \mathbb{G}\left(x_0, x_1, \dots, x_n\right)$

\subsubsection{Определение степени числа}

\textbf{Определение:} 
$a^n = \underbrace{a \times a \times a \times \dots \times a}_{n \text{ множителей}}
= a^{n - 1}\times a$ ($a \in \mathbb{R}, n \in \mathbb{N}$)

Рассмотрим правило $\mathbb{G}(x_0, x_1, \dots, x_n)=ax_n$ при $x_0 = 1$

Тогда для любой последовательности $\{x_n\}$ со свойствами:

\begin{equation}
  \begin{cases}
    x_{n + 1} & = \mathbb{G}(x_0, x_1, \dots, x_n) = ax_n \\
    x_0 & = 1
  \end{cases}
\end{equation}

верно, что $x_n = a^n$, а $\{x_n\}$ - последовательность степеней числа $a$

\subsubsection{Определение факториала}

\textbf{Определение:} $n! = 1 \times 2 \times \dots \times n$

Рассмотрим правило $G(x_0, x_1, \dots, x_n) = (n + 1)\times x_n$ ($x_0 = 1$)

Тогда для любой последовательности $\{x_n\}$ со свойствами:

\begin{equation}
  \begin{cases}
    x_{n + 1} & = \mathbb{G}(x_0, x_1, \dots, x_n) = x_n\times(n + 1) \\
    x_0 & = 1
  \end{cases}
\end{equation}

верно, что $x_n = n!$, а $\{x_n\}$ - последовательность факториалов натуральных чисел

\subsubsection{Определение индуктивной суммы}

Пусть дана бесконечная последовательность $\{a_n\}$ (
$\forall n \in \mathbb{N} \rightarrow  a_n \in \mathbb{R}$)

Рассмотрим правило $\mathbb{G}(x_0, x_1, \dots, x_n) = x_n + a_{n + 1}$
($x_0 = a_0$). Тогда существует единственная последовательность $\{x_n\}$,
такая что $x_0 = 0$, $x_n = \sum\limits_{k=0}^n{a_k}$ и $x_{n + 1} = x_n + a_{n + 1}$

\subsubsection{Определение индуктивного произведения}

Пусть дана бесконечная последовательность $\{b_n\}$
($\forall n \in \mathbb{N} \rightarrow b_n \in \mathbb{R}$)

Рассмотрим правило $G(x_0, x_1, \dots, x_n) = x_n\times b_{n+1}$
($x_0 = b_0$). Тогда существует единственная последовательности $\{x_n\}$,
такая что $x_0 = b_0$, $x_n = \prod\limits_{k=0}^n{b_k}$ и $x_{n + 1} = x_n \times b_{n + 1}$

\subsection{Доказательство формул по индукции}

\subsubsection{Неравенство Бернулли}

\textbf{Само неравенство:} $(1 + \alpha)^n \geq 1 + n\alpha$ ($a > -1, n \in \mathbb{N}$)

\begin{enumerate}
  \item {
    Для $\alpha = 0$ или $n = 0$: $1 \geq 1$ - верно
  }
  \item {
    Допустим выражение верно для какого-то числа $n$
  }
  \item {
    Докажем, что выражение верно для $n + 1$:
    \begin{equation}
      %\begin{multline}
        (1 + \alpha)^{(n + 1)} \geq 1 + (n + 1)\alpha% \; \Rightarrow \\
        \Rightarrow \;
        \underbrace{(1 + \alpha)^n}_{\geq 1 + n\alpha}\underbrace{(1 + \alpha)}_{> 0} \geq 1 + n\alpha + \alpha %\; \Rightarrow %\\
        %\Rightarrow \;
      %\end{multline}
    \end{equation}

    Тогда $(1 + \alpha)^{n + 1} \geq (1 + n\alpha)(1 + \alpha) =
    1 + n\alpha + \alpha + \underbrace{n\alpha^2}_{\geq 0}$.

    Т.к. $1 + n\alpha + \alpha + n\alpha^2 \geq 1 + n\alpha + \alpha$, то
    и $(1 + \alpha)^{n + 1} \geq 1 + n\alpha + \alpha$ ($\blacksquare$)
  }
\end{enumerate}

\subsubsection{Ещё какая-то формула}

\textbf{Доказать:} $\sum\limits_{i = 0}^n{q^i} = \frac{1 - q^{n + 1}}{1 - q}$ при $q \neq 1$

\begin{enumerate}
  \item {
    Рассмотрим случай $n = 0$, тогда:
    \begin{equation}
      \sum\limits_{i = 0}^0{q^i} = q^0 = 1 = \frac{1 - q^{0 + 1}}{1 - q} = 1 \text{ (верно)}
    \end{equation}
  }
  \item {
    Допустим, выражение верно для какого-то $k$:
    \begin{equation}
      \sum\limits_{i = 0}^k{q^i} = \frac{1 - q^{k + 1}}{1 - q}
    \end{equation}
  }
  \item {
    Проверим выражение при $n = k + 1$:
    \begin{multline}
      \sum\limits_{i = 0}^{k + 1}{q^i} = \sum\limits_{i = 0}^{k}{q^i} + q^{k + 1} = \\ =
      \frac{1 - q^{k + 1}}{1 - q} + q^{k + 1} 
      = \frac{1 - q^{k + 1} + q^{k + 1}(1 - q)}{1 - q} = \\ =
      \frac{1 - q^{k + 1} +q^{k + 1} - q^{k + 1  + 1}}{1 - q} 
      = \frac{1 - q^{(k + 1) + 1}}{1 - q} \:\: (\blacksquare)
    \end{multline}
  }
\end{enumerate}

\end{document}
