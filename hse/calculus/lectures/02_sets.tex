% Лекция по Математическому Анализу 07.09.2022
% Михедов Константин Константиновчи, БИБ 224

% Тип документа: статья, размер бумаги - A4, написано 14 кегелем
% Предназначено для импорта из другого документа
\documentclass[class=article,a4paper,12pt,crop=false]{standalone}

% Поиск по скомпилированному PDF
\usepackage{cmap}
% Кодировка выходного текста
\usepackage[T2A]{fontenc}
% Кодировка исходного текста
\usepackage[utf8]{inputenc}
% Поддержка необходимых языков
\usepackage[english,russian]{babel}

% Поддержка изображений
\usepackage{graphicx}
% Путь до внешних изображений
\graphicspath{ {./figures/}}

% Умная запятая
\usepackage{icomma}

% Ссылки на электронные ресурсы
\usepackage{hyperref}
% Настройка внешнего вида ссылок
\hypersetup{
  % Отключить прямоугольную рамку
  pdfborder={0 0 0},
  % Включить цветные ссылки
  colorlinks=true,
  % Цвет для ссылок на веб-ресурсы
  urlcolor=blue,
  % Цвет внутренних ссылок
  linkcolor=black
}

% Дополнительная математика
\usepackage{amsmath,amsfonts,amssymb,amsthm,mathtools}
% Показывать номера только у тех выржений, на которые кто-то ссылается
\mathtoolsset{showonlyrefs=true}

% Дополнительные символы
\usepackage{mathbbol}

% Подключние пакетов для импорта других .tex
\usepackage[subpreambles=true]{standalone}
\usepackage{import}

% Правильное оформление
% Настройка отступов
\usepackage[left=2cm,right=1cm,top=2cm,bottom=2cm]{geometry}
% Настройка шрифта
\usepackage{fontspec}
\setmainfont{Times New Roman}
% Настройка межстрочных интервалов
\usepackage{setspace}
\onehalfspacing
% и межабзацных
\usepackage{parskip}
\setlength{\parindent}{1.25cm} 

% Красная строка
\usepackage{indentfirst}
\setlength{\parindent}{1.25cm} 

% Корректное положение рисунков?
\usepackage{float}

% Расположение блоков относительно текста
\usepackage{wrapfig}

\begin{document}

Любое числовое числовое множество является подмножеством в $\mathbb{R}$

Например, $\left\{1; \frac{3}{2}; -3\right\}$ - конечное (обычное) множество

\subsection{Виды множеств}

\begin{itemize}
  \item {
    Отрезок \begin{equation}
      \forall a,b \in \mathbb{R}: a \leq b
      \left\{x \in \mathbb{R}: a \leq x \leq b \right\} =
      \left[a; b\right]
    \end{equation}
  } \item {
    Интервал \begin{equation}
      \forall a,b \in \mathbb{R}: a < b
      \left\{x \in \mathbb{R}: a < x < b \right\} =
      \left(a; b\right)
    \end{equation}
  } \item {
    Полуинтервалы \begin{equation}
      \forall a,b \in \mathbb{R}: a < b
      \left\{x \in \mathbb{R}: a \leq x < b \right\} =
      \left[a; b\right)
    \end{equation}
    \begin{equation}
      \forall a,b \in \mathbb{R}: a < b
      \left\{x \in \mathbb{R}: a < x \leq b \right\} =
      \left(a; b\right]
    \end{equation}
  } \item {
    Бесконечные полуинтервалы \[ \begin{aligned}
      \forall a \in \mathbb{R}: \left\{x \in \mathbb{R}: a < x\right\} &= \left(a; +\infty\right) \\
      \forall a \in \mathbb{R}: \left\{x \in \mathbb{R}: a \leq x\right\} &= \left[a; +\infty\right) \\
      \forall a \in \mathbb{R}: \left\{x \in \mathbb{R}: x < a\right\} &= \left(-\infty; a\right) \\
      \forall a \in \mathbb{R}: \left\{x \in \mathbb{R}: x \leq a\right\} &= \left(-\infty; a\right] 
    \end{aligned} \]
  }
\end{itemize}

\subsection{Операции над множествами}

Пусть $\mathbb{X}$ и $\mathbb{Y}$ такие, что
$\mathbb{X} \subseteq \mathbb{R}$ и $\mathbb{Y} \subseteq \mathbb{R}$
    
\begin{itemize}
  \item {
    $
    -\mathbb{X} = \left\{-x: x \in \mathbb{X} \right\} =
    \left\{z \in \mathbb{R}: \forall x \in \mathbb{X}: z = -x\right\}
    $
  }
  \item {
    $
    \mathbb{X} + \mathbb{Y} = \left\{ z \in \mathbb{R}: \exists
    x \in \mathbb{X} \text{ и } \exists y \in \mathbb{Y}: z = x + y\right\}
    $
  }
  \item {
    Операции $\times, \div, -$ описываются аналогично сложению
  }
  \item {
    $
    \mathbb{X} \leq \mathbb{Y} \Leftrightarrow \forall x \in \mathbb{X}
    \text{ и } \forall y \in \mathbb{Y}: x \leq y
    $
  } \item {
    Операции $\geq, <, >$ описываются аналогично $\leq$
  }
\end{itemize}

\subsubsection{Примеры}

%\begin{equation}
  \begin{align*}
    [0; 1] + [-1; 1] & = [-1; 2] & [0; 1] + (-1; 1) & = (-1; 2) \\
    [0; 1] & \nleq [-1; 1] & [-1; 1] & \nleq (0; 1) \\
    [0; 1] & \leq [1; 2] & (0; 1) & < [1; 2]
  \end{align*}
%\end{equation}

\subsection{Максимум и минимум множества}

\begin{equation}
  \text{Пусть } \mathbb{X} \subseteq \mathbb{R} \text{ тогда } \min{\mathbb{X}}
  = a, \text{ если } \forall x \in \mathbb{X}: a \leq x
\end{equation}
\begin{equation}
  \text{Пусть } \mathbb{Y} \subseteq \mathbb{R} \text{ тогда } \max{\mathbb{Y}}
  = b, \text{ если } \forall y \in \mathbb{Y}: b \geq y
\end{equation}

\subsection{Точная верхняя и нижняя грани}

\textbf{Определение:} множество $\mathbb{X}$ называется
ограниченным снизу, в том случае, если $\exists s \in \mathbb{R}: s \leq \mathbb{X}$,
то есть $s \leq \forall x \in \mathbb{X}$

\begin{picture}(360,60)
  \put(120,25){\vector(1,0){160}}

  \put(150,25){\circle*{4}}
  \put(147.5,15){\text{s}}

  \qbezier(180,25)(180,35)(190,35)
  \put(190,35){\line(1,0){85}}

  \put(182,25){\line(1,1){10}}
  \put(200,25){\line(1,1){10}}
  \put(220,25){\line(1,1){10}}
  \put(240,25){\line(1,1){10}}
  \put(260,25){\line(1,1){10}}

  \put(270,12.5){$\mathbb{R}$}
  \put(225, 40){$\mathbb{X}$}
\end{picture}

\textbf{Определение:} множество $\mathbb{Y}$ называется
ограниченным сверху, в том случае, если $\exists p \in \mathbb{R}: p \geq \mathbb{Y}$,
то есть $p \geq \forall y \in \mathbb{Y}$

\begin{picture}(360, 60)
  \put(120,25){\vector(1,0){160}}

  \put(250,25){\circle*{4}}
  \put(247.5,15){\text{p}}

  \qbezier(220,25)(220,35)(210,35)
  \put(210,35){\line(-1,0){90}}

  \put(218,25){\line(-1,1){10}}
  \put(200,25){\line(-1,1){10}}
  \put(180,25){\line(-1,1){10}}
  \put(160,25){\line(-1,1){10}}
  \put(140,25){\line(-1,1){10}}

  \put(270,12.5){$\mathbb{R}$}
  \put(160, 40){$\mathbb{Y}$}
\end{picture}

\textbf{Определение:} точная нижняя грань множества $\mathbb{X} \subseteq \mathbb{R}$:
\begin{equation}
  \inf{\mathbb{X}} = 
  \begin{cases}
    -\infty: \text{ если множество } \mathbb{X} \text{ не ограничено снизу} \\
    \min{\mathbb{X}: \text{ в другом случае}}
  \end{cases}
\end{equation}

\textbf{Определение:} точная верхняя грань множества $\mathbb{Y} \subseteq \mathbb{Y}$
\begin{equation}
  \sup{\mathbb{Y}} =
  \begin{cases}
    \infty: \text{ если множество } \mathbb{Y} \text{ не ограничено сверху} \\
    \max{\mathbb{Y}}: \text{ в другом случае}
  \end{cases}
\end{equation}

\textbf{Примечание:} $\sup$ - supremum, $\inf$ - infinum

\subsubsection{Теоремы о точных гранях}

\begin{itemize}
  \item {
    Если $\exists \min{\mathbb{X}}$, то $\inf{\mathbb{X}} = \min{\mathbb{X}}$
  }
  \item {
    Если $\exists \max{\mathbb{Y}}$, то $\sup{\mathbb{Y}} = \max{\mathbb{Y}}$
  }
  \item {
    Если $\mathbb{Y}$ ограничено сверху, то $\sup{\mathbb{Y}} \in \mathbb{R}$ (является числом)
  }
  \item {
    Если $\mathbb{X}$ ограничено снизу, то $\inf{\mathbb{X}} \in \mathbb{R}$ (является числом)
  
    \textbf{Доказательство:} пусть $\mathbb{X} \subseteq \mathbb{R}$ ограничено снизу
    
    Рассмотрим множество $\mathbb{S}$ всех таких $s \in \mathbb{R}$,
    которые ограничивают $\mathbb{X}$ слева:
    $\mathbb{S} = \left\{s \in \mathbb{R}: s \leq \mathbb{X}\right\}$.
    Из определения $\mathbb{S}$ видно, что $\mathbb{S} \leq \mathbb{X}$,
    то есть $\forall x \in \mathbb{X} \land \forall s \in \mathbb{S}: s \leq x$
    $\Rightarrow \exists c \in \mathbb{R}: \mathbb{S} \leq c \leq \mathbb{R}$
  
    Тогда $c$ является одной из тех $s$ для которых $s \leq \mathbb{X}$
    $\Rightarrow c \in \mathbb{S}$

    Т.к. $\mathbb{S} \leq c$ и $c \in \mathbb{S}$
    $\Rightarrow c = \max{\mathbb{S}} = \inf{\mathbb{X}} = c \in \mathbb{R}$
    $\; (\blacksquare)$
  }
\end{itemize}

\subsubsection{Свойства граней}

\begin{enumerate}
  \item {
    Монотонность
    \begin{equation}
      \begin{aligned}
        \mathbb{X} \subseteq \mathbb{Y} & \Rightarrow \inf{\mathbb{X}} \geq \inf{\mathbb{Y}} \\
        \mathbb{X} \subseteq \mathbb{Y} & \Rightarrow \sup{\mathbb{X}} \leq \sup{\mathbb{Y}}
      \end{aligned}
    \end{equation}
  }
  \item {
    Инвариативность относительно сдвига
    \begin{equation}
      \begin{aligned}
        \forall \mathbb{X} \subseteq \mathbb{R} \text{ и } \forall a \in \mathbb{R}: \inf{(\mathbb{X} + a)} & = \inf{\mathbb{X}} + a\\
         \sup{(\mathbb{X} + b)} & = \sup{\mathbb{X}} + b
      \end{aligned}
    \end{equation}
  }
  \item {
    Двойственность
    \begin{equation}
      \begin{aligned}
        \inf{(-\mathbb{X})} & = -\sup{\mathbb{X}} \\
        \sup{(-\mathbb{Y})} & = -\inf{\mathbb{Y}}
      \end{aligned}
    \end{equation}
  }
\end{enumerate}

\subsection{Неточные грани}

\paragraph{Верхней гранью} (мажорантой) числвого множетсва $\mathbb{X}$ называют такое число 
$a$, что $\forall x \in \mathbb{X}$ верно $x \leq a$

\paragraph{Нижней гранью} (минорантой) числового множества $\mathbb{Y}$ называют такое число
$b$, что $\forall y \in \mathbb{Y}$ верно $b \leq y$

\subsubsection{Связь с точными гранями}

Точной верхней/нижней гранью называют $\min$/$\max$ множества всех неточных верхних/нижних граней

\end{document}
