% Лекция по Математическому Анализу 03.10.2022
% Михедов Константин Константиновчи, БИБ 224

% Тип документа: статья, размер бумаги - A4, написано 14 кегелем
% Предназначено для импорта из другого документа
\documentclass[class=article,a4paper,12pt,crop=false]{standalone}

% Поиск по скомпилированному PDF
\usepackage{cmap}
% Кодировка выходного текста
\usepackage[T2A]{fontenc}
% Кодировка исходного текста
\usepackage[utf8]{inputenc}
% Поддержка необходимых языков
\usepackage[english,russian]{babel}

% Поддержка изображений
\usepackage{graphicx}
% Путь до внешних изображений
\graphicspath{ {./figures/} {./../figures/}}

% .esp support
\usepackage{epstopdf}

% Умная запятая
\usepackage{icomma}

% Ссылки на электронные ресурсы
\usepackage{hyperref}
% Настройка внешнего вида ссылок
\hypersetup{
  % Отключить прямоугольную рамку
  pdfborder={0 0 0},
  % Включить цветные ссылки
  colorlinks=true,
  % Цвет для ссылок на веб-ресурсы
  urlcolor=blue,
  % Цвет внутренних ссылок
  linkcolor=black
}

% Дополнительная математика
\usepackage{amsmath,amsfonts,amssymb,amsthm,mathtools}
% Показывать номера только у тех выржений, на которые кто-то ссылается
\mathtoolsset{showonlyrefs=true}

% Дополнительные символы
\usepackage{mathbbol}

% Подключние пакетов для импорта других .tex
\usepackage[subpreambles=true]{standalone}
\usepackage{import}

% Функция card
\DeclareMathOperator{\card}{card}

% Правильное оформление
% Настройка отступов
\usepackage[left=2cm,right=1cm,top=2cm,bottom=2cm]{geometry}
% Настройка шрифта
\usepackage{fontspec}
\setmainfont{Times New Roman}
% Настройка межстрочных интервалов
\usepackage{setspace}
\onehalfspacing
% и межабзацных
\usepackage{parskip}
\setlength{\parindent}{1.25cm} 

% Красная строка
\usepackage{indentfirst}
\setlength{\parindent}{1.25cm} 

% Корректное положение рисунков?
\usepackage{float}

% Расположение блоков относительно текста
\usepackage{wrapfig}

% Теперь мы еще и графики рисуем
\usepackage{pgfplots}
\usetikzlibrary{
  calc,
  pgfplots.groupplots,
}
\pgfplotsset{compat=1.16}

% Списки в два и более стобцов
\usepackage{multicol}

\begin{document}

\subsection{Последовательности}

\textbf{Последовательностью} называют функцию, определенную на множестве натуральных
чисел, то есть $f: \mathbb{N} \rightarrow \mathbb{R}$ (в последовательностях можно
делать вот так: $f(n) = f_n$ )

\subsubsection{Почти все...}

Пусть $P$ - какое-то свойство $\mathbb{N}$

Говорят, что \textbf{почти все} числа $n \in \mathbb{N}$ обладают свойством $P$, если
только конечный набор чисел $n \in \mathbb{N}$ этим свойством не обладает:
\begin{equation}
  \card \{n \in \mathbb{N}: \lnot P\} < \infty
\end{equation}

где $\card$ - мощность множества (количество элементов, от cardinality)

\subsubsection{Теорема Архимеда}

Для любого числа $c \in \mathbb{R}$ условие $n > c$
верно для почти всех чисел $n \in \mathbb{N}$

\paragraph{Доказательство} Зафиксируем $c \in \mathbb{R}$. По принципу Архимеда 
$\exists N \in \mathbb{N}$, что $N > c$, тогда $\forall n \in \mathbb{N}: n \geq N$ верно:
$n \geq N > c \Rightarrow n > c$

Данное сравнение неверно только для первых $N$ чисел, то есть неверно только для
$n = \{0, 1, 2, \dots, N - 1\}$, то есть неверно только для конечного числа
элементов

\paragraph{Пример} Пусть $\{x_n\} = \{\frac{1}{n}\}$ где $n \in \mathbb{N}$, верно ли,
что почти все элементы данной последовательности $\in \mathbb{M}$, где $\mathbb{M} = (-0.1; 0.1)$

\begin{equation}
  x_n \in \mathbb{M} \Leftrightarrow \frac{1}{n} \in (-0.1; 0.1) \Leftrightarrow
  \begin{cases}
    \frac{1}{n} &> -0.1 \\
    \frac{1}{n} &< 0.1
  \end{cases}
  \Leftrightarrow n > 10
\end{equation}

Тогда по теореме Архимеда $n > 10$ верно почти для всех $n$ $\Rightarrow$ и исходное
выражение верно для почти всех $n$ $(\blacksquare)$

\subsubsection{Монотонность}

Последовательность $\{x_n\}$ называют
\begin{itemize}
  \item {
    возрастающей, если $\forall n \in \mathbb{N}:$ $x_n < x_{n + 1}$ 
  }
  \item {
    убывающей, если $\forall n \in \mathbb{N}:$ $x_n > x_{n + 1}$ 
  }
  \item {
    невозрастающей, если $\forall n \in \mathbb{N}:$ $x_n \geq x_{n + 1}$ 
  }
  \item {
    неубывающей, если $\forall n \in \mathbb{N}:$ $x_n \leq x_{n + 1}$ 
  }
\end{itemize}

Последовательность называют монотонной, если она неубывающая или невозрастающая

\end{document}
