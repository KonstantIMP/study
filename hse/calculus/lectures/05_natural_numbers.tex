% Лекция по Математическому Анализу 12.09.2022
% Михедов Константин Константиновчи, БИБ 224

% Тип документа: статья, размер бумаги - A4, написано 14 кегелем
% Предназначено для импорта из другого документа
\documentclass[class=article,a4paper,12pt,crop=false]{standalone}

% Поиск по скомпилированному PDF
\usepackage{cmap}
% Кодировка выходного текста
\usepackage[T2A]{fontenc}
% Кодировка исходного текста
\usepackage[utf8]{inputenc}
% Поддержка необходимых языков
\usepackage[english,russian]{babel}

% Поддержка изображений
\usepackage{graphicx}
% Путь до внешних изображений
\graphicspath{ {./figures/}}

% Умная запятая
\usepackage{icomma}

% Ссылки на электронные ресурсы
\usepackage{hyperref}
% Настройка внешнего вида ссылок
\hypersetup{
  % Отключить прямоугольную рамку
  pdfborder={0 0 0},
  % Включить цветные ссылки
  colorlinks=true,
  % Цвет для ссылок на веб-ресурсы
  urlcolor=blue,
  % Цвет внутренних ссылок
  linkcolor=black
}

% Дополнительная математика
\usepackage{amsmath,amsfonts,amssymb,amsthm,mathtools}
% Показывать номера только у тех выржений, на которые кто-то ссылается
\mathtoolsset{showonlyrefs=true}

% Дополнительные символы
\usepackage{mathbbol}

% Подключние пакетов для импорта других .tex
\usepackage[subpreambles=true]{standalone}
\usepackage{import}

\begin{document}

\subsection{Лемма о неограниченности}

\textbf{Лемма:} множество $\mathbb{N}$ не ограниченно сверху

Предположим противное: пусть $\mathbb{N}$ ограниченно сверху $\Rightarrow$
$\exists$ такое действительное число $B$, которое лежит правее $\mathbb{N}$ ($\forall x \in \mathbb{N}: x < B$):

\begin{picture}(300, 50)
  
  \put(75, 20){\vector(1, 0){200}}

  \put(75, 30){\line(1, 0){100}}
  \qbezier(175, 30)(185, 30)(185, 20)

  \put(75, 30){\line(1, -1){10}}
  \put(90, 30){\line(1, -1){10}}
  \put(105, 30){\line(1, -1){10}}
  \put(120, 30){\line(1, -1){10}}
  \put(135, 30){\line(1, -1){10}}
  \put(150, 30){\line(1, -1){10}}
  \put(165, 30){\line(1, -1){10}}
  
  \put(125, 5){$\mathbb{N}$}

  \put(225, 20){\circle*{5}}
  \put(222.5, 5){B}

\end{picture}

Тогда по теореме о точной верхней грани множество $\mathbb{N}$
обладает точной верхней гранью: $\exists \sup{\mathbb{N}} = C \in \mathbb{R}$ ($C \leq B$),
тогда $C - 1$ является неточной верхней гранью (не ограничивает $\mathbb{N}$ справа)
$\Rightarrow \exists n \in \mathbb{N}: n > C - 1$ $\Rightarrow$
т.к. $\mathbb{N}$ - индуктивное $n + 1 \in \mathbb{N}$. $n > C - 1$
$\Rightarrow n + 1 > (C - 1) + 1 \Rightarrow n + 1 > C \Rightarrow$
$C \neq \sup{\mathbb{N}} \Rightarrow$ множество $\mathbb{N}$ не ограниченно сверху
$(\blacksquare)$

\subsection{Принцип Архимеда}

\textbf{Формулировка:} $\forall C \in \mathbb{R}$ $\exists n \in \mathbb{N}: C < n$

Зафиксируем какое-то $C \in \mathbb{R}$ и предположим, что
$\nexists n \in \mathbb{N}$ для которого $C < n$ $\Rightarrow$
$\forall n \in \mathbb{N}: n \leq C$ $\Rightarrow$ множество
$\mathbb{N}$ ограниченно сверху числом $C$, но по лемме о неограниченности
такого быть не может $\Rightarrow$ принцип Архимеда верен $(\blacksquare)$

\end{document}
