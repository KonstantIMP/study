% Лекция по Математическому Анализу 16.11.2022
% Михедов Константин Константиновчи, БИБ 224

% Тип документа: статья, размер бумаги - A4, написано 14 кегелем
% Предназначено для импорта из другого документа
\documentclass[class=article,a4paper,12pt,crop=false]{standalone}

% Поиск по скомпилированному PDF
\usepackage{cmap}
% Кодировка выходного текста
\usepackage[T2A]{fontenc}
% Кодировка исходного текста
\usepackage[utf8]{inputenc}
% Поддержка необходимых языков
\usepackage[english,russian]{babel}

% Поддержка изображений
\usepackage{graphicx}
% Путь до внешних изображений
\graphicspath{ {./figures/} {./../figures/}}

% .esp support
\usepackage{epstopdf}

% Умная запятая
\usepackage{icomma}

% Ссылки на электронные ресурсы
\usepackage{hyperref}
% Настройка внешнего вида ссылок
\hypersetup{
  % Отключить прямоугольную рамку
  pdfborder={0 0 0},
  % Включить цветные ссылки
  colorlinks=true,
  % Цвет для ссылок на веб-ресурсы
  urlcolor=blue,
  % Цвет внутренних ссылок
  linkcolor=black
}

% Дополнительная математика
\usepackage{amsmath,amsfonts,amssymb,amsthm,mathtools}
% Показывать номера только у тех выржений, на которые кто-то ссылается
\mathtoolsset{showonlyrefs=true}

% Дополнительные символы
\usepackage{mathbbol}

% Подключние пакетов для импорта других .tex
\usepackage[subpreambles=true]{standalone}
\usepackage{import}

% Правильное оформление
% Настройка отступов
\usepackage[left=2cm,right=1cm,top=2cm,bottom=2cm]{geometry}
% Настройка шрифта
\usepackage{fontspec}
\setmainfont{Times New Roman}
% Настройка межстрочных интервалов
\usepackage{setspace}
\onehalfspacing
% и межабзацных
\usepackage{parskip}
\setlength{\parindent}{1.25cm} 

% Красная строка
\usepackage{indentfirst}
\setlength{\parindent}{1.25cm} 

% Корректное положение рисунков?
\usepackage{float}

% Расположение блоков относительно текста
\usepackage{wrapfig}

% Функция card
\DeclareMathOperator{\card}{card}

% зачеркивание
\usepackage{cancel}

% Теперь мы еще и графики рисуем
\usepackage{pgfplots}
\usetikzlibrary{
  calc,
  pgfplots.groupplots,
}
\pgfplotsset{compat=1.16}

% Списки в два и более стобцов
\usepackage{multicol}

\begin{document}

Функция $f(x)$ имеет предел $A$ в точке $a \in \mathbb{R}$, если 
$f(x)$ определена в некой $\dot{v_{\varepsilon}(a)} = v_{\varepsilon}(a) \setminus \{a\} =
(a - \varepsilon; a) \cup (a; a + \varepsilon)$ и $\forall x_n \in v_{\varepsilon}(a)$:
$x_n \underset{n \rightarrow \infty}{\rightarrow} a$ $\Rightarrow$
$f(x_n) \underset{n \rightarrow \infty}{\rightarrow} A$

\subsection{Пределы справа и слева}

Предел справа: $\lim\limits_{x \rightarrow a + 0}f(x) = A$ $\Leftrightarrow$
$\forall x_n \in v_{\varepsilon}(a)$ $x_n > a$, $x_n \underset{n \rightarrow \infty}{\rightarrow} a$
$\Rightarrow$ $f(x_n) \underset{n \rightarrow \infty}{\rightarrow} A$

Предел слева: $\lim\limits_{x \rightarrow a - 0}f(x) = A$, определение аналогично

\subsection{Связь предела и непрерывности}

Пусть $f(x)$ определена в окрестности точки $a$ и непрерывна в $a$, тогда
$f(a) = \lim\limits_{n \rightarrow a}f(x)$

\paragraph{Доказательство} $\forall x_n \in \dot{v_{\varepsilon}(a)}$
$x_n \underset{n \rightarrow \infty}{\rightarrow} a$ $\Rightarrow$
$f(x_n) \underset{n \rightarrow \infty}{\rightarrow} f(a)$ $\Rightarrow$
$\lim\limits_{x \rightarrow a}f(x) = f(a)$

\subsection{Теорема о замене переменных}

$g(y) = g(f(x))$ при $y = f(x)$

Пусть $f$ и $g$ - две функции, причем $\lim\limits_{n \rightarrow a}f(x) = b$, и
$\lim\limits_{y \rightarrow b}g(y) = c$, и $\forall x \in \dot{v_{\varepsilon}(a)}$
$f(x) \neq b$, тогда $\lim\limits_{x \rightarrow a}g(f(x)) = \lim\limits_{y \rightarrow b}g(y) = c$

\paragraph{Доказательство} Пусть $x_n \in \dot{v_{\varepsilon}(a)}$, тогда $x_n \underset{n \rightarrow \infty}{\rightarrow}
a$ и $f(x_n) \underset{n \rightarrow \infty}{\rightarrow} b$. С другой стороны
$f(x_n) \neq b$ $\Rightarrow$ $f(x_n) \in v_{\varepsilon}(b)$ и 
$f(x_n) \underset{n \rightarrow \infty}{\rightarrow} b$
$\Rightarrow$ $g(f(x_n)) \underset{n \rightarrow \infty}{\rightarrow} c$

\subsection{Арифметические операции над пределами функций}

Пусть $f$ и $g$ имеют пределы в точке $a$, тогда
\begin{enumerate}
  \item {
    $\lim\limits_{x \rightarrow a}(f(x)g(x)) = \lim\limits_{x \rightarrow a}f(x)\lim\limits_{x \rightarrow a}g(x)$
  }
  \item {
    $\lim\limits_{x \rightarrow a}(f(x)\pm g(x)) = \lim\limits_{x \rightarrow a}f(x)\pm\lim\limits_{x \rightarrow a}g(x)$
  }
  \item {
    Если $\lim\limits_{x \rightarrow a}g(x) \neq 0$, то $\lim\limits_{x \rightarrow a}(\frac{f(x)}{g(x)}) = \frac{\lim\limits_{x \rightarrow a}f(x)}{\lim\limits_{x \rightarrow a}g(x)}$
  }
\end{enumerate}

\end{document}
