% Лекция по Математическому Анализу 12.09.2022
% Михедов Константин Константиновчи, БИБ 224

% Тип документа: статья, размер бумаги - A4, написано 14 кегелем
% Предназначено для импорта из другого документа
\documentclass[class=article,a4paper,12pt,crop=false]{standalone}

% Поиск по скомпилированному PDF
\usepackage{cmap}
% Кодировка выходного текста
\usepackage[T2A]{fontenc}
% Кодировка исходного текста
\usepackage[utf8]{inputenc}
% Поддержка необходимых языков
\usepackage[english,russian]{babel}

% Поддержка изображений
\usepackage{graphicx}
% Путь до внешних изображений
\graphicspath{ {./figures/}}

% Умная запятая
\usepackage{icomma}

% Ссылки на электронные ресурсы
\usepackage{hyperref}
% Настройка внешнего вида ссылок
\hypersetup{
  % Отключить прямоугольную рамку
  pdfborder={0 0 0},
  % Включить цветные ссылки
  colorlinks=true,
  % Цвет для ссылок на веб-ресурсы
  urlcolor=blue,
  % Цвет внутренних ссылок
  linkcolor=black
}

% Дополнительная математика
\usepackage{amsmath,amsfonts,amssymb,amsthm,mathtools}
% Показывать номера только у тех выржений, на которые кто-то ссылается
\mathtoolsset{showonlyrefs=true}

% Дополнительные символы
\usepackage{mathbbol}

% Подключние пакетов для импорта других .tex
\usepackage[subpreambles=true]{standalone}
\usepackage{import}

\begin{document}

\subsection{Неформальные определения}

Французское определение: $\mathbb{N} = \left\{0; 1; 2; \dots\right\}$

Немецкое определение: $\mathbb{N}^* = \left\{1; 2; 3; \dots\right\}$

\subsection{Индуктивные множества}

Множество $\mathbb{X}$ называется индуктивным, если
$0 \in \mathbb{X}$ и $\forall x \in \mathbb{X}: x + 1 \in \mathbb{X}$,
например: $[0; +\infty), \mathbb{N}, [-1; +\infty), (-\infty; +\infty)$

\subsubsection{Наименьшее индуктивное множество}

Множество $\mathbb{N}$ является наименьшим индуктивным множеством, то
есть для любого другого индуктивного множества $\mathbb{X}$ верно 
$\mathbb{N} \subseteq \mathbb{X}$

\subsubsection{Теорема о единственности множества натуральных чисел}

\textbf{Формулировка:} множество $\mathbb{N}$ существует и единственно.

Пусть $\mathbf{X}$ - множество всех индуктивных подмножеств в $\mathbb{R}$
($\mathbf{X} \neq \emptyset$, потому что $\mathbb{R}$ является индуктивным множеством)

Положим, что $\mathbb{N} = \underset{\mathbb{X} \in \mathbf{X}}{\cap{\mathbb{X}}} =
\left\{x: \forall \mathbb{X} \in \mathbf{X}, x \in \mathbb{X}\right\}$

Т.к. $0 \in \mathbb{N}$, потому что $\forall \mathbb{X} \in \mathbf{X}: 0 \in \mathbb{X}$
$\Rightarrow 0 \in \underset{\mathbb{X} \in \mathbf{X}}{\cap{\mathbb{X}}} = \mathbb{N}$

Если $x \in \mathbb{N}$, то $x \in \underset{\mathbb{X}\in \mathbf{X}}{\cap{\mathbb{X}}}$,
то $\forall \mathbb{X} \in \mathbf{X}: x \in \mathbb{X}$. Т.к.
множества $\mathbb{X} \in \mathbf{X}$ индуктивны, то и
$\forall \mathbb{X} \in \mathbf{X}: x + 1 \in \mathbb{X}$
$\Rightarrow$ $x \in \mathbb{N}$ и $x + 1 \in \mathbb{N}$
$\Rightarrow$ множество $\mathbb{N}$ индуктивное.

$\forall$ индуктивного множества $\mathbb{X}_i$ верно что
$\mathbb{X}_i \in \mathbf{X}$. Т.к.
$\mathbb{N} = \underset{\mathbb{X}\in \mathbf{X}}{\cap{\mathbb{X}}}$, a
$\mathbb{X}_i \in \mathbf{X}$, то $\mathbb{N} \subseteq \mathbb{X}_i$

\textbf{Вывод:} множество $\mathbb{N}$ содержится в любом индуктивном
множестве $\mathbb{X}_i$ и является наименьшим индуктивным.

Допустим, что $\tilde{\mathbb{N}}$ обладает теми же свойствами,
что и $\mathbb{N}$, тогда $\mathbb{N} \subseteq \tilde{\mathbb{X}}$
(т.к. $\mathbb{N}$ минимальное по мощности) и
$\tilde{\mathbb{N}} \subseteq \mathbb{N}$
(т.к $\tilde{\mathbb{N}}$ тоже минимальное по мощности).
Отсюда следует, что $\mathbb{N}$ и $\tilde{\mathbb{N}}$
минимальные по мощности индуктивные множества
$\Rightarrow \mathbb{N} = \tilde{\mathbb{N}} \Rightarrow$
множество натуральных чисел единственно.

\end{document}
