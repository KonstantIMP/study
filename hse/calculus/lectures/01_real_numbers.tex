% Лекция по Математическому Анализу 07.09.2022
% Михедов Константин Константиновчи, БИБ 224

% Тип документа: статья, размер бумаги - A4, написано 14 кегелем
% Предназначено для импорта из другого документа
\documentclass[class=article,a4paper,12pt,crop=false]{standalone}

% Поиск по скомпилированному PDF
\usepackage{cmap}
% Кодировка выходного текста
\usepackage[T2A]{fontenc}
% Кодировка исходного текста
\usepackage[utf8]{inputenc}
% Поддержка необходимых языков
\usepackage[english,russian]{babel}

% Поддержка изображений
\usepackage{graphicx}
% Путь до внешних изображений
\graphicspath{ {./figures/}}

% Умная запятая
\usepackage{icomma}

% Ссылки на электронные ресурсы
\usepackage{hyperref}
% Настройка внешнего вида ссылок
\hypersetup{
  % Отключить прямоугольную рамку
  pdfborder={0 0 0},
  % Включить цветные ссылки
  colorlinks=true,
  % Цвет для ссылок на веб-ресурсы
  urlcolor=blue,
  % Цвет внутренних ссылок
  linkcolor=black
}

% Дополнительная математика
\usepackage{amsmath,amsfonts,amssymb,amsthm,mathtools}
% Показывать номера только у тех выржений, на которые кто-то ссылается
\mathtoolsset{showonlyrefs=true}

% Дополнительные символы
\usepackage{mathbbol}

% Подключние пакетов для импорта других .tex
\usepackage[subpreambles=true]{standalone}
\usepackage{import}

% Правильное оформление
% Настройка отступов
\usepackage[left=2cm,right=1cm,top=2cm,bottom=2cm]{geometry}
% Настройка шрифта
\usepackage{fontspec}
\setmainfont{Times New Roman}
% Настройка межстрочных интервалов
\usepackage{setspace}
\onehalfspacing
% и межабзацных
\usepackage{parskip}
\setlength{\parindent}{1.25cm} 

% Красная строка
\usepackage{indentfirst}
\setlength{\parindent}{1.25cm} 

% Корректное положение рисунков?
\usepackage{float}

% Расположение блоков относительно текста
\usepackage{wrapfig}

\begin{document}
  \subsection{Как определить вещественные числа?}

  \begin{itemize}
    \item {
      Теории первого порядка -
      полное определение с нуля, без ничего (например: ZF, GB, MK)
    }
    \item {
      Сконструировать вещественные числа внутри теории множеств
    }
    \item {
      Расширить существующую теорию множеств
      (построить в ней новую теорию, как дифференциальное расширение)
    }
  \end{itemize}

  \subsection{Модель вещественных чисел}

  Моделью вещественных чисел называют шестерку объектов: \[
    \left\{ \mathbb{R}, \; 0, \;  1, \;  +, \;  \times, \;  \leq \right\} 
  \] где $\mathbb{R}$ - некоторое множество, 0 и 1 - элементы множества
  $\mathbb{R}$, $+$ и $\times$ - некоторые бинарные операции, а $\leq$ - 
  бинарное отношение.

  \subsubsection{Бинарные операции и отношения}

  Бинарные операции $+$ и $\times$ замкнуты на множестве $\mathbb{R}$,
  то есть $\mathbb{R} + \mathbb{R} \rightarrow \mathbb{R}$ и
  $\mathbb{R} \times \mathbb{R} \rightarrow \mathbb{R}$

  \begin{equation}
    \begin{aligned}
      \underbrace{\left(x, \; y \right)}_{\mathbb{R} \times \mathbb{R}} &
      \rightarrow \underbrace{x + y}_{\mathbb{R}} \\
      \underbrace{\left(x, \; y \right)}_{\mathbb{R} \times \mathbb{R}} &
      \rightarrow \underbrace{x \times y}_{\mathbb{R}}
    \end{aligned}
  \end{equation}

  $\forall x,y\in\mathbb{R}$ бинарное отношение $x \leq y$ верно или неверно.

  \subsection{Аксиомы вещественных чисел}

  \begin{enumerate}
    \item[A1.] {
      Коммутативность сложения:
      \[x + y = y + x\]
    }
    \item[A2.] {
      Коммутативность умножения:
      \[x \times y = y \times x\]
    }
    \item[A3.] {
      Ассоциативность сложения:
      \[\left(x + y \right) + z = x + \left(y + z\right)\]
    }
    \item[A4.] {
      Ассоциативность умножения:
      \[\left(x \times y \right) \times z = x \times \left(y \times z\right)\]
    }
    \item[A5.] {
      Закон дистрибутивности:
      \[\left(x + y\right) \times z = x \times z + y \times z\]
    }
    \item[A6.] {
      Свойство нуля:
      \[x + 0 = x\]
    }
    \item[A7.] {
      Существование противоположного элемента:
      \[\forall x \in \mathbb{R}, \exists {-x} \in \mathbb{R}: x + \left(-x\right) = 0\]
    }
    \item[A8.] {
      Свойство единицы:
      \[x \times 1 = x\]
    }
    \item[A9.] {
      Существование обратного элемента:
      \[\forall x \in \mathbb{R} \land x \neq 0, \exists x^{-1}: x\times{x^{-1}} = 1 \]
    }
    \item[A10.] {
      Рефлексивность:
      \[\forall x \in \mathbb{R}, x \leq x \text{ - верно}\]
    }
    \item[A11.] {
      Транзитивность:
      \[x \leq y \land y \leq z \Rightarrow x \leq z\]
    }
    \item[A12.] {
      Антисимметричность:
      \[x \leq y \land y \leq x \Rightarrow x = y\]
    }
    \item[A13.] {
      Линейность:
      \[\forall x,y \in \mathbb{R}, x\leq y \lor y \leq x\]
    }
    \item[A14.] {
      Монотонность сложения:
      \[x \leq y \Rightarrow x + z \leq y + z\]
    }
    \item[A15.] {
      Сохранение знака при умножении:
      \[x \geq 0 \land y \geq 0 \Rightarrow x \times y \geq 0\]
    }
    \item[A16.] {
      Непрерывность:
      
      \begin{equation}
        \mathbb{X},\mathbb{Y}\subseteq \mathbb{R}:
        \forall x \in \mathbb{X} \text{ и } \forall y \in \mathbb{Y}
        \text{ если } x \leq y, \text{ то } \exists{c} \in \mathbb{R}:
        x \leq c \leq y
      \end{equation}

      \begin{picture}(380,50)
        \put(15,35){\line(1,0){35}}
        \qbezier(50,35)(60,35)(60,25)

        \qbezier(100,25)(100,35)(110,35)
        \put(110,35){\line(1,0){35}}

        \put(15,25){\vector(1,0){130}}

        \put(37.5,25){\circle*{3}}
        \put(35, 15){x}

        \put(80,25){\circle*{3}}
        \put(77.5, 15){c}

        \put(122.5,25){\circle*{3}}
        \put(120, 15){y}

        \put(140, 10){$\mathbb{R}$}
        \put(32.5, 40){$\mathbb{X}$}
        \put(117.5, 40){$\mathbb{Y}$}

        \put(17.5,35){\line(1,-1){10}}
        \put(27.5,35){\line(1,-1){10}}
        \put(37.5,35){\line(1,-1){10}}
        \put(47.5,35){\line(1,-1){10}}

        \put(102.5,25){\line(1,1){10}}
        \put(112.5,25){\line(1,1){10}}
        \put(122.5,25){\line(1,1){10}}
        \put(132.5,25){\line(1,1){10}}

        \put(180, 25){или}

        \put(235,35){\line(1,0){55}}
        \qbezier(290,35)(300,35)(300,25)

        \qbezier(300,25)(300,15)(310,15)
        \put(310,15){\line(1,0){55}}

        \put(235,25){\vector(1,0){130}}

        \put(300,25){\circle*{5}}
        \put(290, 15){c}

        \put(267.5,25){\circle*{3}}
        \put(265, 15){x}

        \put(332.5,25){\circle*{3}}
        \put(330, 30){y}

        \put(360, 30){$\mathbb{R}$}
        \put(262.5, 40){$\mathbb{X}$}
        \put(330, 2.5){$\mathbb{Y}$}

        \put(240,35){\line(1,-1){10}}
        \put(250,35){\line(1,-1){10}}
        \put(260,35){\line(1,-1){10}}
        \put(270,35){\line(1,-1){10}}
        \put(280,35){\line(1,-1){10}}
        \put(290,35){\line(1,-1){10}}

        \put(300,25){\line(1,-1){10}}
        \put(310,25){\line(1,-1){10}}
        \put(320,25){\line(1,-1){10}}
        \put(330,25){\line(1,-1){10}}
        \put(340,25){\line(1,-1){10}}
        \put(350,25){\line(1,-1){10}}
      \end{picture}
    }
  \end{enumerate}

  \subsection{Дополнительные бинарные операции}

  \begin{equation}
    \begin{aligned}
      x \geq y & \Leftrightarrow y \leq x \\
      x < y & \Leftrightarrow x \leq y \land x \neq y \\
      x > y & \Leftrightarrow y < x
    \end{aligned}
  \end{equation}

  \subsection{Примеры}

  \subsubsection{Однозначность определения нуля}

  \textbf{Утверждение:} число $0$ однозначно определеяется с помощью
  аксиом вещественных чисел (может быть только один $0$).

  Пусть $0$ и $\widetilde{0}$ два различных числа таких, что $0$ и $\widetilde{0}$ $\in \mathbb{R}$, тогда
  по свойству нуля (A6.) $\forall{x}\in \mathbb{R}: x + 0 = x \text{ и } x + \widetilde{0} = x$

  Пусть $\star : x + 0 = x$, а $\star \star : x + \widetilde{0} = x$,
  тогда $\widetilde{0} \overset{\star}{=} \widetilde{0} + 0$ и 
  $0 \overset{\star \star}{=} 0 + \widetilde{0}$, но исходя из
  коммутативности сложения (A1.) $0 + \widetilde{0} = \widetilde{0} + 0$,
  поэтому $0 = \widetilde{0}$ $\Rightarrow$ существует только один ноль.

  \subsubsection{Однозначность противоположного числа}

  \textbf{Утверждение:} для данного $x \in \mathbb{R}$ противоположное
  число $-x \in \mathbb{R}$ единственно.

  Допустим, для указанного $x \in \mathbb{R}$ $\exists a, b \in \mathbb{R},
  a \neq b: x + b = 0 \text{ и } x + a = 0$, то есть числа a и b 
  противоположны числу x.

  Пусть $\star : x + b = 0$, а $\star \star : x + a = 0$
  
  Тогда $a \overset{A6.}{=} a + 0 \overset{A1.}{=} 0 + a \overset{\star}{=}
  (x + b) + a \overset{A3.}{=} (a + x) + b \overset{\star \star}{=}
  0 + b \overset{A6.}{=} b$

  Отсюда $a = b$, значит $\forall x \in \mathbb{R}, x \neq 0$ обратное
  число $-x$ единственно.

  %
  % Здесь должно быть еще 4 практических задания
  % На них есть решение, но нет желания его писать
  %

\end{document}
