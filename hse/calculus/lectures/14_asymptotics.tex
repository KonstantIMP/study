% Лекция по Математическому Анализу 16.11.2022
% Михедов Константин Константиновчи, БИБ 224

% Тип документа: статья, размер бумаги - A4, написано 14 кегелем
% Предназначено для импорта из другого документа
\documentclass[class=article,a4paper,12pt,crop=false]{standalone}

% Поиск по скомпилированному PDF
\usepackage{cmap}
% Кодировка выходного текста
\usepackage[T2A]{fontenc}
% Кодировка исходного текста
\usepackage[utf8]{inputenc}
% Поддержка необходимых языков
\usepackage[english,russian]{babel}

% Поддержка изображений
\usepackage{graphicx}
% Путь до внешних изображений
\graphicspath{ {./figures/} {./../figures/}}

% .esp support
\usepackage{epstopdf}

% Умная запятая
\usepackage{icomma}

% Ссылки на электронные ресурсы
\usepackage{hyperref}
% Настройка внешнего вида ссылок
\hypersetup{
  % Отключить прямоугольную рамку
  pdfborder={0 0 0},
  % Включить цветные ссылки
  colorlinks=true,
  % Цвет для ссылок на веб-ресурсы
  urlcolor=blue,
  % Цвет внутренних ссылок
  linkcolor=black
}

% Дополнительная математика
\usepackage{amsmath,amsfonts,amssymb,amsthm,mathtools}
% Показывать номера только у тех выржений, на которые кто-то ссылается
\mathtoolsset{showonlyrefs=true}

% Дополнительные символы
\usepackage{mathbbol}

% Подключние пакетов для импорта других .tex
\usepackage[subpreambles=true]{standalone}
\usepackage{import}

% Правильное оформление
% Настройка отступов
\usepackage[left=2cm,right=1cm,top=2cm,bottom=2cm]{geometry}
% Настройка шрифта
\usepackage{fontspec}
\setmainfont{Times New Roman}
% Настройка межстрочных интервалов
\usepackage{setspace}
\onehalfspacing
% и межабзацных
\usepackage{parskip}
\setlength{\parindent}{1.25cm} 

% Красная строка
\usepackage{indentfirst}
\setlength{\parindent}{1.25cm} 

% Корректное положение рисунков?
\usepackage{float}

% Расположение блоков относительно текста
\usepackage{wrapfig}

% Функция card
\DeclareMathOperator{\card}{card}

% зачеркивание
\usepackage{cancel}

% Теперь мы еще и графики рисуем
\usepackage{pgfplots}
\usetikzlibrary{
  calc,
  pgfplots.groupplots,
}
\pgfplotsset{compat=1.16}

% Списки в два и более стобцов
\usepackage{multicol}

\begin{document}

Пусть $a$ - точка на прямой $\mathbb{R}$ или символ бесконечности (возможно со знаком),
а $f_1$ и $f_2$ - функции, определенные в выколотой окрестности $a$, тогда говорят, что
$f_1$ эквивалентна $f_2$ пр  $x \rightarrow a$, если существует функция $\Theta$,
определенная в выколотой окрестности $a$:
\begin{equation}
    \begin{cases}
        \Theta(x) \underset{x \rightarrow a}{\longrightarrow} 1 \\
        f_1(x) = \Theta(x)f_2(x)
    \end{cases} \Rightarrow f_1(x) \underset{x \rightarrow a}{\sim } f_2(x)
\end{equation}

\subsection{Примеры для стандартных функций}
\begin{multicols}{2}
    \begin{enumerate}
        \item $\sin x \underset{x \rightarrow 0}{\sim} x$
        \item $\tg x \underset{x \rightarrow 0}{\sim} x$
        \item $e^x - 1 \underset{x \rightarrow 0}{\sim} x$
        \item $a^x - 1 \underset{x \rightarrow 0}{\sim} x\ln{a}$
    \end{enumerate}
\end{multicols}

\subsection{Свойства асимптотической эквивалентности}

\begin{enumerate}
    \item {
        Если $f_1(x) \underset{x \rightarrow a}{\sim} f_2(x)$, то
        \begin{equation}
            \begin{aligned}
                f_1(x) = 0, x\in \overset{\cdot}{v}_\varepsilon(a) \Leftrightarrow
                f_2(x) = 0, x\in \overset{\cdot}{v}_\varepsilon(a) \\
                f_1(x) \neq 0, x\in \overset{\cdot}{v}_\varepsilon(a) \Leftrightarrow
                f_2(x) \neq 0, x\in \overset{\cdot}{v}_\varepsilon(a) \\
            \end{aligned}
        \end{equation}
    }
    \item {
        Рефлексивность: $f_1(x) \underset{x \rightarrow a}{\sim} f_1(x)$
    }
    \item {
        Симметричность: $f_1(x) \underset{x \rightarrow a}{\sim} f_2(x) \Rightarrow
        f_2(x) \underset{x \rightarrow a}{\sim} f_1(x)$
    }
    \item {
        Транзитивность
        \begin{equation}
            f_1(x) \underset{x \rightarrow a}{\sim} f_2(x) \land
            f_2(x) \underset{x \rightarrow a}{\sim} f_3(x) \Rightarrow f_1(x) \underset{x \rightarrow a}{\sim} f_3(x)
        \end{equation}
    }
    \item {
        Мультипликативность
        \begin{equation}
            \begin{cases}
                f_1(x) \underset{x \rightarrow a}{\sim} f_2(x) \\
                g_1(x) \underset{x \rightarrow a}{\sim} g_2(x)
            \end{cases} \Rightarrow
            f_1(x)g_1(x) \underset{x \rightarrow a}{\sim} f_2(x)g_2(x) \land
            \frac{f_1(x)}{g_1(x)} \underset{x \rightarrow a}{\sim} \frac{f_2(x)}{g_2(x)}
        \end{equation}
    }
    \item {
        Связь с пределом
        \begin{equation}
            f_1(x) \underset{x \rightarrow a}{\sim} f_2(x) \underset{x \rightarrow a}{\longrightarrow} A
            \Rightarrow f_1(x) \underset{x \rightarrow a}{\longrightarrow} A
        \end{equation}
    }
\end{enumerate}

\subsection{Асимптотическое сравнение функций}

Говорят, что функции $f$ бесконечно мала по сравнению с функцией $g$ при $x \rightarrow a$ 
(запись выглядит так:  $f(x) \underset{x \rightarrow a}{<<} g(x)$), если $\exists \alpha$:
$f(x) = \alpha(x)g(x)$ при $\alpha(x) \underset{x \rightarrow a}{\longrightarrow} 0$

\subsection{Шкала бесконечности}

\begin{equation}
    C << \log_a{x} << x^{\lambda} << a^x (x \rightarrow +\infty)
\end{equation}

\end{document}
