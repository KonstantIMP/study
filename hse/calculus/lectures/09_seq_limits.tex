% Лекция по Математическому Анализу 10.10.2022
% Михедов Константин Константиновчи, БИБ 224

% Тип документа: статья, размер бумаги - A4, написано 14 кегелем
% Предназначено для импорта из другого документа
\documentclass[class=article,a4paper,12pt,crop=false]{standalone}

% Поиск по скомпилированному PDF
\usepackage{cmap}
% Кодировка выходного текста
\usepackage[T2A]{fontenc}
% Кодировка исходного текста
\usepackage[utf8]{inputenc}
% Поддержка необходимых языков
\usepackage[english,russian]{babel}

% Поддержка изображений
\usepackage{graphicx}
% Путь до внешних изображений
\graphicspath{ {./figures/} {./../figures/}}

% .esp support
\usepackage{epstopdf}

% Умная запятая
\usepackage{icomma}

% Ссылки на электронные ресурсы
\usepackage{hyperref}
% Настройка внешнего вида ссылок
\hypersetup{
  % Отключить прямоугольную рамку
  pdfborder={0 0 0},
  % Включить цветные ссылки
  colorlinks=true,
  % Цвет для ссылок на веб-ресурсы
  urlcolor=blue,
  % Цвет внутренних ссылок
  linkcolor=black
}

% Дополнительная математика
\usepackage{amsmath,amsfonts,amssymb,amsthm,mathtools}
% Показывать номера только у тех выржений, на которые кто-то ссылается
\mathtoolsset{showonlyrefs=true}

% Дополнительные символы
\usepackage{mathbbol}

% Правильное оформление
% Настройка отступов
\usepackage[left=2cm,right=1cm,top=2cm,bottom=2cm]{geometry}
% Настройка шрифта
\usepackage{fontspec}
\setmainfont{Times New Roman}
% Настройка межстрочных интервалов
\usepackage{setspace}
\onehalfspacing
% и межабзацных
\usepackage{parskip}
\setlength{\parindent}{1.25cm} 

% Красная строка
\usepackage{indentfirst}
\setlength{\parindent}{1.25cm} 

% Корректное положение рисунков?
\usepackage{float}

% Расположение блоков относительно текста
\usepackage{wrapfig}

% Подключние пакетов для импорта других .tex
\usepackage[subpreambles=true]{standalone}
\usepackage{import}

% Функция card
\DeclareMathOperator{\card}{card}

% зачеркивание
\usepackage{cancel}

% Теперь мы еще и графики рисуем
\usepackage{pgfplots}
\usetikzlibrary{
  calc,
  pgfplots.groupplots,
}
\pgfplotsset{compat=1.16}

% Списки в два и более стобцов
\usepackage{multicol}

\begin{document}

\subsection{Пределы последовательностей}

\subsubsection{Окрестность точки}

$v_{\varepsilon}(a) = (a - \varepsilon; a + \varepsilon)$ - эпсилон($\varepsilon$) окрнесность точки $a$

\begin{picture}(300, 50)
  \put(50, 15){\vector(1,0){200}}

  \put(150, 15){\circle*{4}}
  \put(147.5, 0){$a$}

  \put(100, 14.75){\circle{5}}
  \put(87.5, 0){$a - \varepsilon$}

  \put(200, 14.75){\circle{5}}
  \put(187.5, 0){$a + \varepsilon$}

  \bezier{0}(100,15)(100,25)(110,25)
  \put(110,25){\line(1,0){80}}
  \bezier{0}(200,15)(200,25)(190,25)

  \put(100, 15){\line(1,1){10}}
  \put(120, 15){\line(1,1){10}}
  \put(140, 15){\line(1,1){10}}
  \put(160, 15){\line(1,1){10}}
  \put(180, 15){\line(1,1){10}}

  \put(135, 35){$v_{\varepsilon}(a)$}
\end{picture}

\subsubsection{Вводное определение предела последовательности}

$a = \lim{x_n} (\text{или } x_n \underset{n \rightarrow \infty}{\rightarrow} a)$, если
$\forall v_{\varepsilon}(a)$ содержит почти все значения последовательности $\{x_n\}$

Или по-другому $\forall \varepsilon > 0$ $x_n \in v_{\varepsilon}(a)$ верно для почти всех $x_n$

\subsubsection{Точное определение предела последовательности}

$a = \lim{x_n}$ если $\forall \varepsilon > 0$ $\exists N \in \mathbb{N}:$ $\forall n > N$ $|x_n - a| < \varepsilon$

\paragraph{Пример} Докажем, что $\lim{\frac{n - 1}{n}} = 1$

Зафиксируем $\varepsilon > 0$, тогда $x_n \in v_{\varepsilon}(a) \underset{a = 1}{\Leftrightarrow}$
$\frac{n - 1}{n} \in (1 - \varepsilon; 1 + \varepsilon)$, что равносильно системе:
\begin{equation}
  \begin{cases}
    \frac{n - 1}{n} &< 1 + \varepsilon \\
    \frac{n - 1}{n} &> 1 - \varepsilon
  \end{cases} \Leftrightarrow
  \begin{cases}
    \frac{-1}{n} &< \varepsilon \\
    \frac{-1}{n} &> -\varepsilon
  \end{cases} \Leftrightarrow
  \begin{cases}
    -1 &< n\varepsilon \\
    -1 &> -n\varepsilon
  \end{cases}
\end{equation}

Т.к. $\varepsilon > 0$ и $n >= 0$, то $-1 < n\varepsilon$ верно всегда, что же касается $-1 > -n\varepsilon$,
то из него следует, что $1 < n\varepsilon$ $\Leftrightarrow$ $n > \varepsilon^{-1}$, что по теореме
Архимеда верно для почти всех $n$ $\Rightarrow$ 1 действительно является пределом данной
последовательности $(\blacksquare)$

\subsubsection{Бесконечные пределы}

Говорят, что $\{x_n\} \underset{n \rightarrow \infty}{\rightarrow} +\infty$, если $\forall E \in \mathbb{R}$
условие $x_n > E$ выполняется для почти всех $n$. Аналогично $\{x_n\} \underset{n \rightarrow \infty}{\rightarrow} -\infty$,
если $\forall I \in \mathbb{R}$ $x_n < I$ верно для почти всех $n$

Еще $\{x_n\} \underset{n \rightarrow \infty}{\rightarrow} \infty$, если $\{|x_n|\} \underset{n \rightarrow \infty}{\rightarrow} +\infty$

\subsubsection{Бесконечно-малые и -большие последовательности}

Последовательность $\{x_n\}$ называют
\begin{itemize}
  \item {
    Бесконечно-малой, если $\lim\limits_{n \rightarrow \infty}{x_n} = 0$
  }
  \item {
    Бесконечно-большой, если $\lim\limits_{n \rightarrow \infty}{x_n} = \infty$
  }
\end{itemize}

\subsubsection{Теорема}

Пусть $x_n \neq 0$, тогда $\{x_n\}$ бесконечно-мала, только тогда, когда $\{\frac{1}{x_n}\}$ бесконечно-велика

\subsubsection{Еще теорема}

$\lim\limits_{n \rightarrow \infty}{x_n} = a \in \mathbb{R}$ в том, и только том случае, если последовательность
$\{x_n\}$ представима в виде $x_n = a + \alpha_n$, где $\alpha_n$ - бесконечно-малая последовательность

\subsubsection{Арифметические свойства пределов}

\begin{enumerate}
  \item {
    $x_n \underset{n \rightarrow \infty}{\rightarrow} a$ и $c \in \mathbb{R}$:
    $cx_n \underset{n \rightarrow \infty}{\rightarrow} ca$
  }
  \item {
    $x_n \underset{n \rightarrow \infty}{\rightarrow} a$ и $y_n \underset{n \rightarrow \infty}{\rightarrow} b$, то
    $x_n + y_n \underset{n \rightarrow \infty}{\rightarrow} a + b$ (верно также для $-$, $\times$, а еще для
    $\div$ при $b \neq 0$ )
  }
\end{enumerate}

\subsubsection{Теорема о предельном переходе в неравенство}

Пусть $\exists$ $\{x_n\}$ и $\{y_n\}$, такие что $\{x_n\} \leq \{y_n\}$ и $\lim\limits_{n \rightarrow \infty}x_n = a$,
а $\lim\limits_{n \rightarrow \infty}y_n = b$, тогда $a \leq b$

\paragraph{Доказательство} Предположим, что это не так, тогда $a > b$

Пусть $\varepsilon = \frac{a - b}{2}$ ($\varepsilon > 0$, т.к. $a > b$), тогда

\[\begin{aligned}
  v_{\varepsilon}(b) = (b - \varepsilon; b + \varepsilon) =
  (b - \frac{a - b}{2}; b + \frac{a - b}{2}) = (\frac{3b - a}{2}; \frac{b + a}{2}) \\
  v_{\varepsilon}(a) = (a - \varepsilon; a + \varepsilon) =
  (a - \frac{a - b}{2}; a + \frac{a - b}{2}) = (\frac{a + b}{2}; \frac{3a - b}{2})
\end{aligned}\]

Так как почти все $x_n \in v_{\varepsilon}(a)$ и почти все $y_n \in v_{\varepsilon}(b)$, то
$y_n \in (\frac{3b - a}{2}; \frac{a + b}{2})$, а $x_n \in (\frac{a + b}{2}; \frac{3a - b}{2})$, то
есть $y_n < x_n$, но это противоречит условию $\Rightarrow$ теорема верна $(\blacksquare)$

\subsection{Теорема о двух миллиционерах}

Пусть $\{x_n\}$, $\{y_n\}$ и $\{z_n\}$ - последовательности, причем $\forall n \in \mathbb{N}$ верно
$x_n \leq y_n \leq z_n$ и $\lim\limits_{n \rightarrow \infty}x_n = \lim\limits_{n \rightarrow \infty}y_n = c \in \mathbb{R}$,
тогда и $\lim\limits_{n \rightarrow \infty}z_n = c \in \mathbb{R}$

\paragraph{Доказательство} Зафиксируем $\varepsilon > 0$, тогда:
\begin{equation}
  \begin{aligned}
    \text{т.к. } x_n \underset{n \rightarrow \infty}{\rightarrow} c, \text{ то } x_n \in
    v_{\varepsilon}(c)& \text{ верно для почти всех } n, \text{ и} \\
    \text{т.к. } y_n \underset{n \rightarrow \infty}{\rightarrow} c, \text{ то } y_n \in
    v_{\varepsilon}(c)& \text{ также верно для почти всех } n
  \end{aligned}
\end{equation}

$\Rightarrow$ выражения $x_n > c - \varepsilon$ и $z_n < c + \varepsilon$ верны для почти всех $n$ 

По условию $x_n \leq y_n \leq z_n$, тогда из ранее доказанного для почти всех $n$ будет верно, что
$c - \varepsilon < x_n \leq y_n \leq z_n < c + \varepsilon$ $\Leftrightarrow$
$c - \varepsilon < y_n < c + \varepsilon$

Получается, что $y_n \in (c - \varepsilon; c + \varepsilon) = v_{\varepsilon}(c)$ для почти
всех $n$, следовательно $\lim\limits_{n \rightarrow \infty}y_n = c$ $(\blacksquare)$

\subsubsection{Теорема Вейерштрасса (какая?)}

Любая монотонная ограниченная последовательность имеет предел
(если $\{x_n\}$ ограничена и монотонна, то $\exists \lim\limits_{n \rightarrow \infty}x_n = c$, причем $c \in \mathbb{R}$)

\paragraph{Доказательство} (для неубывающей последовательности)

Пусть $\{x_n\}$ неубывающая и ограничена, тогда $\exists \sup{x_n} = A \in \mathbb{R}$ и
$\exists \inf(x_n) = B \in \mathbb{R}$. Зафиксируем $\varepsilon > 0$, тогда $A - \varepsilon \neq \sup{x_n}$,
то есть $A - \varepsilon$ - неточная верхняя грань. Тогда $\exists k \in \mathbb{N}$, что
$x_k > A - \varepsilon$ $\Rightarrow$ $\forall n > k:$ $A - \varepsilon < x_k \leq x_n$
$\Rightarrow$ $\forall n > k:$ $A-\varepsilon < x_n < A + \varepsilon$ $\Rightarrow$
$A = \lim\limits_{n \rightarrow \infty}x_n$ $(\blacksquare)$

\subsubsection{Вложенные отрезки}

$\{[a_n; b_n]\}$ - называется последовательностью вложенных отрезков, если верно, что $[a_1; b_1] \supseteq [a_2; b_2]
\supseteq \dots \supseteq [a_n; b_n]$, или что $\forall n\in \mathbb{N}:$ $a_{n + 1} \geq a_n$ и $b_n \leq b_{n + 1}$

Причем $[a_n; b_n] \underset{n \rightarrow \infty}{\rightarrow} 0$ $\Leftrightarrow$
$b_n - a_n \underset{n \rightarrow \infty}{\rightarrow} 0$

\subsubsection{Теорема о вложенных отрезках}

Для любой последовательности вложенных отрезков $\{[a_n; b_n]\}$ $\exists ! c \in \mathbb{R}:$
$c \in [a_n; b_n] \forall n$ (причем $c = \lim\limits_{n \rightarrow \infty}a_n = \lim\limits_{n \rightarrow \infty}b_n $)

\paragraph{Доказательство} Исходя из определения последовательности вложенных отрезков, последовательность
$\{a_n\}$ - неубываюшая и ограниченная ($a_1 \in \mathbb{R}$ и $a_n \leq b_n$), а последовательность
$\{b_n\}$ - невозрастающая и ограниченная ($b_1 \in \mathbb{R}$ и $b_n \geq a_n$)

$\Rightarrow$ $\exists \lim\limits_{n \rightarrow \infty} a_n = \sup{a_n}$ и $\exists \lim\limits_{n \rightarrow \infty} b_n = \inf b_n$

Заметим, что $\lim\limits_{n \rightarrow \infty}(b_n - a_n) = \lim\limits_{n \rightarrow \infty}b_n -
\lim\limits_{n \rightarrow \infty} a_n = 0$ $\Rightarrow$ $\lim\limits_{n \rightarrow \infty} a_n = 
\lim\limits_{n \rightarrow \infty} b_n$ 

P.s. последовательности монтонны и ограничены, поэтому обязательно имеют действительный предел ($= c \in \mathbb{R}$)

Предположим, что $\exists \tilde{c}:$ $\forall n$ $\tilde{c} \in [a_n; b_n]$ и $\tilde{c} \neq c$.
Т.к. $\tilde{c} \neq c$, то или $\underline{c < \tilde{c}}$, или $c > \tilde{c}$
$\Rightarrow$ $\forall n$ $[c; \tilde{c}] \subseteq [a_n; b_n]$ $\Rightarrow$
$b_n - a_n \geq \tilde{c} - c = const > 0$ $\Rightarrow$ возникает противоречие уловию:
$b_n - a_n \cancel{\underset{n \rightarrow \infty}{\rightarrow}} 0$ и
$a \underset{n \rightarrow \infty}{\rightarrow} \tilde{c} - c$ $(\blacksquare)$

\end{document}
