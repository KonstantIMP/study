% Лекция по Математическому Анализу 02.11.2022
% Михедов Константин Константиновчи, БИБ 224

% Тип документа: статья, размер бумаги - A4, написано 14 кегелем
% Предназначено для импорта из другого документа
\documentclass[class=article,a4paper,12pt,crop=false]{standalone}

% Поиск по скомпилированному PDF
\usepackage{cmap}
% Кодировка выходного текста
\usepackage[T2A]{fontenc}
% Кодировка исходного текста
\usepackage[utf8]{inputenc}
% Поддержка необходимых языков
\usepackage[english,russian]{babel}

% Поддержка изображений
\usepackage{graphicx}
% Путь до внешних изображений
\graphicspath{ {./figures/} {./../figures/}}

% .esp support
\usepackage{epstopdf}

% Умная запятая
\usepackage{icomma}

% Ссылки на электронные ресурсы
\usepackage{hyperref}
% Настройка внешнего вида ссылок
\hypersetup{
  % Отключить прямоугольную рамку
  pdfborder={0 0 0},
  % Включить цветные ссылки
  colorlinks=true,
  % Цвет для ссылок на веб-ресурсы
  urlcolor=blue,
  % Цвет внутренних ссылок
  linkcolor=black
}

% Дополнительная математика
\usepackage{amsmath,amsfonts,amssymb,amsthm,mathtools}
% Показывать номера только у тех выржений, на которые кто-то ссылается
\mathtoolsset{showonlyrefs=true}

% Дополнительные символы
\usepackage{mathbbol}

% Правильное оформление
% Настройка отступов
\usepackage[left=2cm,right=1cm,top=2cm,bottom=2cm]{geometry}
% Настройка шрифта
\usepackage{fontspec}
\setmainfont{Times New Roman}
% Настройка межстрочных интервалов
\usepackage{setspace}
\onehalfspacing
% и межабзацных
\usepackage{parskip}
\setlength{\parindent}{1.25cm} 

% Красная строка
\usepackage{indentfirst}
\setlength{\parindent}{1.25cm} 

% Корректное положение рисунков?
\usepackage{float}

% Расположение блоков относительно текста
\usepackage{wrapfig}

% Подключние пакетов для импорта других .tex
\usepackage[subpreambles=true]{standalone}
\usepackage{import}

% Функция sgn
\DeclareMathOperator{\sgn}{sgn}

% Теперь мы еще и графики рисуем
\usepackage{pgfplots}
\usetikzlibrary{
  calc,
  pgfplots.groupplots,
}
\pgfplotsset{compat=1.16}

% Списки в два и более стобцов
\usepackage{multicol}

% Перечеркивание элементов
\usepackage{cancel}

\begin{document}

\subsection{Непрерывные функции}

\textit{Здесь могла быть ваша реклама, а должны быть рисунки}

Если функция $f(x)$ непрерывна в точке $a$, то из того, что $x_n \underset{n \rightarrow \infty}{\rightarrow} a$
$\Rightarrow$ $f(x_n) \underset{n \rightarrow \infty}{\rightarrow} f(a)$, иначе такой переход совершать нельзя

\paragraph{Точное определение} $f(x)$ непрерывна в точке $a$ на множестве $\mathbb{E}$, если функция $f(x)$
определена на этом множетсве, $a \in \mathbb{E}$ и $\forall x_n \in \mathbb{E}$
$x_n \underset{n \rightarrow \infty}{a}$ $\Rightarrow$ $f(x_n) \underset{n \rightarrow \infty}{\rightarrow} f(a)$

\paragraph{И еще определение} $f(x)$ непрерывна на множестве $\mathbb{E}$, если 
$\forall x_n \in \mathbb{E}$ $\forall a \in \mathbb{E}$: $x_n \underset{n \rightarrow \infty}{\rightarrow} a$
$\Rightarrow$ $f(x_n) \underset{n \rightarrow \infty}{\rightarrow} f(a)$

\textit{Здесь примеры прерывной функции (sgn) и непрерывной (x), img}

\subsubsection{Операции с непрерывными функциями}

Пусть $f$ и $g$ - непрерывные функции на множестве $\mathbb{E}$, тогда

\begin{enumerate}
    \item {
        $\forall c \in \mathbb{R}$ $c \times f$ также непрерывна на $\mathbb{E}$
    }
    \item {
        $f + g$, $f - g$, $f \times g$ также непрерывны на $\mathbb{E}$
    }
    \item {
        Если $f \neq 0$ на $\mathbb{E}$, то и $\frac{g}{f}$ также непрерывна на $\mathbb{E}$
    }
\end{enumerate}

\paragraph{Доказательство}

Пусть $\exists \{x_n\}$ и $a \in \mathbb{E}$: $x_n \underset{n \rightarrow \infty}{\rightarrow} a$,
тогда $f(x_n) \underset{n \rightarrow \infty}{\rightarrow} f(a)$ и
$g(x_n) \underset{n \rightarrow \infty}{\rightarrow} g(a)$ и $\Rightarrow$
$f(x_n)g(x_n) \underset{n \rightarrow \infty}{ \rightarrow} f(a)g(a) = (fg)(a) = (fg)(x_n)$

Получается, $\forall \{x_n\}$ и $a \in \mathbb{E}$ $x_n \underset{n \rightarrow \infty}{ \rightarrow} a$
$\Rightarrow$ $(fg)(x_n) \underset{n \rightarrow \infty}{\rightarrow} (fg)(a)$

Вывод: $fg$ непрерывна на $\mathbb{E}$ $(\blacksquare)$

\subsubsection{Теорема о сохранении знака}

Пусть $f(x)$ определена в некой $v_{\varepsilon}(a)$ и непрерывна в точке a, тогда
если $f(a) \neq 0$, то $\exists \delta > 0$ $\forall x \in v_{\delta}(a)$
$\sgn{f(x)} = \sgn{f(a)}$

\paragraph{Доказательство} Пусть $f(a) > 0$, тогда предположим, что исходное
утверждение неверно: $\forall \delta > 0$ $\exists$ $x \in v_{\delta}(a)$
$\sgn{f(x)} \neq \sgn{f(a)} = 1$

$\Rightarrow$ $\forall \delta > 0$ $\exists$ $x \in v_{\delta}(a)$: $f(x) \leq 0$
$\Rightarrow$ $\forall n \in \mathbb{N}$ $\exists$ $x_n \in v_{\frac{1}{n}}(a)$
$f(x_n) \leq 0$

$\Rightarrow$ $\forall n \in \mathbb{N}^{*}$ $\exists$
$x_n \in (a - \frac{1}{n}; a + \frac{1}{n})$: $f(x_n) \leq 0$

Рассмотрим эту последовательность $\{x_n\}$

$\underbrace{a - \frac{1}{n}}_{\underset{n \rightarrow \infty}{\longrightarrow  a}}
< x_n < \underbrace{a + \frac{1}{n}}_{\underset{n \rightarrow \infty}{\longrightarrow  a}}$ и
$f(x_n) \leq 0$ $\Rightarrow$
$x_n \underset{n \rightarrow \infty}{\rightarrow} a$ $\Rightarrow$
$f(x_n) \underset{n \rightarrow \infty}{\rightarrow} f(a)$

(По теореме о двух миллиционерах)

$\Rightarrow$ $\exists$ $\lim\limits_{n \rightarrow \infty}f(x) = f(a) > 0$, но с другой
стороны $f(x_n) \leq 0$ $\Rightarrow$ по теореме о переходе к пределу
$\lim\limits_{n \rightarrow \infty}f(x_n) \leq \lim f(a) = 0$, что противоречит
условию $\Rightarrow$ теорема верна $(\blacksquare)$

\subsubsection{Композиция непрерывных функций}

Пусть $f$ и $g$ - две функции, тогда
$f \circ g$ - композиция этих функций, причем $f \circ g = g(f(x))$

\paragraph{Теорема} Если $f$ и $g$ - непрерывные на всей области определения функции,
то их композиция тоже непрерывна на всей области определения

\paragraph{Доказательство} Пусть $a \in \mathbb{R}$ - точка из области определения функций
$f$ и $g$, то есть $\exists$ $g(f(a))$, тогда рассмотрим произвольную последовательность
$\{x_n\}$ из области определения $g \circ f$ такую, что $x_n \underset{n \rightarrow \infty}{\rightarrow} a$

$\Rightarrow$ $x_n \in D(g \circ f)$, $x_n \in D(f)$, $a \in D(g \circ f)$ и $a \in D(f)$
$\Rightarrow$ так как $f$ непрерывна, то $f(x_n) \underset{n \rightarrow \infty}{\rightarrow} f(a)$

\begin{multicols}{2}

    Так как $x_n \in D(g \circ f)$, то
    \begin{enumerate}
        \item {
            $\exists$ $f(x_n)$ $\Rightarrow$ $x_n \in D(f)$
        }
        \item {
            $\exists$ $g(f(x_n))$ $\Rightarrow$ $f(x_n) \in D(g)$ 
        }
    \end{enumerate}

    \columnbreak

    Так как $a \in D(g \circ f)$, то
    \begin{enumerate}
        \item {
            $a \in D(f)$ 
        }
        \item {
            $f(a) \in D(g)$
        }
    \end{enumerate}

\end{multicols}

Исходная функция непрерывна, поэтому из $g(f(x_n)) \underset{n \rightarrow \infty}{\rightarrow} g(f(a))$
следует, что $(g\circ f)(x_n) \underset{n \rightarrow \infty}{\rightarrow} (g \circ f)(a)$

$\Rightarrow$ $g \circ f$ непрерывна в точке $a$, где $a$ - любая точка из $D(g \circ f)$ $(\blacksquare)$

\subsubsection{Теорема Коши о промежуточном значении}

Если $f(x)$ определена и непрерывна на $[a; b]$, то $\forall c$: $\min\{a; b\} < c < \max\{a; b\}$
$\exists$ $d \in (a; b)$: $f(d) = c$

\paragraph{Лемма о промежуточном значении} Пусть $\phi: [a; b] \longrightarrow \mathbb{R}$,
причем $\sgn{\phi(a)} \neq \sgn{\phi(b)}$ ($\phi(a) > 0 > \phi(b)$ или $\phi(a) < 0 < \phi(b)$);
тогда обязательно $\exists$ $d \in (a; b)$: $\phi(d) = 0$ 

\paragraph{Доказательство леммы} Будем считать, что $\phi(a) < 0 < \phi(b)$

\textit{Опять таки не хватает рисуночка}

Возьмем отрезок $[a_0; b_0]$, где $a_0 = a$ и $b_0 = b$ (тогда $\phi(a_0) < 0 < \phi(b_0)$),
и разделим его пополам (середина в точке $\zeta $)

Возможны несколько случаев:
\begin{itemize}
    \item {
        $\phi(\zeta) = 0$, тогда $c = \zeta$ и $\phi(c) = 0$ (конец)
    }
    \item {
        $\phi(\zeta) < 0$, тогда $a_1 = \zeta$, а $b_1 = b_0$ ($\phi(a_1) < 0 < \phi(b_1)$)
    }
    \item {
        $\phi(\zeta) > 0$, тогда $b_1 = \zeta$, а $a_1 = a_0$ ($\phi(a_1) < 0 < \phi(b_1)$)
    }
\end{itemize}

В результате мы или сразу найдем нужную точку $c$, или найдем $[a_1; b_1] \subseteq [a_0; b_0]$

Если результат еще не найден, то проделываем тоже самое с отрезком $[a_1; b_1]$; тут снова
возможны два варианта: или мы найдем точку $c$, или - отрезок $[a_2; b_2] \subseteq [a_1; b_1]$

Организовав этот выбор бесконечное число раз, мы получим последовательность вложенных отрезков:
$[a_0; b_0] \supseteq [a_1; b_1] \supseteq \dots \supseteq [a_n; b_n]$, при этом
$(b_n - a_n) \underset{n \rightarrow \infty}{\rightarrow} 0$, а $\phi(a_n) < 0 < \phi(b_n)$

Тогда по теореме о вложенных отрезках $\exists$ $c$: 
$c = \lim\limits_{n \rightarrow \infty}a_n = \lim\limits_{n \rightarrow \infty}b_n$ и также
$c \in \overset{\infty}{\underset{n = 0}{\cap}}[a_n; b_n]$
$\Rightarrow$ $a_n \rightarrow c \leftarrow b_n$

Т.к. $\phi(x)$ непрерывна, то $0 > \phi(a_n) \underset{n \rightarrow \infty}{\rightarrow} \phi(c)$
и $0 < \phi(b_n) \underset{n \rightarrow \infty}{\rightarrow} \phi(c)$

$\Rightarrow$ $0 \leq \phi(c) \leq 0$ $\Rightarrow$ $\phi(c) = 0$ (по теореме о предельном переходе
в неравенстве) $(\blacksquare)$

\paragraph{Вернемся к теоереме} Рассмотрим $\phi(x) = f(x) - c$, непрерывную на $[a; b]$,
тогда или $\phi(a) < 0 < \phi(b)$, или $\phi(b) < 0 < \phi(a)$ и по лемме $\exists$ $d$:
$\phi(d) = 0$

Тогда так как $f(x) = \phi(x) + c$, то $f(d) = \phi(d) + c = 0 + c = c$ $(\blacksquare)$

\subsubsection{Теорема Вейерштрасса об ограниченности}

Если $f(x)$ определена и непрерывна на $[a; b]$, то она ограничена на $[a; b]$

\paragraph{Доказательство} Предположим, что $f(x)$ неограничена на $[a; b]$, тогда
найдется такая $\{x_n\} \subseteq [a; b]$, что $f(x_n) \underset{n \rightarrow \infty}{\rightarrow} \infty$

По теореме Больцано-Вейерштрасса найдня некая сходящаяся подпоследовательность для $\{x_n\}$:
$\{\underset{\in [a; b]}{x_{n_k}} \underset{k \rightarrow \infty}{\rightarrow} \underset{\in [a; b]}{c}\}$
(причем $c \in [a; b]$)

$\Rightarrow$ $f(x_{n_k}) \underset{k \rightarrow \infty}{\rightarrow} f(c)$ ($f(x)$ непрерывная) и что
$f(x_n) \underset{k \rightarrow \infty}{\rightarrow} \infty$, а так, как $f(x_{n_k}) = f(x_n)$, то
$f(c) = \infty$, что невозможно
$\Rightarrow$ $f$ ограничена $(\blacksquare)$

\subsubsection{Теорема Вейерштрасса об экстремумах}

Если $f(x)$ непрерывна и определена на $[a; b]$, то она достигает своих локальных экстремумов
на $[a; b]$
\begin{equation}
    \begin{aligned}
        \exists\: \alpha\in [a; b]: \:
        \underset{x \in [a; b]}{\min f(x)} = f(\alpha), & &
        \exists\: \beta \in [a; b]: \:
        \underset{x \in [a; b]}{\max f(x)} = f(\beta) 
    \end{aligned}
\end{equation}

\paragraph{Доказательство} (существования максимума, дальше аналогично)

По теоереме Вейерштрасса об ограниченности $f(x)$ ограничена сверху $\Rightarrow$
$\exists$ $A$: $f(x) \leq A$ $\forall x \in [a; b]$

Пусть $E = \{f(x): x\in [a; b]\}$, тогда $E \leq A$ и $E \subseteq \mathbb{R}$

Получается, что $E$ - ограниченное сверху множество $\Rightarrow$ $\exists$
$\underset{x \in [a; b]}{\sup f(x)} = c$ (причем $c \leq A$, $c \in \mathbb{R}$). 
$\forall \varepsilon > 0$ $E \nleq c - \varepsilon$ $\Rightarrow$
$\forall n \in \mathbb{N}^{*}$ $E \nleq c - \frac{1}{n}$

$\forall n \in \mathbb{N}^{*}$ $\exists$ $x_n \in E$: $f(x_n) > c - \frac{1}{n}$ $\Rightarrow$
$\exists$ $x_n \subseteq [a; b]$: $c - \frac{1}{n} < f(x_n) \leq c$ $\Rightarrow$ по
теореме Больцано-Вейерштрасса $\exists$ $\{x_{n_k}\} \underset{k \rightarrow \infty}{\rightarrow} d$
$\Rightarrow$
$f(x_{n_k}) \underset{k \rightarrow \infty}{\rightarrow} f(d)$

$\Rightarrow$ $\exists$ $d \in [a; b]$: $f(d) = c = \sup \{f(x): \forall x \in [a; b]\}$
$\Rightarrow$ $f(d) = \max f(x)$ 

\subsubsection{Равномерная непрерывность}

Говорят, что $f(x)$ равномерно непрерывна на $\mathbb{M}$, если она
на нем определена и $\forall \alpha_n$ и $\beta_n \in \mathbb{M}$ таких, что
$\alpha_n - \beta_n \underset{n \rightarrow \infty}{\rightarrow} 0$ $\Rightarrow$
$f(\alpha_n) - f(\beta_n) \underset{n \rightarrow \infty}{\rightarrow} 0$

\paragraph{Пример} $f(x) = kx + b$ (непрерывна на $\mathbb{R}$)

Если $\alpha_n - \beta_n \underset{n \rightarrow \infty}{\rightarrow} 0$, то
$f(\alpha_n) - f(\beta_n) = k\alpha_n + b - k\beta_n - b = k\alpha_n - k\beta_n =
k(\alpha_n - \beta_n)
\underset{n \rightarrow \infty}{\rightarrow} 0$

\subsubsection{Теорема Кантора}

Всякая $f(x)$ непрерывная на $[a; b]$ равномерно непрерывна на $[a; b]$

\end{document}

