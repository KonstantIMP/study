% ДЗ на 19.12.2022
% Михедов Константин БИБ224

% Статья на А4, 12 кегель
\documentclass[a4paper,12pt]{article}

% Поиск по скомпилированному PDF
\usepackage{cmap}
% Кодировка выходного текста
\usepackage[T2A]{fontenc}
% Кодировка исходного текста
\usepackage[utf8]{inputenc}
% Поддержка необходимых языков
\usepackage[english,russian]{babel}

% Поддержка изображений
\usepackage{graphicx}
% Путь до внешних изображений
\graphicspath{ {./figures/}}

% Умная запятая
\usepackage{icomma}

% Ссылки на электронные ресурсы
\usepackage{hyperref}
% Настройка внешнего вида ссылок
\hypersetup{
  % Отключить прямоугольную рамку
  pdfborder={0 0 0},
  % Включить цветные ссылки
  colorlinks=true,
  % Цвет для ссылок на веб-ресурсы
  urlcolor=blue,
  % Цвет внутренних ссылок
  linkcolor=black
}

% Дополнительная математика
\usepackage{amsmath,amsfonts,amssymb,amsthm,mathtools}
% Показывать номера только у тех выржений, на которые кто-то ссылается
\mathtoolsset{showonlyrefs=true}

% Дополнительные символы
\usepackage{mathbbol}

% Правильное оформление
% Настройка отступов
\usepackage[left=2cm,right=1cm,top=2cm,bottom=2cm]{geometry}
% Настройка шрифта
\usepackage{fontspec}
\setmainfont{Times New Roman}
% Настройка межстрочных интервалов
\usepackage{setspace}
\onehalfspacing
% и межабзацных
\usepackage{parskip}
\setlength{\parindent}{1.25cm} 

% Красная строка
\usepackage{indentfirst}
\setlength{\parindent}{1.25cm} 

% Корректное положение рисунков?
\usepackage{float}

% Расположение блоков относительно текста
\usepackage{wrapfig}

\begin{document}
    \paragraph{Работу выполнил} Михедов Константин Константинович БИБ224

    \section*{Задача }
 
    \paragraph{Условие} Невесомый блок укреплен на вершине двух наклонных плоскостей, составляющих с горизонтом углы
    $\alpha = 30^{\circ}$ и $\beta = 45^{\circ}$. Гири $A$ и $B$ равного веса $P_1 = P_2 = 1$кгс соединены нитью,
    перекинутой через блок. Коэффициенты трения гири $A$ и $B$ о наклонной плоскости $f_1 = f_2 = 0.1$. Найдите
    ускорение, с которым движутся гири, и натяжение нити (трением в блоке пренебречь)

    \begin{figure}[H]
        \minipage{0.2\textwidth}
            \begin{tabular}{p{0.8\textwidth} |}
                \textbf{Дано} \\
                \hline
                $\alpha = 30^{\circ}$ \\
                $\beta = 45^{\circ}$ \\
                $P_1 = P_2 = 1$кгс \\
                $f_1 = f_2 = 0.1$ \\
                \hline
                $T_1=$ ? $T_2=$ ? \\
                $a = $ ?
            \end{tabular}
            %\vspace{1.5cm}
        \endminipage
        \minipage{0.79\textwidth}
            \textbf{Решение}
            
            \begin{figure}[H]
                \centering
                \begin{picture}(300, 110)
                    \put(0,20){\line(1,0){300}}

                    \put(30,20){\line(2,1){120}}
                    \put(150,80){\line(1,-1){60}}

                    \put(150,80){\circle{25}}
                    \put(150,80){\circle*{4}}

                    \qbezier(60,20)(60,25)(50,30)
                    \qbezier(185,20)(190,30)(195,35)

                    \put(65,25){$\alpha$}
                    \put(175,25){$\beta$}

                    \put(70,40){\line(-2,3){10}}
                    \put(100,55){\line(-2,3){10}}
                    \put(60,55){\line(2,1){30}}

                    \put(170,60){\line(1,1){15}}
                    \put(190,40){\line(1,1){15}}
                    \put(185,75){\line(1,-1){20}}

                    \put(92.5,62){\line(2,1){47.5}}
                    \put(180,65){\line(-1,1){20}}

                    \put(80,55){\circle*{3}}

                    \put(187.5,57.5){\circle*{3}}

                    \thicklines

%                    \put(80,55){\vector(1,-2){12}}
                    \put(75,75){$\vec{N_1}$}
                    \put(80,55){\vector(-1,2){12}}
                    \put(100,75){$\vec{T_1}$}
                    \put(80,55){\vector(0,-1){30}}
                    \put(85,30){$\vec{P_1}$}
                    \put(80,55){\vector(2,1){30}}
                    \put(80,55){\vector(-2,-1){30}}
                    \put(40,50){$\vec{F_{\text{тр}1}}$}

                    \put(187.5,57.5){\vector(1,-1){20}}
                    \put(187.5,57.5){\vector(-1,1){20}}
                    \put(187.5,57.5){\vector(1,1){20}}
                    %\put(187.5,57.5){\vector(-1,-1){20}}
                    \put(187.5,57.5){\vector(0,-1){30}}

                    \put(160,86){\circle*{3}}
                    \put(160,85){\vector(1,-1){10}}
                    \put(140,86){\circle*{3}}
                    \put(140,86){\vector(-2,-1){20}}

                    \put(200,80){$\vec{N_2}$}
                    \put(210,40){$\vec{F_{\text{тр}2}}$}
                    \put(170,40){$\vec{P_2}$}
                    \put(175,80){$\vec{T_2}$}

                    \put(160,90){$\vec{T_2^{'}}$}
                    \put(125,90){$\vec{T_1^{'}}$}
                \end{picture}
            \end{figure}
        \endminipage
    \end{figure}

    Так как нить невесома и нерастяжима (по условию), то $|\vec{a_1}| = |\vec{a_2}| = a$

    Блок невесомый $\Rightarrow$ его масса равна нулю, а так как $J = \int_mr^2dm$,
    то $J = 0$. Распишем основной закон динамики вращательного движения:
    \begin{equation}
        \vec{M_{T_1}} + \vec{M_{T_2}} = J\varepsilon = 0
    \end{equation}

    Рассмотрим вращение по часовой стрелке как положительное, а против часовой - как отрицательное,
    тогда верно:
    \begin{equation}
        rT_2 - rT_1 = 0 \Leftrightarrow T_2 = T_1 \Rightarrow T = T_2 = T_1
    \end{equation}

    Рассмотрим первое тело: введем систему координат, где OX сонаправлена $\vec{T_1}$,
    а OY сонаправлена $\vec{N_1}$:
    \begin{equation}
        \begin{cases}
            OX: \:\: T - F_{\text{тр}1} - \frac{P_1}{\sin{\alpha}} = m_1a \\
            OY: \:\: N_1 - \frac{P_1}{\cos\alpha} = 0
        \end{cases}
        \overset{F_{\text{тр} = fN}}{\Leftrightarrow} \begin{cases}
            N_1 = \frac{P_1}{\cos\alpha} \\
            T = \frac{P_1a}{g} + \frac{P_1f_1}{\cos\alpha} + \frac{P_1}{\sin{\alpha}}
        \end{cases}
    \end{equation}

    Тогда $T = P_1(\frac{a}{g} + \frac{f_1}{\cos\alpha} + \frac{1}{\sin{\alpha}})$. Тогда рассмотрим второе тело,
    запишем для него второй закон Ньютона для OX сонаправленной с $\vec{T_2}$ и OY сонаправленной
    с $\vec{N_2}$
    \[\begin{cases}
        T_2 - F_{\text{тр}2} - \frac{P_2}{\sin\beta} = m_2a \\
        N_2 - \frac{P_2}{\cos\beta} = 0
    \end{cases} \Leftrightarrow \begin{cases}
        N_2 = \frac{P_2}{\cos\beta} \\
        T = \frac{P_2a}{g} + \frac{f_2P_2}{\cos\beta} + \frac{P_2}{\sin\beta}
    \end{cases}\]

    Получаем следующую систему:
    \begin{equation}
        \begin{cases}
            T = P_2(\frac{a}{g} + \frac{f_2}{\cos\beta} + \frac{1}{\sin\beta}) \\
            T = P_1(\frac{a}{g} + \frac{f_1}{\cos\alpha} + \frac{1}{\sin\alpha})
        \end{cases}
    \end{equation}

    \begin{flushright}
        \textbf{Ответ:} $a = 2.45$м/с$^2$, $T_1 = 1.29$Н, $T_2 = 1.47$Н
    \end{flushright}
\end{document}
