% ДЗ на 06.12.2022
% Михедов Константин БИБ224

% Статья на А4, 12 кегель
\documentclass[a4paper,12pt]{article}

% Поиск по скомпилированному PDF
\usepackage{cmap}
% Кодировка выходного текста
\usepackage[T2A]{fontenc}
% Кодировка исходного текста
\usepackage[utf8]{inputenc}
% Поддержка необходимых языков
\usepackage[english,russian]{babel}

% Поддержка изображений
\usepackage{graphicx}
% Путь до внешних изображений
\graphicspath{ {./figures/}}

% Умная запятая
\usepackage{icomma}

% Ссылки на электронные ресурсы
\usepackage{hyperref}
% Настройка внешнего вида ссылок
\hypersetup{
  % Отключить прямоугольную рамку
  pdfborder={0 0 0},
  % Включить цветные ссылки
  colorlinks=true,
  % Цвет для ссылок на веб-ресурсы
  urlcolor=blue,
  % Цвет внутренних ссылок
  linkcolor=black
}

% Дополнительная математика
\usepackage{amsmath,amsfonts,amssymb,amsthm,mathtools}
% Показывать номера только у тех выржений, на которые кто-то ссылается
\mathtoolsset{showonlyrefs=true}

% Дополнительные символы
\usepackage{mathbbol}

% Правильное оформление
% Настройка отступов
\usepackage[left=2cm,right=1cm,top=2cm,bottom=2cm]{geometry}
% Настройка шрифта
\usepackage{fontspec}
\setmainfont{Times New Roman}
% Настройка межстрочных интервалов
\usepackage{setspace}
\onehalfspacing
% и межабзацных
\usepackage{parskip}
\setlength{\parindent}{1.25cm} 

% Красная строка
\usepackage{indentfirst}
\setlength{\parindent}{1.25cm} 

% Корректное положение рисунков?
\usepackage{float}

% Расположение блоков относительно текста
\usepackage{wrapfig}

\begin{document}
    \paragraph{Работу выполнил} Михедов Константин Константинович БИБ224

    \section*{Задача }
 
    \paragraph{Условие} Тело массой $m_1 = 0.25$кг, соединенное невесомой нитью посредством блока
    (в виде полого тонкостенного цилиндра) с телом массов $m_2 = 0.2$кг, скользит по поверхности
    горизонтального стола. Масса блока $m = 0.15$кг, а коэффициент трения $f$ тела о поверхность
    равен $0.2$. Пренебрегая трением в подшипникам определите: ускорение $a$, с которым будут
    двигаться эти тела и силы натяжения $T_1$ и $T_2$ нити по обе стороны блока.

    \begin{figure}[H]
        \minipage{0.2\textwidth}
            \begin{tabular}{p{0.8\textwidth} |}
                \textbf{Дано} \\
                \hline
                $m_1 = 0.25$кг \\
                $m_2 = 0.2$кг \\
                $m = 0.15$кг \\
                $f = 0.2$ \\
                \hline
                $T_1=$ ? $T_2=$ ? \\
                $a = $ ? \\
                $ $ \\ $ $ \\
            \end{tabular}
            %\vspace{1.5cm}
        \endminipage
        \minipage{0.79\textwidth}
            \begin{wrapfigure}{r}{0.40\textwidth}
                \begin{picture}(10, 150)
                    \put(0,100){\line(1,0){90}}
                    \put(90,100){\line(0,-1){30}}

                    \put(110,95){\circle{30}}

                    \thicklines
                    \put(90,95.5){\line(1,0){20}}
                    \put(90,95){\line(1,0){20}}
                    \put(90,94.5){\line(1,0){20}}
                    \thinlines

                    \put(20,100){\line(0,1){20}}
                    \put(20,120){\line(1,0){30}}
                    \put(50,120){\line(0,-1){20}}

                    \put(50,111.5){\line(1,0){60}}
                    \put(126.5,95){\line(0,-1){50}}

                    \put(116.5,45){\line(1,0){20}}
                    \put(116.5,15){\line(1,0){20}}
                    \put(116.5,45){\line(0,-1){30}}
                    \put(136.5,45){\line(0,-1){30}}

                    \put(10,70){\vector(1,0){40}}
                    \put(45,60){$x$}
                    \put(15,75){\vector(0,-1){40}}
                    \put(20,40){$y$}

                    \put(0,120){$\vec{F_{\text{тр}}}$}
                    \put(40, 125){$\vec{N}$}
                    \put(60, 117){$\vec{T_1}$}
                    \put(90, 117){$\vec{T_{b1}}$}
                    \put(40, 90){$m_1\vec{g}$}
                    \put(130, 0){$m_2\vec{g}$}
                    \put(130, 50){$\vec{T_2}$}
                    \put(130, 80){$\vec{T_{b2}}$}

                    \linethickness{0.35mm}
                    \put(35,111.5){\vector(1,0){30}}
                    \put(35,111.5){\vector(0,1){20}}
                    \put(35,111.5){\circle*{3}}
                    \put(35,111.5){\vector(-1,0){30}}
                    \put(35,111.5){\vector(0,-1){20}}

                    \put(126.5,30){\vector(0,1){30}}
                    \put(126.5,30){\circle*{3}}
                    \put(126.5,30){\vector(0,-1){30}}

                    \put(126.5,95){\circle*{3}}
                    \put(110,111){\vector(-1,0){20}}
                    \put(110,111){\circle*{3}}
                    \put(126.5,95){\vector(0,-1){20}}
                \end{picture}
            \end{wrapfigure}
            \textbf{Решение}
            
            Так как по условию нить не рвется при движении блоков, можем считать ее нерастяжимой.
            Тогда по 3 закону Ньютона верно, что:
            \begin{equation}
                \begin{aligned}
                    |\vec{T_{b1}}| = |\vec{T_1}| &= T_1 \\
                    |\vec{T_{b1}}| = |\vec{T_1}| &= T_2 \\
                    |\vec{a_1}| = |\vec{a_2}| &= a 
                \end{aligned}
            \end{equation}
        \endminipage
    \end{figure}

    Распишем второй закон Ньютона для тела массой $m_1$:
    \begin{equation}
        \left.
        \begin{aligned}
            OX: & \:\:\: T_1 - F_{\text{тр}} = m_1a \\
            OY: & \:\:\: m_1g - N = 0
        \end{aligned} \:\:\: \right\} \underset{F_{\text{тр}} = fN}{\Leftrightarrow}
        \begin{cases}
            N = m_1g\\
            T_1 - fN = m_1a
        \end{cases} \Rightarrow
        T_1 = m_1(a + fg)
    \end{equation}

    И также распишем второй закон Ньютона для тела массой $m_2$:
    \begin{equation}
        \left.
        \begin{aligned}
            OX: & \:\:\: 0 = 0 \\
            OY: & \:\:\: m_2g - T_2 = m_2a
        \end{aligned}
        \right\} \Leftrightarrow T_2 = m_2(g - a)
    \end{equation}
    
    Рассмотрим момент инерции блока (который представляет собой тонкостенный цилиндр):
    \begin{equation}
        J = \int_0^m{r^2dm} = r^2m, \text{ где } r \text{ - радиус блока}
    \end{equation}

    Так как мы пренебрегаем трением подшипника и нити о блок, можем основным законом
    динамики вращательного движения:
    \begin{equation}
        \sum{M} = J\varepsilon \Leftrightarrow M_{T_2} + M_{T_1} = J\varepsilon
    \end{equation}

    Трение нити о блок не учитывается, тогда:
    \begin{equation}
        \varepsilon = \frac{d\omega}{dt} = \frac{d\varphi}{dt^2} = \frac{\frac{dS}{r}}{dt^2} = 
        \frac{\int vdt}{rdt^2} = \frac{\int \int a dtdt}{rdt^2} = \frac{at^2}{rdt^2} = \frac{a}{r}
    \end{equation}

    Тогда, так как $M_{T_2} = T_2r$ и $M_{T_1} = -T_1r$, можно утрвеждать, что:
    \begin{equation}
        M_{T_2} + M_{T_1} = J\varepsilon \Leftrightarrow r(T_2 - T_1) = mr^2 \cdot \frac{a}{r}
        \Leftrightarrow T_2 - T_1 = ma
    \end{equation}

    Получаем систему с термя неизвестными и тремя уравнениями
    \begin{equation}
        \begin{cases}
            T_2 - T_1 = ma \\
            T_2 = m_2(g - a) \\
            T_1 = m_1(a + fg)
        \end{cases} \Rightarrow
        \begin{cases}
            a = \frac{g(m_2 - m_1f)}{m + m_1 + m_2} \\
            T_1 = \frac{m_1g(fm + m_2(1 + f))}{m + m1 + m2} \\
            T_2 = \frac{m_2g(m + m_1(1 + f))}{m + m_1 + m_2}
        \end{cases}
    \end{equation}

    Выполняем расчеты:
    \begin{equation}
        a = \frac{9.8\text{м/c}^2(0.2\text{кг} - 0.25\text{кг}\cdot 0.2)}{0.15\text{кг} + 0.2\text{кг} + 0.25\text{кг}} = 2.45 \text{м/с}^2
    \end{equation}
    \begin{equation}
        T_1 = \frac{0.25\text{кг}9.8\text{м/с}^2(0.2\cdot 0.15\text{кг} + 0.2\text{кг}(1 + 0.2))}{0.15\text{кг} + 0.2\text{кг} + 0.25\text{кг}} \approx 1.29\text{Н}
    \end{equation}
    \begin{equation}
        T_2 = \frac{0.2\text{кг}9.8\text{м/с}^2(0.15\text{кг} + 0.25\text{кг}(1 + 0.2))}{0.15\text{кг} + 0.2\text{кг} + 0.25\text{кг}} = 1.47\text{Н}
    \end{equation}

    \begin{flushright}
        \textbf{Ответ:} $a = 2.45$м/с$^2$, $T_1 = 1.29$Н, $T_2 = 1.47$Н
    \end{flushright}
\end{document}
