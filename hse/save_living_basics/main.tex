% Домашняя работа по БЖД - "Вышка для бизнес-единорогов"
% Выполнил Михедов Константин - НИУ ВШЭ, МИЭМ БИБ224

% Тип документа: статья, размер бумаги - A4, написано 14 кегелем
\documentclass[a4paper,12pt]{article}

% Поиск по скомпилированному PDF
\usepackage{cmap}
% Кодировка выходного текста
\usepackage[T2A]{fontenc}
% Кодировка исходного текста
\usepackage[utf8]{inputenc}
% Поддержка необходимых языков
\usepackage[english,russian]{babel}

% Поддержка изображений
\usepackage{graphicx}
% Путь до внешних изображений
\graphicspath{ {./figures/}}

% Умная запятая
\usepackage{icomma}

% Ссылки на электронные ресурсы
\usepackage{hyperref}
% Настройка внешнего вида ссылок
\hypersetup{
    % Отключить прямоугольную рамку
    pdfborder={0 0 0},
    % Включить цветные ссылки
    colorlinks=true,
    % Цвет для ссылок на веб-ресурсы
    urlcolor=blue,
    % Цвет внутренних ссылок
    linkcolor=black
}

\title{UBA}
\author{Михедов Константин Константинович}
\date{23 Сентября 2022}

\begin{document}
    % Титульная страница
    \begin{titlepage}
        \begin{center}
            \vspace*{1.5cm}

            \Huge
            \textbf{Universal Banking Application}

            \vspace{0.5cm} \large
            \textbf{Вышка для бизнес-единорогов}

            \vspace{1.5cm} \normalsize
            \textbf{Михедов Константин Константинович}

            \vfill

            \includegraphics[width=0.2\textwidth]{hse_logo}

            \vspace{1cm} \footnotesize
            Национальный Исследовательский Университет "Высшая Школа Экономики" \\
            Московский Институт Электроники и Математики им. А. Н. Тихонова \\
            Информационная Безопасность, БИБ224 \\
            Москва, 2022
        \end{center}
    \end{titlepage}

    \section{Введение}

    В современном мире люди всё чаще используют пластиковые
    банковские карты для оплаты чего-либо.

    Такой подход имеет огромный ряд преимуществ: не нужно постоянно
    носить с собой купюры и монеты различных номиналов, можно совершать
    покупки в Интернете, а также переводить
    деньги куда и кому угодно.
    
    Однако, и у использования банковских карт есть недостатки.

    \subsection{Немного статистики}

    Согласно данным за 2021 год, население РФ составляло приблизительно 147 млн. человек (источник \ref{res:strana2020} на стр. \pageref{res:strana2020}).

    Также, исходя из данных Национальной Платежной Системы (источник
    \ref{res:cbr} на стр. \pageref{res:cbr}) за тот же 2021 год,
    количество активных дебетовых и кредитных карт, выданных
    российскими банками, составило чуть более 310 млн. штук.

    Получается, что на человека, проживающего
    на территории Российской Федерации, приходится в среднем
    2.5 банковские карты.

    \subsection{В чем же проблема?}

    Чем больше денежных средств перетекает из реальности
    на виртуальные банковские счета, тем труднее становится за
    ними следить.

    Большинство банковских компаний предоставляют
    специальные сервисы для учета имеющихся у
    клиентов финансов: личный кабинет на сайте банка, специальное
    мобильное приложение и т.п.

    Однако не у каждого человека все карты открыты в одном и том
    же банке; наоборот, часто зарплата начисляется на один счёт,
    сбережения хранятся на другом, а сам человек для ежедневных
    нужд использует третий.

    Происходит это по разным причинам: некоторые привыкли
    пользоваться только одной картой, другие платят той, на которую
    за покупку начислятся больше бонусных баллов.

    Получается так, что карт несколько, счетов тоже,
    деньги приходят на один, списываются с другого. Уследить за
    всеми подобными перемещениями может быть очень нелегко, в
    особенности потому, что для получения полной информации
    придется открыть не одно приложение или сайт.

    \subsection{Решение}

    Чтобы упростить процесс ведения финансов в условиях тотального
    перехода денежных средств на пластиковые карты, необходимо
    создать единую платформу, позволяющую узнавать информацию
    о всех, принадлежащих человеку, банковских счетах единовременно.

    Такой платформой может стать, например, мобильное приложение.
    Пользователи получат возможность подключать к нему банки, клиентами которых 
    являются, и управлять уже открытыми там счетами и картами,
    а также создавать новые.

    \section{Описание продукта}

    Как уже говорилось, в качестве продукта, решающего
    поставленную проблему, предлагается создать мобильное
    приложение, имеющее следующий базовый функционал:

    \begin{itemize}
        \item {
            Возможность подключать к приложению различные банки,
            чтобы абсолютно все финансы пользователя отображались
            в одном месте
        }
        \item {
            Управлять счетами(открывать, закрывать и замораживать)
            и картами(блокировать и выпускать новые) в тех банках,
            которые были привязаны к приложению ранее
        }
        \item {
            Переводить средста между своими счетами,
            отправлять деньги другим людям, а также оплачивать
            что-либо(например услуги ЖКХ или мобильную связь)
        }
    \end{itemize}

    Базовые функции данного приложения схожи с теми, что
    практически каждый банк предоставляет своим пользователям.
    Однако ключевой целью данного продукта является
    избавление людей от множества различных интерфейсов управления
    финансами и предоставления единого окна доступа.

    Именно по этой причине приложение получило название
    ''Universal Banking Application'', или сокращенно - \underline{UBA}.

    \section{Стратегия развития и состояние рынка}

    \subsection{Целевая аудитория}
    
    В первую очередь приложение ориентированно на тех, кто имеет
    счета в нескольких различных российских банках и активно пользуется хотя
    бы двумя из них. Под данное описание \textbf{не} подходят
    дети, у которых скорее всего или нет своих банковских карт, или
    они оформлены на родителей, и большая часть пожилых, уже вышедших на пенсию людей. 

    А вот дееспособным гражданам от 18 до 45 лет
    оно точно пригодится. Нахоядщиеся в данном возрастном
    диапазоне в большинстве случаев работают или учатся,
    поэтому практически всегда имеют стипендиальную
    или зарплатную карту. Такие люди стараются обзавестись
    недвижимым имуществом, обычно при помощи ипотек, и скорее всего открывают счёт в тех банках, где ставка на них минимальна.   

    Приложение привлечет и тех людей, которые активно охотятся
    за различного рода акциями: повышенными кэшбеками, процентами на остаток и т.п.
    Они часто заводят карты различных
    банков, чтобы при каждой оплате нашлась такая, на которую придетё
    больший бонус.

    \subsection{Похожие продукты}

    \begin{enumerate}
        \item {
            Сервис \href{zenmoney.ru}{Дзен-мани}
            \footnote{
                Все названия кликабельны и ведут на веб-страницы описанных продуктов
            }   

            предлагает пользователям удобный интерфейс 
            для ведения личных финансов: учет расходов, доходов,
            планирование бюджета и многое другое.

            В нем имеется возможность интеграции с банковскими
            приложениями: в истории операций автоматически будут
            будут учитываться транзакци с привязанных счетов.

            Однако Дзен-мани не предоставляет пользователю
            возможности управления банковскими продуктами и 
            не умеет соверщать переводы и вообще какие-либо
            банковские операции.

        }
        \item {
            Мобильное приложение \href{https://vk.com/ubank}{UBank}

            хранит данные о банковских
            картах пользователя на устройстве и позволяет использовать их
            для оплаты каких-либо услуг через Интернет.

            На данный момент оно больше не функционирует. Его разработка и поддержка
            была прекращена в связи с низкой популярностью, вызванной
            взятием комиссий и нестабильной работой.
        }
    \end{enumerate}

    \subsection{Повышенная актуальность}

    В связи с санкциями, многие приложения российских
    банков были удалены из стандартных магазинов приложений на
    Android и IOS устройствах(источник \ref{res:habr_block} на стр.
    \pageref{res:habr_block}).

    Из-за тех же санкций в России были заблокированы платежные
    системы Visa и Mastercard (источник \ref{res:forbes_block} на стр. \pageref{res:forbes_block}),
    поэтому очень многие выпустили дополнительные карты МИР, тем самым
    добавив еще один счет, за которым также необходимо следить.

    По этим причинам предложенный продукт будет особенно актуален в текущей.

    \section{Производственный план}

    Чтобы охватить сразу все основные платформы мобильных
    устройств (Android и IOS), приложение должно быть написано 
    на языке программирования Dart с использованием библиотеки Flutter.

    Такой подход позволит минимизировать время разработки,
    потому что кодовая база будет всего одна и не понадобиться
    писать отдельные приложения для разных платформ.

    Интерфейс продукта должен быть похож на стандартный для
    любого банковского приложения, так пользователю будет легче
    в нем разобраться.

    \subsection{Интеграция с банками}

    Наиболее интересный вопрос заключается в том, как подключить
    тот или иной банк к приложению, каким образом получать информацию
    об открытых счетах и отправлять запросы на открытие новых?

    Некоторые банки предоставляют специальный API для выполнения
    таких операций. В их число входят например Сбер, ВТБ,
    Тинькофф, Альфа-банк и другие (источники \ref{res:api_sber},
    \ref{res:api_vtb}, \ref{res:api_tinkoff}, \ref{res:api_alfa} на
    стр. \pageref{res:api_alfa}).

    Построить интеграцию с такими банками просто и быстро, потому что
    существует специальный интерфейс и документация на него. Необходимо
    только создать единый способ работы с этими API.

    Однако не все так просто. Существуют банки, которые не предоставляют
    подобных сервисов и инструментов. Чтобы получить информацию из
    них, необходимо будет вручную разобраться, как работает их
    собственное приложение или сайт, чтобы описать такой же способ
    связи с инфраструктурой банка.

    Это не очень быстро и удобно, но если говорить о первоначальных
    версиях приложения или MVP, то хватит и поддержки тех банков,
    которые предоставляют API, так как они являются самыми
    наиболее часто используемыми (источник \ref{res:gazeta}
    на стр. \pageref{res:gazeta}).

    %\subsection{Аспекты безопасности}
    %
    %Приложение, имеющее возможность управлять финансовыми
    %продуктами, должно быть хорошо защищено, чтобы
    %не допустить кражу денежных средств клиента.
    %
    %Чтобы не подвергать данные опасности, предлагается хранить
    %их только на устройстве, на которое установлено
    %приложение, и только в зашифрованном виде.
    %
    %Вход в приложение должен осуществляться по пин-коду или
    %биометрии (отпечатку пальца или FaceID).

    \section{Монетизация приложения}

    \subsection{Интегрированная реклама}

    Каждый банк старается различными способами привлечь как
    можно больше клиентов. Некоторые предлагают выгодные ставки,
    другие - бесплатное обслуживание.

    И чтобы рассказать об этом потенциальным клиентам, используется
    различного рода реклама. Так как приложение предназначено для
    управления финансами, то его пользователи в большей мере будут
    заинтересованы в банковских услугах.

    Таким образом можно будет предлагать банкам рекламное место,
    интегрированное в приложение, например, небольшую прямоугольную
    область на главном экране. Так как пользователи банка заинтересованы
    в финансах, то банки заинтересованы в этих пользователях и 
    следовательно будут стараться выкупить это рекламное место раньше
    других.

    \subsection{Премиум-функции}

    У пользователей не возникнет желание платить за базовый
    функционал, потому что в таком случае им будет выгоднее воспользоваться
    продуктами, которые предоставляет им каждый отдельный банк. Поэтому основные функции должны быть доступны всем.

    Но всегда можно предложить какой-либо сверх-функционал, например:
    интеграцию с популярными мессенджерами, отправку уведомлений
    об операциях, улучшенную аналитику и прогноз трат на месяц.

    За подобного рода функции уже можно брать дополнительную
    плату, например в форме месячной или годовой подписки.

    \section{Планы развития}

    Необходимо постоянно развивать интеграцию с банковскими
    сервисами: увеличивать стабильность, добавлять поддержку
    новых банков, реализовывать иные финансовые услуги. 

    Также можно добавить в приложение синхронизацию с биржей
    или криптовалютными кошельками, тем самым расширив его целевую аудиторию.
    
    \section{Источники информации}

    \begin{enumerate}
        \item {
            Сайт Центрального Банка Российской Федерации \\
            URL: \href{https://cbr.ru/statistics/nps/psrf/}{cbr.ru}
            (дата обращения: 24.09.2022)
        } \label{res:cbr}
        \item {
            Сайт Всероссийской Переписи Населения 2021 года \\
            URL: \href{https://www.strana2020.ru/novosti/opublikovany-pervye-dannye-perepisi-naseleniya/}{strana2020.ru}
            (дата обращения: 24.09.2022)
        } \label{res:strana2020}
        \item {
            Сайт Хабрахабр \\
            URL: \href{https://habr.com/ru/news/t/657567/?ysclid=l8hh4c5m1482599735}{habr.com}
            (дата обращения: 25.09.2022)
        } \label{res:habr_block}
        \item {
            Сайт Forbes \\
            URL: \href{https://www.forbes.ru/finansy/458195-visa-i-mastercard-priostanovili-rabotu-v-rossii-cto-eto-znacit}{forbes.ru}
            (дата обращения: 25.09.2022)
        } \label{res:forbes_block}
        \item {
            Сайт SberBuisnessApi \\
            URL: \href{https://developers.sber.ru/portal/products/sberbusiness-api?ysclid=l8hhwz7ikt281121203}{developers.sber.ru}
            (дата обращения: 25.09.2022)
        } \label{res:api_sber}
        \item {
            Сайт Tinkoff ID \\
            URL: \href{https://tinkoff.github.io/tinkoff-id/debit/}{tinkoff.github.io}
            (дата обращения: 25.09.2022)
        } \label{res:api_tinkoff}
        \item {
            Сайт ВТБ API \\
            URL: \href{https://developer.vtb.ru/?ysclid=l8hhqbnckn188330288}{developer.vtb.ru}
            (дата обращения: 25.09.2022)
        } \label{res:api_vtb}
        \item {
            Сайт Alfa API \\
            URL: \href{https://alfaapi.github.io/specification/#alfa-api}{alfaapi.github.io}
            (дата обращения: 25.09.2022)
        } \label{res:api_alfa}
        \item {
            Сайт Газета.ру \\
            URL: \href{https://www.gazeta.ru/business/news/2020/10/07/n_15047653.shtml?ysclid=l8hitymd0612071717}{gazeta.ru}
            (дата обращения: 25.09.2022)
        } \label{res:gazeta}
    \end{enumerate}

\end{document}
